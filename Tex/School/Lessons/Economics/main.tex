\documentclass[12pt,a4paper]{article}
\usepackage{tpl}
\dbegin[25 января 2020 г.]{Уроки в школе 179}{Экономика, 11 класс}{Симко Зинаида Викторовна}

\hdr{Риск в экономике}

Пусть человек делает апельсиновый сок. Если апельсины подорожают, то он получит меньше заработка, т.к. купит меньше апельсинов.\\

\definition{Фьючерс} сделка, когда покупатель и продавец договариваются о сделке, которая будет в будущем. Помогает предотвратить проблему выше.

\definition{Спекулянт} брокер, который перепродаёт активы.

\definition{Инвестор} брокер, который покупает активы, чтобы ими пользоваться.

Очевидно, что биржа рискованная: если кто-то много заработал, то кто-то другой много потерял. Это похоже на казино, только от скилла зависит гораздо больше.\\

\hdr{Альтернативные издержки}

Вы хотите посмотреть фильм. Вы можете посмотреть его в кино сейчас (за деньги + потратив время на движение), а можете посмотреть дома на 2 месяца позже. Деньги и время на перемещение в кино --- это альтернативная издержка.

Среди бизнесменов был девиз <<качество-деньги>>, сейчас <<время-деньги>>. То есть раньше за минимальные деньги люди ехали в Германию покупать товар, а сейчас заходят в соседнюю китайскую лавку.\\

\hdr{Невидимая рука рынка}

Это идея о том, что когда кто-то действует на рынке в личных интересах, он продвигает не только себя, но и всё общество. Например, когда Эдисон изобрёл новые лампочки, они стали всем полезны и их начали покупать. Поскольку они всем нужны, кто-то другой начинает их производить. Тогда Эдисон вынужден сделать лампочки дешевле, за счёт этого он помогает другим людям и делает свой товар доступнее.\\

\hdr{Экономические пузыри}

Пузыри возникают, когда акции становятся выше реальной цены, потому что брокеры покупают слишком много акций. Первый такой случай --- в Голландии на XVII веке. Там были фьючерсы на покупку тюльпанов. В итоге цены на тюльпаны долго росли, а потом рухнули. Другой известный пример --- в конце XX века компания рассказала методы заработка в интернете, им всем верили и акции компании росли. Затем оказалось, что методы не работают, и компания обанкротилась.\\

\hdr{Мальтусианская ловушка}

Мальтус сделал теорию о том, что мы попадём в ловушку: население растёт по экспоненте, а продукты растут медленнее, значит, мы все умрём. Это много раз опровергали. Мы были в ловушке в XVIII веке, но в результате войн (и в результате развития) у нас снизилась рождаемость, а еды мы стали производить больше. Ещё такая же идея есть для нефти и пр.

В теории есть и верные идеи. Например, Diminishing Returns на доходность. То есть первые $n$ работников приносят $nk$ дохода, но дальше это уменьшается.

\end{document}
