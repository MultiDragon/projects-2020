\documentclass[12pt,a4paper]{article}
\usepackage{tpl}
\dbegin[5 декабря 2019 г.]{Уроки в школе 179}{Современная литература}{Олзоева Софья Андреевна}

С 1890-х годов начинается модернизм. Его основные представители --- Д.Джойсс, М.Пруст, Ф.Кафка; в России --- Чехов, Бунин, Куприн. В то же самое время во Франции начинается эпоха <<проклятых поэтов>> --- Ш.Бодлера, А.Рембо, П.Верлена.\\

\shdr{Музы Бодлера}
\begin{enumerate}
	\item Больная муза:
	\begin{itemize}
		\item Мёртвое тело, со следами разложения.
		\item Автор хочет её <<оживить>> (часть греческого пантеона!)
	\end{itemize}
	\item Продажная муза:
	\begin{itemize}
		\item Поэзия для увеселения публики.
		\item У автора не было денег, поэтому оно такое.
	\end{itemize}
\end{enumerate}\vskip10pt

\hdr{Символизм}
Жан Мореас написал свой манифест, в котором провозгласил главную идею символизма: <<искусство ради искусства>>; поэзия отражает личность художника. Главные символисты России --- Бальмонт, Мережковский, З. Гиппиус (его жена), В. Брюсов (младосимволист), Ф. Сологуб.

\definition{Символ} что угодно, чтобы скрыть тайну. <<Вся жизнь покрыта тайной, мы не можем всю её знать и можем только символически описывать.>>

\end{document}
