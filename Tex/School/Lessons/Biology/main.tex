\documentclass[12pt,a4paper]{article}
\usepackage{tpl}
\usepackage{chemfig}
\dbegin[7 сентября 2019 г.]{Уроки в школе 179}{Биология, 11 класс}{Карцев Александр Леонидович}

\definition{Биологическая эволюция (БЭ)} необратимое историческое изменение живой природы.\\

\shdr{БЭ сопровождается}
\begin{itemize}
	\item Изменением генетического состава (генофондов) популяций организмов.
	\item Формированием адаптаций.
	\item Развитием сложных форм из простых.
	\item Изменениями биоценозов и биосферы в целом.
\end{itemize}

\textbf{Сущность эволюции} --- в непрерывных адаптациях организмов к разнообразным и постоянно меняющимся условиям среды.\\

\hdr{Развитие эволюционных идей в додарвиновский период}

До конца XVIII века развитие биологии определяли идеи креационизма.

\definition{Креационизм} идея, по которой:
\begin{enumerate}
	\item Все виды организмов возникли в результате одного акта творения (длившегося 6 дней).
	\item Все виды не связаны между собой и неизменны.
	\item Все виды обладают изначальной (данной Богом) абсолютной целесообразностью строения.
\end{enumerate}

\hdr{Сторонники креационизма}
\begin{itemize}
	\item \textbf{Карл Линней} ($1707-1778$):
	\begin{itemize}
		\item Описал около 10000 растений. Также описывал минералы, животных и т.п.
		\item Основатель систематики (принцип иерархии, бинарная номенклатура --- \textit{Pinus Pinea}).
		\item Неизменность видов: <<разновидности появляются скрещиванием, но видов ровно столько, сколько сделал Бог>>.
	\end{itemize}
	\item \textbf{Шарль Бонне} (в 1762 ввёл термин <<эволюция>>).
	\begin{itemize}
		\item Поддерживался преформизма: <<внутри какой-то гаметы есть микрочеловек, который разворачивается при связи с другой гаметой>>.
	\end{itemize}
	\item \textbf{Жорж Кюнье} ($1769-1832$):
	\begin{itemize}
		\item Постоянство видов.
		\item Смена населения Земли --- результат не эволюции, а катастрофы.
		\item Появление новых организмов --- результат расселения или повторного акта творения.
	\end{itemize}
\end{itemize}

\definition{Трансформизм} представление об изменяемости организмов и превращении одних видов в другие.\\

\textbf{Жан Батист Ламарк}:
\begin{itemize}
	\item Создал более совершенную систему животного мира (в частности, разделение на позвоночных и беспозвоночных).
	\item Предполагал происхождение человека от обезьяны.
	\item Считал, что в процессе эволюции сложные организмы произошли от простых:
	\begin{enumerate}[I]
		\item Инфузории и полипы.
		\item Лучистые и черви.
		\item Насекомые и паукообразные.
		\item Ракообразные и кольчатые черви (почему?).
		\item Усоногие (подмножество ракообразных) и моллюски.
		\item Рыбы, рептилии, птицы, млекопитающие.
		\item Человек.
	\end{enumerate}
	\item Условия среды --- важный фактор эволюции.
	\item Эволюция --- очень длительный процесс.
	\item \textit{Ошибки}:
	\begin{enumerate}
		\item \textbf{Принцип градации}: усложнение организмов --- результат изначального внутреннего стремления к совершенствованию.
		\item Организмы целесообразно приспосабливаются к изменениям среды:
		\begin{itemize}
			\item Путём прямого приспособления (для организмов без нервной системы).
			\item Путём упражнения или неупражения органов (для организмов с нервной системой).
		\end{itemize}
		\item Отрицал реальное существование видов: считал, что если посадить семечко лютика обыкновенного на сухой холм, то вырастет лютик едкий. Сам не проверял.
		\item Низшие формы жизни постоянно самозарождаются из живой природы.
	\end{enumerate}
\end{itemize}
\vspace{10pt}

\textbf{Чарльз Дарвин (1809 --- 1882)}:
\begin{enumerate}
	\item Учение об изменчивости.
	\item Учение об искусственном отборе.
	\item Учение об естественном отборе.
\end{enumerate}

\hdr{Учение об изменчивости}

\shdr{Виды наследственности}
\begin{itemize}
	\item Групповая, или ненаследственная.
	\item Определённая, или наследственная.
	\item Неопределённая наследственная.
	\item Почковая (мутации возникают в клетках тела).
\end{itemize}

\hdr{Учение об искусственном отборе}
\begin{enumerate}
	\item Изменчивость и наследственность дают материал для отбора.
	\item Отбор и размножение небольшого числа <<лучших>> наиболее совершенных особей и устранение всех менее совершенных форм.
	\item Накопление положительных изменений, т.к. для размножения выбирают только лучших в каждом новом поколении.
\end{enumerate}

\hdr{Учение об естественном отборе}
\begin{enumerate}
	\item Изменчивость признаков и свойств характерна для всех организмов.
	\item Способность всех видов размножаться в геометрической прогрессии.
	\item Большинство появляющихся на свет особей гибнет в борьбе за существование:
	\begin{itemize}
		\item Внутривидовой.
		\item Межвидовой.
		\item Борьбе с неблагоприятными абиотическими факторами.
	\end{itemize}
	\item Больше шансов выжить и оставить после себя потомство имеют особи, отличающиеся от других полезными свойствами (\textbf{принцип ЕО}).
	\item В процессе естественного отбора могут возникнуть новые разновидности и виды. В процессе этого происходит дивергенция (расхождение) признаков особей, имеющих общего предка.
\end{enumerate}

\hdr{Сравнение искусственного и естественного отбора}
\vspace{10pt}
\begin{tabular}{ccccc}\\
	\textbf{Отбор} & \textbf{Материал} & \textbf{Фактор} & \textbf{Сущность} & \textbf{Результат} \\\hline\\
	 & Организмы с& & Оставление для размно- & Выведение новых и\\
	\textbf{ИО} & мутационными & Человек & жения особей с полезными & улучшение существу-\\
	 & изменениями & & для человека признаками & ющих пород и сортов\\\\
	 & & & Преимущественное выжи- & Адаптация орг.\\
	\textbf{ЕО} & --//-- & Природа & вание особей с полезными & и появление \\
	 & & & признаками для вида & новых видов\\\\
\end{tabular}

\shdr{Предпосылки к эволюционному процессу}
\begin{itemize}
	\centering
	\item Перепроизводство потомков;
	\item Ограниченность потомков.
\end{itemize}\centerline{$\downarrow$}

\centerline{Конкуренция}

\centerline{$\downarrow$}

\centerline{Борьба за существование}

\centerline{$\downarrow$}

\centerline{Естественный отбор}

\centerline{$\downarrow$}

\begin{itemize}
	\centering
	\item Адаптации организмов;
	\item Многообразие органического мира;
	\item Историческое развитие сложных организмов.
\end{itemize}

\shdr{Доказательства эволюции:}
\begin{enumerate}
	\item Биохимические --- сходство организмов на молекулярном уровне:
	\begin{itemize}
		\item Элементарный химический состав.
		\item Принцип генетического кодирования (триплеты нуклеотид).
		\item Принцип биосинтеза белков и нуклеиновых кислот.
		\item Механизм энергетического обмена.
	\end{itemize}
	\item Цитологические --- сходство на клеточном уровне:
	\begin{itemize}
		\item Почти все организмы имеют клеточное строение.
		\item Сходство строения, функционирования и деления клеток.
	\end{itemize}
	\item Эмбриологические.
	\begin{itemize}
		\item Сходство строения и развития зародышей позвоночных.
	\end{itemize}
	\item Морфологические.
	\begin{itemize}
		\item Промежуточные формы.
		\item Гомологичные органы.
		\item Аналогичные органы.
		\item Рудименты.
		\item Атавизмы.
	\end{itemize}
	\item Палеонтологические.
	\begin{itemize}
		\item Переходные формы (напр., птеродактили).
	\end{itemize}
	\item Биогеографические.
	\begin{itemize}
		\item Различие форм в разных изолированных местах и климатических зонах.
	\end{itemize}
	\item Молекулярно-генетические.
	\begin{itemize}
		\item Механизм репликации ДНК.
		\item Копирование нуклеотидов из ДНК в РНК.
		\item Транслирование нуклеотидов РНК в аминокислоты белков.
	\end{itemize}
\end{enumerate}

\shdr{Формы борьбы за существование}
\begin{enumerate}
	\item Внутривидовая.
	\begin{itemize}
		\item Конкуренция.
		\item Каннибализм.
	\end{itemize}
	\item Межвидовая.
	\begin{itemize}
		\item Конкуренция.
		\item Хищничество.
		\item Паразитизм.
	\end{itemize}
	\item Абиотическая.
	\begin{itemize}
		\item Борьба с засухой.
		\item Борьба с жарой и холодом.
		\item и пр.
	\end{itemize}
\end{enumerate}

\shdr{Формы эволюционного отбора}
\begin{enumerate}
	\item Движущий отбор.
	\begin{itemize}
		\item При изменении условий существования предпочтение в пользу особей, имеющих отклонение от средней нормы. Например, устойчивость к химикатам.
	\end{itemize}
	\item Стабилизирующий отбор.
	\begin{itemize}
		\item При стабильных условиях среды предпочтение в пользу средних по признкакам особей.
	\end{itemize}
	\item Дизруптивный отбор, или рассекающий.
	\begin{itemize}
		\item При изменении условий существования предпочтение в пользу особей двух крайних форм. Например, при сильных ветрах сохраняются птицы либо с сильными крыльями, либо с рудиментарными.
	\end{itemize}
	\item Половой отбор.
	\begin{itemize}
		\item Происходит всегда, предпочтение в пользу особей более яркого внешнего вида. Например, павлины.
	\end{itemize}
\end{enumerate}

\definition{Дрейф генов} непредвиденное изменение отношения количеств генов.

К дрейфу генов могут привести явления колебания численности особей популяции (популяционных волн).

\shdr{Классификация популяционных волн}
\begin{enumerate}
	\item Периодические колебания численности коротко живущих организмов.
	\item Непериодические колебания численности, зависящие от сложной комбинации разных факторов.
\end{enumerate}

Приспособленность --- результат действия факторов эволюции.

Одним из важнейших итогов эволюции является адаптация организмов к условиям среды обитания.

\shdr{Три типа адаптаций}
\begin{itemize}
	\item Морфологические:
	\begin{itemize}
		\item Форма тела --- обтекаемая, листовидная и т.п.
		\item Окраска тела --- маскирующая, расчленяющая (зебры).
		\item \textbf{Мимикрия} --- подражание защищённым животным. Например, молочная змея выглядит почти так же, как коралловая, но первая безопасная, вторая ядовитая.
	\end{itemize}
	\item Поведенческие, или этологические:
	\begin{itemize}
		\item Приспособительное поведение. Например, забота о потомстве.
		\item Угрожающие позы.
	\end{itemize}
	\item Физиологические:
	\begin{itemize}
		\item Метаболическая вода.
		\item Теплокровновность и хладнокровность.
	\end{itemize}
\end{itemize}

Как бы ни были совершенны адаптации организмов, они всё же относительны.

Одним из важнейших этапов эволюции является образование новых видов. Поэтому вид --- одно из центральных понятий систематики и эволюции.\\

\shdr{Этапы развития представлений о виде}
\begin{enumerate}
	\item Виды изначально постоянны (креационизм).
	\item Организмы, составляющие виды, отличаются большой изменчивостью (трансформизм).
	\item Виды реально существуют, но меняются с течением времени (дарвинизм).
\end{enumerate}\vspace{10pt}

\definition{Вид} исторически сложившееся множество особей, которые:
\begin{itemize}
	\item обладают общими морфологическими и физиологическими признаками;
	\item населяют определённый ареал (\textit{область распространения вида});
	\item имеют общее происхождение;
	\item способны к скрещиванию с образованием плодовитого потомства.
\end{itemize}

\shdr{Причины сохранения единства вида}
\begin{itemize}
	\item Взаимодействие между отдельными особями.
	\begin{itemize}
		\item Родители-дети.
		\item Самцы-самки.
	\end{itemize}
	\item Репродуктивная изоляция от особей другого вида.
\end{itemize}

\shdr{Основы биологической изоляции между видами}
\begin{itemize}
	\item Различие в брачном поведении.
	\item Морфологические различия.
	\item Биологические различия.
	\item Генетические различия (кол-во и строение хромосом).
	\item Бесплодие гибридов.
\end{itemize}

\shdr{Критерии вида}
\begin{itemize}
	\item Генетический критерий.
	\item Морфологический критерий.
	\item Географический критерий.
	\item Экологический критерий.
	\item Этологический критерий (поведение).
	\item Физиолого-биохимический критерий.
\end{itemize}

\hdr{Структура вида}
Реально в природе особи любого вида внутри ареала распределены неравномерно.

\definition{Популяции} частично или полностью изолированные группировки особей одного вида.

\shdr{Причины существования популяций}
\begin{itemize}
	\item Благоприятные условия существования.
	\item Возможность свободного скрещивания.
\end{itemize}

\definition{Популяция} элементарная биологическая часть вида, способная к историческим изменениям. Поэтому это элементарная единица эволюции.

\newpage
\hdr{Возникновение жизни на Земле}

\shdr{Этапы химической эволюции}

\begin{enumerate}
	\item \chemfig{CO}, \chemfig{CO_2}, \chemfig{H_2}, \chemfig{CH_4}, \chemfig{NH_3}, \chemfig{H_2S}, \chemfig{HCN}.
	\item Аминокислоты, азотистые основания, сахара и прочие растворимые орг. в-ва.
	\item Образование \textbf{полимеров} --- нуклеиновых кислот, белков, полисахаридов.
	\item Образование комплексов из белков и нуклеиновых кислот --- \textbf{коацерватов}, обладающих некоторыми свойствами живого (обмен веществ, воспроизведение себе подобных).
	\item Образование биологических мембран (\textbf{пробионтов}) из молекул липидов и белков.
\end{enumerate}\vspace{10pt}

\hdr{Происхождение человека}

\definition{Антропология} наука, изучающая происхождение человека, становление его как вида.\\

\shdr{Положение Человека разумного в системе органического мира}

\begin{itemize}
	\item Империя \textbf{Клеточные}.
	\item Надцарство \textbf{Эукариоты}.
	\item Царство \textbf{Животные}.
	\item Подцарство \textbf{Многоклеточные}.
	\item Тип \textbf{Хордовые}.
	\item Подтип \textbf{Позвоночные}.
	\item Класс \textbf{Млекопитающие (звери)}.
	\item Подкласс \textbf{Плацентарные}.
	\item Отряд \textbf{Приматы} (К. Линней, 1758).
	\item Подотряд \textbf{Настоящие обезьяны}.
	\item Секция \textbf{Узконосые обезьяны Старого Света}.
	\item Надсемейство \textbf{Гоминоиды} (человекоподобные).
	\item Семейство \textbf{Гоминииды} (люди).
	\item Род \textbf{Человек}.
	\item Вид \textbf{Человек разумный} (\texttt{Homo Sapiens Sapiens}).
\end{itemize}

\shdr{Сходство человека и млекопитающих}

\begin{itemize}
	\item Толстая кожа (например, относительно птиц) --- большой подкожный слой жировой клетчатки, потовые, сальные и млечные железы.
	\item Диафрагма.
	\item Развитие плода в матке.
	\item Молочные и постоянные зубы трёх типов в ячейках челюстей.
	\item Ушные раковины и слуховые кости.
\end{itemize}

\shdr{Сходство человека и приматов}

\begin{itemize}
	\item Пятипалые конечности с ногтями.
	\item Одна пара молочных желёз.
	\item Противопоставленный первый палец кисти.
	\item Подвижные ключица и плечевой сустав.
\end{itemize}

\shdr{Сходство человека и человекообразных обезьян}

\begin{itemize}
	\item Сходство в строении и количестве хромосом.
	\item 4 группы крови ($0,A,B,AB$) и резус-фактор.
	\item Папиллярный узор.
	\item Сходные болезни.
	\item Богатая мимика (выражение чувств).
	\item Использование и изготовление орудий труда.
\end{itemize}

\hdr{Отличия человека от человеческих обезьян}

\begin{enumerate}
	\item Связанные с прямохождением:
	\begin{itemize}
		\item Большое заголовочное отверстие черепа снизу.
		\item Позвоночник с 4 изгибами.
		\item Плоская грудная клетка.
		\item Особенное строение тазобедренного и коленного суставов.
		\item Более развитые большие ягодичные и икроножные мышцы.
		\item Сводчатая стопа.
	\end{itemize}
	\item Связанные с трудовой деятельностью:
	\begin{itemize}
		\item Большой головной мозг ($S\sim 1200 \text{км}^2$).
		\item Небольшие челюсти и клыки.
		\item Подвижная кисть с длинным большим пальцем.
	\end{itemize}
	\item Связанные с высшей нервной деятельностью:
	\begin{itemize}
		\item Сознание, рассудочная деятельность.
		\item Абстрактное мышление.
		\item Изготовление орудий труда с помощью других предметов.
		\item Членораздельная речь.
	\end{itemize}
	\item Связанные с трудом в коллективе.
\end{enumerate}

\newpage
\hdr{Этапы антропогенеза}

\begin{enumerate}
	\item Отделение приматов от насекомоядных.
	\item Разделение полуобезьян и настоящих обезьян.
	\item Появление общего предка у человекообразных обезьян --- дриопитека ($25-9$ млн лет назад).
	\item Разделение гоминиид и человекообразных обезьян.
	\item Эволюция гоминиид.
\end{enumerate}\vspace{10pt}

\shdr{Главные стадии эволюции гоминиид}\vspace{10pt}

\noindent\begin{tabular}{p{2.2cm}p{4cm}p{1.5cm}p{5.4cm}p{4cm}}
\textbf{Стадия} & \textbf{Представители} & \textbf{Возраст} & \textbf{Признаки антропогенеза} & \textbf{Распространение}\\\hline\\
	Парантропы & Австралопитек афарский, африканский, могучий, Бойса & $9-1.5$ млн лет & Прямохождение, объём мозга не более $600 \text{ см}^3$, использует натуральные инструменты & Африка\\\\
	Архантропы & Человек умелый (\texttt{Homo Habilis}) & $2.6-1.5$ млн лет & Объём мозга $650-800 \text{ см}^3$, изготовление инструментов другими инструментами & Африка\\\\
			   & Человек прямоходящий (\texttt{Homo Erectus}) & $1.6-0.2$ млн лет & Речь отсутствует, более совершенные орудия, \mbox{\textbf{поддержание огня}} & Африка, Индонезия, Китай, Грузия, Германия\\\\
	Палеоан- тропы & Неандерталец \texttt{Homo Sapiens Neandertalensis} & $300-28$ тысяч лет & Объём мозга до $1400 \text{ см}^3$, широкие крепкие кости, \mbox{добывание огня}, коллективная деятельность, забота о ближних & Более 400 находок, Европа\\\\
	% Кромань- онцы & \texttt{Homo Sapiens Sapiens} & $10^5-0$ лет & Другое строение ДНК & Большая часть \mbox{Земли}\\
\end{tabular}


\end{document}
