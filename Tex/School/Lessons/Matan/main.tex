\documentclass[12pt,a4paper]{article}
\usepackage{tpl}

\newcommand{\Meth}[1]{\textbf{\textcolor{blue}{Методы #1.}} }
\newcommand{\Comb}[1]{\textbf{\textcolor{green}{Комбинаторика #1.}} }
\newcommand{\Num}[1]{\textbf{\textcolor{red}{Числа #1.}} }
\newcommand{\Al}[1]{\textbf{\textcolor{violet}{Алгебра #1.}} }
\newcommand{\Gr}[1]{\textbf{\textcolor{orange}{Графы #1.}} }
\newcommand{\Mat}[1]{\textbf{\textcolor{cyan}{Матанализ #1.}} }

\dbegin[7 апреля 2020 г.]{Уроки в школе 179}{Математический магазин 15 мая}{Саня}

\Meth1 Докажите, что существует бесконечно много арифметических прогрессий из 57 чисел, произведение чисел в каждой из которых является точной 179-й степенью.\\

\Meth2 В правильном $21$-угольнике $6$ вершин покрашены в красный, а $7$ других --- в синий. Обязательно ли найдутся два равных треугольника, все три вершины одного из которых красные, а другого --- синие?\\

\Meth3 \\

\Meth4 Существует ли натуральное число, у которого нечётное количество чётных делителей и чётное количество нечётных?\\

\Meth5 Докажите, что в правильный пятиугольник можно так вписать квадрат, что его вершины будут лежать на четырёх сторонах пятиугольника.\\

\Comb{1*} Каждая сторона $\triangle ABC$ разделена на 8 равных отрезков. Сколько существует треугольников с вершинами в точках деления, у которых ни одна из сторон не параллельна какой-то стороне исходного треугольника?\\

\Comb{2*} На сколько нулей оканчивается десятичная запись числа $11^{100}-1$?\\

\Comb{3*} В классе учатся 16 человек. Известно, что если выбрать 3 случайных человека из класса, то с вероятностью $\frac{1}{16}$ они все будут мальчиками. С какой вероятностью 4 случайно выбранных ученика будут девочками?\\

\Comb4 Можно ли расставить в целочисленных точках плоскости натуральные числа так, чтобы были использованы все натуральные числа, и на каждой прямой, не проходящей через центр координат, расстановка была периодической?\\

\Comb5 Вычислите сумму $1\cdot 2+2\cdot 2^2+3\cdot 2^3+\ldots +n\cdot 2^n$.\\

\Num{1*} Пушкин родился 6 июня 1799 года. Какой это был день недели? \\

\Num{2*} Найдите наименьшее $m$ такое, что в $m$-ричной системе счисления одновременно выполняется:

\begin{enumerate}
	\item Число делится на 5 тогда и только тогда, когда сумма его цифр делится на 5.
	\item Число делится на 7 тогда и только тогда, когда число, составленное из его двух последних цифр, делится на 7.
\end{enumerate}

\Num3 \\

\Num4 \\

\Num5 Докажите, что если $C_n^k$ делится на $n$ при всех $1\leq k\leq n-1$, то $n$ простое.\\

\newpage
\Al{1*} Найдите наименьшее натуральное $K$ такое, что ни при каких натуральных $a,b,c,d$ не выполняется $K=\frac{2^a-2^b}{2^c-2^d}$. \\

\Al2 Докажите, что для натуральных $n$ выполняется \[
	\frac{1\cdot 3\cdot 5\cdot \ldots \cdot (2n-1)}{2\cdot 4\cdot 6\cdot\ldots \cdot  2n}\leq \frac{1}{\sqrt{2n-1}}
.\]\\

\Al3 \\

\Al4 Придумайте многочлен с целыми коэффициентами, у которого есть корень \[
	\sqrt[5]{2+\sqrt3}+\sqrt[5]{2-\sqrt3}
.\]

\Al5 \\

\Gr1 На плоскости отмечено несколько красных и несколько синих точек. Некоторые из точек соединены между собой линиями, причём каждая точка, независимо от цвета, соединена с количеством синих точек, на 2 большим, чем красных. При этом имеется 12 линий, оба конца которых красные и 38 линий, оба конца которых синие. Сколько всего точек отмечено?\\

\Gr{2*} Сколько существует неизоморфных деревьев на 6 вершинах?\\

\Gr3 Дед барона Мюнхгаузена построил квадратный замок, разделил его на 9 квадратных залов и в центральном разместил арсенал. Отец барона разделил каждый из восьми оставшихся залов на 9 равных квадратных холлов и во всех центральных холлах устроил зимние сады. Сам барон разделил каждый из 64 свободных холлов на 9 равных квадратных комнат и в каждой из центральных комнат устроил бассейн, а остальные сделал жилыми. Барон хвастается, что ему удалось обойти все жилые комнаты, побывав в каждой по одному разу, и вернуться в исходную (в каждой стене между двумя соседними жилыми комнатами проделана дверь). Могут ли слова барона быть правдой? \\

\Gr{4*} Мяч для космического волейбола --- это многогранник, грани которого --- правильные треугольники и пятиугольники. Известно, что у него удвоенное число вершин на 20 больше числа граней. Какое количество его граней могут быть пятиугольниками?\\

\Gr5 Из чисел от 10 до 99, не кратных 10 (т.е. 11, 12, \ldots , 19, 21 и т.д.) выбрали 37 чисел. Докажите, что среди них можно выбрать 5 таких, что модуль разности любых двух из них больше 10 и не делится на 10.\\

\Mat{1*} Кузнечик прыгает по прямой. Вначале он прыгнул на 1 метр вправо. Каждым следующим шагом он меняет направление прыжка и прыгает в точку, находящуюся вдвое ближе к своему текущему положению, чем к своему предыдущему положению. Существует ли предел у последовательности точек, по которым прыгает кузнечик? Если да, найдите его.\\

\Mat2 \\

\Mat3 При каких $A$ и $B$ у многочлена $Ax^{2020}+Bx^{2019}-1$ есть двукратный корень $x=1$?\\

\Mat4 \\

\Mat5 Существует ли такая функция $f:\mathbb R\to\mathbb R$, что $f(\cos x)+f(\sin x)=\sin x$?\\

\end{document}
