\documentclass[12pt,a4paper]{article}
\usepackage{tpl}
\usepackage{verbatim}
\dbegin{Кружок 7 класса, школа 179}{Решения олимпиады}

\z \textit{(3 балла)} Гном Алин привез Деду Морозу 5 мешков подарков. Чтобы их завязать, он взял веревку, но забыл ножницы. Тогда он сделал на веревке пометки, где надо ее разрезать, чтобы получить нужное количество равных кусков. Потом гном Балин привез еще 2 мешка подарков и, не заметив пометок Алина, сделал на веревке новые пометки. Потом гном Валин привез еще 4 мешка подарков и, не заметив предыдущих пометок, тоже сделал свои пометки. Потом пришла Снегурочка с ножницами и, не долго думая, разрезала веревку по всем меткам. Сколько кусков у нее получилось?

\s Алин сделал 4 пометки, Балин 6 пометок, Валин 10 пометок. Они не совпадают, т.к. верёвка разделена на 5, 7 и 11 частей, а эти числа взаимно простые (\textbf{школьник должен это проверить!}). Значит, всего гномы сделали 23 пометок, значит, Снегурочка сделала 24 кусков верёвки.\QEDA\\

\z \textit{(3 балла)} Игровой автомат <<Новогодний бандит>> может умножать счет игрока на 3, прибавлять к нему 3, делить на 3 (если делится нацело). Хакер Вася вычислил, что автомат можно заставить выполнять действия в нужном порядке, пиная его с разных сторон. Вася знает, что в автомате всего 10 монет. Автомат не выдает выигрыш, если в нем недостаточно монет. Как нужно пинать автомат, чтобы, опустив 1 монету, забрать в конце все?

\s Вначале умножим счёт на 3, получим 3 монеты. Потом прибавим 3 10 раз, получим 33 монеты. Потом поделим на 3, получим 11 монет и заберём их.\QEDA\\

\z \textit{(5 баллов)} На Новый Год фирма «Санта-Клаус и 4 оленя» выпустила наборы шоколадок в виде 5 разных фигурок тетриса. Снеговик Вик хочет собрать из них плитку $6\times6$ так, чтобы в ней был целиком использован хотя бы один набор. Сможет ли он это сделать?

\s Сможет, решение проверяется по рисунку школьника.\QEDA\\

\z \textit{(3 балла)} Белоснежке на Новый Год подарили набор из 27 одинаковых кубиков. Она хочет склеить из них кубик $3\times3\times3$. Для этого она капает по капельке меда между любыми двумя кубиками, которые соприкасаются поверхностями. Сколько капелек меда ей понадобится?

\s В каждой плоскости между слоями кубиков надо использовать 9 капель мёда. Всего 3 направления, в каждом по 2 таких плоскости, всего $9\cdot3\cdot2=54$ капли.\QEDA\\

\z \textit{(4 балла)} Коробка из-под кубиков имеет форму куба без крышки. Сможет ли Белоснежка разрезать ее на 4 части так, чтобы потом сложить из них квадратный коврик?

\s Сможет, решение проверяется по рисунку школьника.\QEDA\\

\z \textit{(5 баллов)} Гномы раздавали детям подарки. Каждый гном раздал на 7 подарков меньше, чем все остальные вместе, но не меньше двух подарков. Сколько всего подарков они раздали?

\s Рассмотрим какого-то гнома. Пусть он раздал $a$ подарков, а все вместе $N$. Тогда $a=N-a-7$, откуда $2a=N-7$ --- постоянное число. Значит, все гномы раздали поровну подарков. Пусть гномов $x$. Тогда $N=xa$ и $7=a(x-2)$. Заметим, что $7$ простое и раскладывается на произведение двух чисел только одним способом, $7=7\cdot1$. Т.к. $a>1$, то $x-2=1$ и $a=7$, откуда $N=21$.\QEDA\\

\z \textit{(5 баллов)} Снеговик Вик любит такие числа, что к какому бы натуральному числу его справа ни приписать, результат будет делиться на это приписанное число. Чтобы не растаять, Вик должен каждый месяц придумывать новое любимое число. Вик умеет считать от 1 до 1000. Сможет ли он продержаться до следующего Нового Года?

\s Да. Например, числа $1,2,5,10,20,25,50,100,200,250,500,1000$ подходят.\QEDA\newpage

\z \textit{(6 баллов)} Киоск «Новогодний Шуллер» продает конфеты за талончики и жетончики. Чтобы купить конфету, можно заплатить 2 талончика и получить 3 жетончика сдачи, а можно заплатить 5 жетончиков и получить 3 талончика сдачи. Гном Дулин не признает талончиков; он запасся жетончиками и купил 50 конфет, причем в конце у него остались только жетончики. Сколько жетончиков он потратил?

\s Пусть он сделал $a$ операций первого типа и $b$ операций второго типа. Тогда, с одной стороны, $a+b=50$, а с другой стороны, он получил $3b$ талончиков, а потратил $2a$. В конце у него кончились талончики, значит, $3b=2a$. Отсюда $b=20,a=30$. Он потратил 100 жетончиков, а получил обратно 90. Ответ: 10 жетончиков.\QEDA\\

\z \textit{(6 баллов)} Перед началом представления Белоснежка, Снегурочка и Красная Шапочка рассаживали школьников на стулья, а Дед Мороз ими руководил. Он подсчитал, что всего было занято 60 стульев (на одном из них сидел он сам), Белоснежка и Снегурочка в сумме усадили 15 детей, Снегурочка и Красная Шапочка --- 18 детей, Белоснежка и Красная Шапочка --- 20 детей. Дед Мороз немного подслеповат, но каждое из этих четырех чисел отличается от правильного не более чем на 13. Сколько детей усадила Белоснежка?

\s Заметим, что количество школьников было не меньше, чем $59-13 = 46$. С другой стороны, Белоснежка и Снегурочка усадили не больше, чем 28 детей, Снегурочка и Красная Шапочка --- не более, чем 31, а Белоснежка и Шапочка --- не более, чем 33. Сложим эти три числа, тогда удвоенное количество усаженных детей не больше, чем 92. То есть число школьников одновременно не больше и не меньше, чем 46. Поскольку мы в самом крайнем случае, Снегурочка и Красная Шапочка усадили ровно 31, то есть Белоснежка усадила 15 детей.\QEDA\\

\z \textit{(8 баллов)} Барабан <<Елки Чудес>> имеет 30 номеров, на каждом лежит подарок. Лиса Алиса и Кот Базилио подошли к барабану и выбрали два каких-то подарка. Если Алиса сможет повернуть барабан так, чтобы перед ней и Базилио опять оказались номера с подарками, то они их забирают и Алиса снова вращает барабан, иначе они уходят. Лисе Алисе удалось так хитро поворачивать барабан, что они с Базилио смогли забрать все подарки. Докажите, что с таким же успехом Алиса могла просто поворачивать барабан каждый раз на два номера.

\s Вначале докажем, что если расстояние между Алисой и Базилио чётное, то поворот на 2 номера позволяет им забрать все призы. Действительно, пусть изначально Алиса стоит у номера 1. Тогда она заберёт приз на 1 месте, потом на 3, на 5, на 7 и т.п. и ни один из этих призов не заберёт Базилио, потому что расстояние между любыми двумя из них нечётное. Аналогично Алиса не заберёт призы Базилио. Значит, они заберут все призы.

Теперь докажем, что если расстояние между ними нечётное, то они, как ни вращая барабан, не смогут забрать все призы. Действительно, рассмотрим все призы на нечётных полях. Алиса и Базилио либо одновременно берут какие-то два из этих 15 призов, либо какие-то два из остальных 15. Значит, они возьмут чётное количество <<нечётных>> призов. Но всего <<нечётных>> призов нечётно, противоречие.\QEDA\\

\begin{comment}
\begin{tikzpicture}
	\newcommand{\q}[2]{\qcsegment{#1}{#2}}
	\qsetscale{0.3}
	\qgrid{15}{15}
	\qcoord A{5}{15}
	\qcoord B{10}{15}
	\qcoord C{10}{10}
	\qcoord D{15}{10}
	\qcoord E{15}{5}
	\qcoord F{10}{5}
	\qcoord G{10}{0}
	\qcoord H50
	\qcoord I55
	\qcoord J05
	\qcoord K0{10}
	\qcoord L5{10}
	\q AB\q BC\q CD\q DE\q EF\q FG\q GH\q HI\q IJ\q JK\q KL\q LA
	\q AF\q CJ
\end{tikzpicture}

\begin{tikzpicture}
	\qsetscale{0.3}
	\qgrid{15}{15}
	\qcoord A50
	\qcoord B{15}5
	\qcoord C{10}{15}
	\qcoord D{0}{10}
	\qcsegment AB\qcsegment CB\qcsegment CD\qcsegment DA
	\qcoord E{12}{11}
	\qcoord F{12}6
	\qcoord G76
	\qcoord H7{11}
	\qcoord I2{11}
	\qcsegment EF\qcsegment FG\qcsegment GH\qcsegment HI
	\qcoord J71
	\qcsegment JG
	\qcoord K26
	\qcsegment GK
\end{tikzpicture}
\end{comment}

\end{document}
