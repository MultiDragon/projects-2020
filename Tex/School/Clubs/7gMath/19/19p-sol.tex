\documentclass[12pt,a4paper]{article}
\usepackage{tpl}
\odbegin{Кружок 7 класса, продолжающие}{Решения занятия №19}

\z На дороге, соединяющей два аула, нет горизонтальных участков. Автобус идет в гору всегда со скоростью 15 км/ч, а под гору~--- 30 км/ч. Путь туда и обратно автобус проезжает за 4 часа. Какое расстояние между аулами?

\s Пусть расстояние между аулами --- $\ell$. Заметим, что в гору и под гору мы проехали одинаковый путь (поскольку мы ехали туда и обратно). Следовательно, в гору и под гору мы проехали $\ell$. Времени мы потратили $\frac{\ell}{15}+\frac{\ell}{30}=4\text{ часа}$, следовательно, $\ell=40\text{ км}$.\QEDA\\

\z Можно ли накрыть равносторонний треугольник двумя меньшими равносторонними треугольниками?

\s Рассмотрим вершины большого треугольника. Расстояние между любыми двумя из них равно стороне треугольника. Теперь заметим, что в любом равностороннем треугольнике между любыми двумя точками внутри него расстояние не больше стороны. Следовательно, в одном маленьком треугольнике не больше вершины большого. По принципу Дирихле есть вершина без треугольника.\QEDA\\

\z Могут ли три человека, имея один двухместный мотоцикл, преодолеть расстояние 60 км за три часа? Скорость пешехода равна 5~км/ч, скорость мотоцикла (с одним и двумя седоками) --- 50 км/ч.

\s Да, могут. Пусть первый час два первых человека едут на мотоцикле вперёд, а третий идёт вперёд. Первые два на отметке 50 км, третий --- на отметке 5 км. Второй час пусть первый слезет с мотоцикла и идёт вперёд, второй едет назад до отметки 10 км, а третий --- идёт вперёд. Первый на отметке 55 км, второй и третий --- на отметке 10 км. Третий час пусть первый идёт вперёд, а остальные двое едут.\QEDA\\

\z На доске написаны семь различных нечетных чисел. Таня подсчитала их среднее арифметическое, а Даня упорядочил эти числа по возрастанию и выбрал из них число, оказавшееся посередине. Если из Таниного числа вычесть Данино, то получится число $\frac{3}{7}$. Не ошибся ли кто-нибудь из них?

\s Домножим и Танино, и Данино числа на 7. У Тани получится сумма, она нечётная. У Дани получится нечётное на нечётное, это тоже нечётно. Разность станет 3. Противоречие.\QEDA\\

\z Эскадрон улан и эскадрон драгун построены в две шеренги так, что позади каждого драгуна стоит улан выше него ростом. Доказать, что если обе шеренги перестроить по росту, то по-прежнему позади каждого драгуна будет стоять улан выше него ростом.

\s Допустим, что для какого-то $a$ драгун под номером $a$ (по убыванию) выше улана под таким же номером. Тогда есть максимум $a-1$ уланов выше $a$-го драгуна. Изначально позади первых (по убыванию) $a$ драгун стояли $a$ уланов, которые были выше, значит, эти $a$ уланов были выше $a$-го драгуна, противоречие.\QEDA\\

\z Можно ли на плоскости нарисовать 6 окружностей так, чтобы каждая касалась ровно четырех окружностей?

\s Можно. Например, так: в центре окружность. С четырёх сторон от неё её внешним образом касаются 4 другие окружности, причём эти окружности касаются друг друга. Вся конструкция вписана в большую окружность.\QEDA\\

\newpage\vskip10pt
\z Что больше в выпуклом четырехугольнике: периметр или удвоенная сумма длин диагоналей?

\s Удвоенная сумма диагоналей. Заметим, что диагонали делят четырёхугольник на 4 треугольника. В каждом из треугольников одна из сторон --- сторона четырёхугольника, и она меньше суммы двух других. Если мы просуммируем все 8 остальных сторон, получим как раз удвоенную сумму диагоналей.\QEDA\\

\z По кругу стоят натуральные числа от 1 до 6 по порядку. Разрешается к любым трем подряд идущим числам прибавить по 1 или из любых трех, стоящих через одно, вычесть 1. Можно ли с помощью нескольких таких операций сделать все числа равными?

\s Рассмотрим суммы противоположных чисел. Изначально они $1+4=5,2+5=7,3+6=9$. После каждого действия все три суммы изменяются на одно и то же число, а значит, равны не станут.\QEDA\\

\z Вокруг стола пустили пакет с семечками. Первый взял 1 семечку, второй --- 2, третий --- 3 и так далее: каждый следующий брал на одну семечку больше. Известно, что на втором круге было взято в сумме на 100 семечек больше, чем на первом. Сколько человек сидело за столом?

\s Заметим, что если человек было $n$, то во втором круге каждый взял всего на $n$ семечек больше, а всего взяли на $n^2$ семечек больше. $\sqrt{100} = 10$.\QEDA\\

\z Имеется 555 гирь весом $1,2,\ldots,555$ грамм. Можно ли их разложить на 3 равные по весу кучи?

\s Разложим первые 9 гирь в кучи $1, 5, 9; 3, 4, 8; 2, 6, 7$. Остальные гири разобьём на подряд идущие группы по 6, а каждую группу по 6 разобьём на 3 равные пары.\QEDA\\

\z Петин счет в банке содержит 500 долларов. Банкомат может совершать операции только двух видов: снимать 300 долларов или добавлять 198 долларов. Какую максимальную сумму Петя может снять со счета, если других денег у него нет?

\s Заметим, что сумма на руках у Пети всегда делится на 6. Следовательно, он может снять не больше 498 долларов. Докажем, что так он умеет. Для того, чтобы снять 198, ему нужно в совокупности 166 раз снять по 300 и 249 раз положить по 198. Заметим, что он всегда может сделать одно из этих действий (если текущий счёт хотя бы 300, он может снять 300, если меньше 300, то у него на руках больше 200 и можно положить 198).\QEDA\\

\end{document}
