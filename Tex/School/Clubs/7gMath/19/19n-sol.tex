\documentclass[12pt,a4paper]{article}
\usepackage{tpl}
\odbegin{Кружок 7 класса, начинающие}{Решения занятия №19}

\z На дороге, соединяющей два аула, нет горизонтальных участков. Автобус идет в гору всегда со скоростью 15 км/ч, а под гору~--- 30 км/ч. Путь туда и обратно автобус проезжает за 4 часа. Какое расстояние между аулами?

\s Пусть расстояние между аулами --- $\ell$. Заметим, что в гору и под гору мы проехали одинаковый путь (поскольку мы ехали туда и обратно). Следовательно, в гору и под гору мы проехали $\ell$. Времени мы потратили $\frac{\ell}{15}+\frac{\ell}{30}=4\text{ часа}$, следовательно, $\ell=40\text{ км}$.\QEDA\\

\z Можно ли накрыть равносторонний треугольник двумя меньшими равносторонними треугольниками?

\s Рассмотрим вершины большого треугольника. Расстояние между любыми двумя из них равно стороне треугольника. Теперь заметим, что в любом равностороннем треугольнике между любыми двумя точками внутри него расстояние не больше стороны. Следовательно, в одном маленьком треугольнике не больше вершины большого. По принципу Дирихле есть вершина без треугольника.\QEDA\\

\z Могут ли три человека, имея один двухместный мотоцикл, преодолеть расстояние 60 км за три часа? Скорость пешехода равна 5~км/ч, скорость мотоцикла (с одним и двумя седоками) --- 50 км/ч.

\s Да, могут. Пусть первый час два первых человека едут на мотоцикле вперёд, а третий идёт вперёд. Первые два на отметке 50 км, третий --- на отметке 5 км. Второй час пусть первый слезет с мотоцикла и идёт вперёд, второй едет назад до отметки 10 км, а третий --- идёт вперёд. Первый на отметке 55 км, второй и третий --- на отметке 10 км. Третий час пусть первый идёт вперёд, а остальные двое едут.\QEDA\\

\z На доске написаны семь различных нечетных чисел. Таня подсчитала их среднее арифметическое, а Даня упорядочил эти числа по возрастанию и выбрал из них число, оказавшееся посередине. Если из Таниного числа вычесть Данино, то получится число $\frac{3}{7}$. Не ошибся ли кто-нибудь из них?

\s Домножим и Танино, и Данино числа на 7. У Тани получится сумма, она нечётная. У Дани получится нечётное на нечётное, это тоже нечётно. Разность станет 3. Противоречие.\QEDA\\

\z Эскадрон улан и эскадрон драгун построены в две шеренги так, что позади каждого драгуна стоит улан выше него ростом. Доказать, что если обе шеренги перестроить по росту, то по-прежнему позади каждого драгуна будет стоять улан выше него ростом.

\s Допустим, что для какого-то $a$ драгун под номером $a$ (по убыванию) выше улана под таким же номером. Тогда есть максимум $a-1$ уланов выше $a$-го драгуна. Изначально позади первых (по убыванию) $a$ драгун стояли $a$ уланов, которые были выше, значит, эти $a$ уланов были выше $a$-го драгуна, противоречие..\QEDA\\

\z Можно ли на плоскости нарисовать 6 окружностей так, чтобы каждая касалась ровно четырех окружностей?

\s Можно. Например, так: в центре окружность. С четырёх сторон от неё её внешним образом касаются 4 другие окружности, причём эти окружности касаются друг друга. Вся конструкция вписана в большую окружность.\QEDA\\

\newpage\vskip10pt
\z Что больше в выпуклом четырехугольнике: периметр или удвоенная сумма длин диагоналей?

\s Удвоенная сумма диагоналей. Заметим, что диагонали делят четырёхугольник на 4 треугольника. В каждом из треугольников одна из сторон --- сторона четырёхугольника, и она меньше суммы двух других. Если мы просуммируем все 8 остальных сторон, получим как раз удвоенную сумму диагоналей.\QEDA\\

\z По кругу стоят натуральные числа от 1 до 6 по порядку. Разрешается к любым трем подряд идущим числам прибавить по 1 или из любых трех, стоящих через одно, вычесть 1. Можно ли с помощью нескольких таких операций сделать все числа равными?

\s Рассмотрим суммы противоположных чисел. Изначально они $1+4=5,2+5=7,3+6=9$. После каждого действия все три суммы изменяются на одно и то же число, а значит, равны не станут.\QEDA\\

\z Несколько ящиков вместе весят 10 тонн, причем каждый из них весит не более тонны. Какое минимальное число трехтонок заведомо достаточно, чтобы увезти этот груз?

\s Заметим, что 5 машин хватит. Начнём складывать груз в 1 машину. Когда в первую машину ничего не влезает, начинаем складывать во вторую. Заметим при этом, что в первой точно 2 тонны груза есть, и так далее.

Теперь докажем, что в 4 машины управиться нельзя. Пусть грузов 13 и все они $\frac{10}{13}$ тонн. 4 груза не влезают в одну машину, значит на 4 машинах мы можем перевести только 12 ящиков.\QEDA\\

\z В каждой клетке таблицы размером $13 \times 13$ записано одно из натуральных чисел от 1 до 25. Клетку назовем <<хорошей>>, если среди двадцати пяти чисел, записанных в ней и во всех клетках одной с ней горизонтали и одной с ней вертикали, нет одинаковых. Могут ли все клетки одной из главных диагоналей оказаться <<хорошими>>?

\s Т.к. $13<25$, какого-то числа от $1$ до $25$ нет на главной диагонали. Пусть это $a$. Тогда каждое число, равное $a$, <<считается>> клетками диагонали 2 раза (один раз сверху или снизу, второй --- справа или слева), значит, $a$ посчитано чётное количество раз. С другой стороны, клетки диагонали посчитали $a$ ровно $13$ раз. Противоречие.\QEDA\\

\end{document}
