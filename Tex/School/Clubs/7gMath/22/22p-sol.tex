\documentclass[12pt,a4paper]{article}
\usepackage{tpl}
\usepackage{chessboard}
\odbegin{Кружок для 7 класса, продолжающие}{Решения занятия №22}

\z Тетрадь, ручка и карандаш стоят 120 рублей. А 5 тетрадей, 2 ручки и 3 карандаша стоят 350 рублей. Что дороже: две тетради или одна ручка?

\s 3 тетради, 3 ручки и 3 карандаша стоят 360 рублей. 5 тетрадей, 2 ручки и 3 карандаша стоят 350 рублей. Получается, что ручка стоит на 10 рублей больше, чем 2 тетради.\QEDA\\

\z Учительница записала на доске два натуральных числа. Леня умножил первое число на сумму цифр второго и получил 201920192019. Федя умножил второе число на сумму цифр первого и получил 202020202020. Не ошибся ли кто-то из ребят?

\s Заметим, что число делится на 9 и на 3 тогда же, когда его сумма цифр делится на 9 и на 3. 201920192019 делится на 9, следовательно, произведение чисел делится на 9. С другой стороны, 202020202020 на 9 не делится. Противоречие.\QEDA\\

\z По кругу стоят 12 детей. Мальчики всегда говорят правду мальчикам и врут девочкам, а девочки всегда говорят правду девочкам и врут мальчикам. Каждый из них сказал одну фразу своему соседу справа: <<Ты --- мальчик>> или <<Ты --- девочка>>. Таких фраз оказалось поровну. Сколько мальчиков и сколько девочек стоит по кругу?

\s Заметим, что мальчик всегда говорит <<Ты --- мальчик>>, а девочка --- <<Ты --- девочка>>. Следовательно, и мальчиков, и девочек по 6 человек.\QEDA\\

\z В круговом шахматном турнире участвовало восемь человек: три девочки и пять мальчиков. Могли ли девочки по итогам турнира набрать в два раза больше очков, чем мальчики, или ещё больше? (В круговом турнире каждый играет с каждым один раз. За победу дается 1 очко, за ничью --- 0,5, за поражение --- 0).

\s Заметим, что в любом матче сумма баллов, которые получают участники, это 1 очко. Следовательно, всего разыгрывается очков столько же, сколько и матчей, то есть 28. Следовательно, у мальчиков должно быть не больше 9, а в матчах между собой они наберут 10. Противоречие.\QEDA\\

\z В соревнованиях велогонщиков на круговом треке приняли участие Вася, Петя и Коля, стартовав одновременно. Вася каждый круг проезжал на 2 секунды быстрее Пети, а Петя --- на три секунды быстрее Коли. Когда Вася закончил дистанцию, Пете осталось проехать один круг, а Коле --- два круга. Сколько кругов составляла дистанция?

\s Пусть в дистанции $k$ кругов, и Вася пробегает круг за $t$ секунд. Тогда рассмотрим момент финиша Васи.
$$kt = (k-1)(t+2) = (k-2)(t+5).$$
$$0 = 2k - t -2 = 5k - 2t -10.$$
$$4k - 2t - 4 = 5k - 2t - 10.$$
\textbf{Ответ:} $k = 6$.\QEDA\\

\newpage
\z Два поезда, в каждом из которых по 20 одинаковых вагонов, двигались навстречу друг другу по параллельным путям с постоянными скоростями. Ровно через 36 секунд после встречи их первых вагонов пассажир Вова, сидя в купе четвертого вагона, поравнялся с пассажиром встречного поезда Олегом, а еще через 44 секунды последние вагоны поездов полностью разъехались. В каком по счету вагоне ехал Олег?

\s Между моментом встречи поездов и моментом их полного разъезда второй поезд сдвигается относительно первого на 40 вагонов, то есть время в 80 секунд модно разделить на 40 равных промежутков. Также заметим, что через $n$ вагонов поравняются пары вагонов, у которых сумма номеров $n+1$. Нумерация ведётся с начала поезда, с первого вагона. Следовательно через 18 промежутков сумма будет 19, Олег едет в пятнадцатом вагоне.\QEDA\\

\z Каждый день, с понедельника по пятницу, ходил старик к синему морю и закидывал в море невод. При этом каждый день в невод попадалось не больше рыбы, чем в предыдущий. Всего за пять дней старик поймал ровно 100 рыбок. Какое наименьшее суммарное количество рыбок он мог поймать за три дня --- понедельник, среду и пятницу?

\s Заметим, что в понедельник старик поймал не меньше, чем во вторник, а в среду --- не больше, чем в четверг. Тогда он в понедельник, среду и пятницу поймал не меньше половины. Докажем, что 50 бывает. $50, 50, 0, 0, 0$.\QEDA\\

\z В одной из вершин шестиугольника лежит золотая монета, а в остальных ничего не лежит. Кощей Бессмертный чахнет над златом и каждое утро снимает с одной вершины произвольное количество монет, после чего тут же кладёт на соседнюю вершину в шесть раз больше монет. Докажите, что хоть злата у Кощея сколько угодно, он не сможет добиться, чтобы во всех вершинах было поровну монет.

\s Рассмотрим такую величину: остаток от (количества монет на чётных местах --- количества монет на нечётных местах) при делении на 7. При каждой операции она не меняется. Действительно, можно считать, что мы убрали 1 монету, тогда количество монет на местах той же чётности мы уменьшили на 1, а количество монет на другой чётности увеличилось на 6, т.е. остаток уменьшился на 1. Изначально она 1, а должна стать 0.\QEDA\\

\z На экране компьютера сгенерирована некоторая конечная последовательность нулей и единиц. С ней можно производить следующую операцию: набор цифр <<01>> заменять на набор цифр <<100>>. Может ли такой процесс замен продолжаться бесконечно или когда-нибудь он обязательно прекратится?

\s Эту задачу можно доказать по индукции и, наверное, ещё множеством других способов. Я (Никита --- прим. ред.) приведу нетривиальное доказательство через инвариант.

Пусть последовательность на экране компьютера --- алгоритм. Изначально на доске записано 1. 0 означает прибавить к числу на доске 1, 1 --- умножить на 2. Заметим, что результат алгоритма не меняется если мы производим замену. Каждое действие увеличивает наше значение. Соответственно, мы не можем из конечного алгоритма получить сколь угодно длинный.\QEDA\\

\z На шахматной доске стоят 8 не бьющих друг друга ладей. Докажите, что можно каждую из них передвинуть ходом коня так, что они по-прежнему не будут бить друг друга. (Все восемь ладей передвигаются одновременно, то есть если, например, две ладьи стоят друг от друга ходом коня, то их можно поменять местами.)

\s Разобьём доску на прямоугольники $2\times 4$ (сонаправленные). Каждый прямоугольник разобьём на 4 пары клеток, соединённых ходом коня.

Теперь передвинем все ладьи со своих клеток на парные. Пусть у нас какие-то две ладьи бьют друг друга. Тогда они стоят на одной строчке или столбце. Передвинем ладьи ещё раз, в изначальное положение. Но наша перестановка клетки одного столбца переводит в клетки одного столбца, а клетки одной строки --- в клетки одной строки. Следовательно в изначальной расстановке ладьи били друг друга, а такого не бывает. Противоречие.\QEDA\\

\newpage
\z Из шахматной доски (размером $8\times 8$) вырезали центральный квадрат размером $2\times 2$. Можно ли оставшуюся часть доски разрезать на равные фигурки в виде буквы <<Г>>, состоящие из четырех клеток? \textit{Фигурки можно поворачивать и переворачивать.}

\s Рассмотрим раскраску доски в два цвета в полосочку. Каждая буква <<Г>> содержит нечётное количество чёрных клеток. Букв <<Г>> 15, а чёрных клеток 30. Противоречие.\QEDA\\

\z В равнобедренном $\triangle ABC$ угол $\angle B$ равен $30^\circ, AB = BC = 6$. Проведены высота $CD$ треугольника $ABC$ и высота $DE$ треугольника $BCD$. Найдите $DE$.

\s $$\angle CBD = 30, DC = \frac{BC}{2} = 3.$$
$$\angle BDE = 90^\circ - \angle DBE = 60^\circ, \angle EDC = 90^\circ - \angle BDE = 30^\circ, EC = \frac{DC}{2} = 1.5.$$
$$BE = BC - EC = 4.5.\QEDA$$

\end{document}
