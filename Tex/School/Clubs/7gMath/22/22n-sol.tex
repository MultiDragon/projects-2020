\documentclass[12pt,a4paper]{article}
\usepackage{tpl}
\usepackage{chessboard}
\odbegin{Кружок для 7 класса, начинающие}{Решения занятия №22}

\z Кикимора выращивает сон-траву на участке прямоугольной формы. Иван-дурак в качестве помощи решил вскопать участок под посев, но внес изменения, одну сторону увеличил на $30\%$, а другую уменьшил на $30\%$. Изменится ли в результате урожай сон-травы, и если изменится, то как?

\s $0.7 * 1.3 = 0.91$. Уменьшится на $9\%$.\QEDA\\

\z Тетрадь, ручка и карандаш стоят 120 рублей. А 5 тетрадей, 2 ручки и 3 карандаша стоят 350 рублей. Что дороже: две тетради или одна ручка?

\s 3 тетради, 3 ручки и 3 карандаша стоят 360 рублей. 5 тетрадей, 2 ручки и 3 карандаша стоят 350 рублей. Получается, что ручка стоит на 10 рублей больше, чем 2 тетради.\QEDA\\

\z Учительница записала на доске два натуральных числа. Леня умножил первое число на сумму цифр второго и получил 201920192019. Федя умножил второе число на сумму цифр первого и получил 202020202020. Не ошибся ли кто-то из ребят?

\s Заметим, что число делится на 9 и на 3 тогда же, когда его сумма цифр делится на 9 и на 3. 201920192019 делится на 9, следовательно, произведение чисел делится на 9. С другой стороны, 202020202020 на 9 не делится. Противоречие.\QEDA\\

\z По кругу стоят 12 детей. Мальчики всегда говорят правду мальчикам и врут девочкам, а девочки всегда говорят правду девочкам и врут мальчикам. Каждый из них сказал одну фразу своему соседу справа: <<Ты --- мальчик>> или <<Ты --- девочка>>. Таких фраз оказалось поровну. Сколько мальчиков и сколько девочек стоит по кругу?

\s Заметим, что мальчик всегда говорит <<Ты --- мальчик>>, а девочка --- <<Ты --- девочка>>. Следовательно, и мальчиков, и девочек по 6 человек.\QEDA\\

\z В соревнованиях велогонщиков на круговом треке приняли участие Вася, Петя и Коля, стартовав одновременно. Вася каждый круг проезжал на 2 секунды быстрее Пети, а Петя --- на три секунды быстрее Коли. Когда Вася закончил дистанцию, Пете осталось проехать один круг, а Коле --- два круга. Сколько кругов составляла дистанция?

\s Пусть в дистанции $k$ кругов, и Вася пробегает круг за $t$ секунд. Тогда рассмотрим момент финиша Васи.
$$kt = (k-1)(t+2) = (k-2)(t+5).$$
$$0 = 2k - t -2 = 5k - 2t -10.$$
$$4k - 2t - 4 = 5k - 2t - 10.$$
\textbf{Ответ:} $k = 6$.\QEDA\\

\z Два поезда, в каждом из которых по 20 одинаковых вагонов, двигались навстречу друг другу по параллельным путям с постоянными скоростями. Ровно через 36 секунд после встречи их первых вагонов пассажир Вова, сидя в купе четвертого вагона, поравнялся с пассажиром встречного поезда Олегом, а еще через 44 секунды последние вагоны поездов полностью разъехались. В каком по счету вагоне ехал Олег?

\s Между моментом встречи поездов и моментом их полного разъезда второй поезд сдвигается относительно первого на 40 вагонов, то есть время в 80 секунд модно разделить на 40 равных промежутков. Также заметим, что через $n$ вагонов поравняются пары вагонов, у которых сумма номеров $n+1$. Нумерация ведётся с начала поезда, с первого вагона. Следовательно через 18 промежутков сумма будет 19, Олег едет в пятнадцатом вагоне.\QEDA\\
\newpage

\z Каждый день, с понедельника по пятницу, ходил старик к синему морю и закидывал в море невод. При этом каждый день в невод попадалось не больше рыбы, чем в предыдущий. Всего за пять дней старик поймал ровно 100 рыбок. Какое наименьшее суммарное количество рыбок он мог поймать за три дня --- понедельник, среду и пятницу?

\s Заметим, что в понедельник старик поймал не меньше, чем во вторник, а в среду --- не больше, чем в четверг. Тогда он в понедельник, среду и пятницу поймал не меньше половины. Докажем, что 50 бывает. $50, 50, 0, 0, 0$.\QEDA\\

\z Какое наибольшее число слонов можно расставить на шахматной доске $8\times8$, чтобы они не били друг друга?

\s Пример на 14 слонов школьник покажет. Докажем, что на клетках одного цвета не может быть 8 слонов. Действительно, есть 7 диагоналей, параллельных большой диагонали, и на каждой из них не более одного слона.\QEDA\\

\z На экране компьютера сгенерирована некоторая конечная последовательность нулей и единиц. С ней можно производить следующую операцию: набор цифр <<01>> заменять на набор цифр <<100>>. Может ли такой процесс замен продолжаться бесконечно или когда-нибудь он обязательно прекратится?

\s Эту задачу можно доказать по индукции и, наверное, ещё множеством других способов. Я (Никита --- прим. ред.) приведу нетривиальное доказательство через инвариант.

Пусть последовательность на экране компьютера --- алгоритм. Изначально на доске записано 1. 0 означает прибавить к числу на доске 1, 1 --- умножить на 2. Заметим, что результат алгоритма не меняется если мы производим замену. Каждое действие увеличивает наше значение. Соответственно, мы не можем из конечного алгоритма получить сколь угодно длинный.\QEDA\\

\z Из шахматной доски (размером $8\times 8$) вырезали центральный квадрат размером $2\times 2$. Можно ли оставшуюся часть доски разрезать на равные фигурки в виде буквы <<Г>>, состоящие из четырех клеток? \textit{Фигурки можно поворачивать и переворачивать.}

\s Рассмотрим раскраску доски в два цвета в полосочку. Каждая буква <<Г>> содержит нечётное количество чёрных клеток. Букв <<Г>> 15, а чёрных клеток 30. Противоречие.\QEDA\\

\end{document}
