\documentclass[12pt,a4paper]{article}
\usepackage{tpl}
\odbegin{Кружок 7 класса, начинающие, школа 179}{Решения занятия №23}

\z Даня задумал два натуральных числа, перемножил их сумму и произведение и назвал Тане результат: 93465. Таня ответила, что этого не может быть. Права ли она? 

\s Если среди чисел есть чётное, то чётное произведение, а если нет, то чётная сумма. В обоих случаях результат чётный, а 93465 нечётное. Значит, Даня ошибся.\QEDA\\

\z Король со свитой едет из пункта $A$ в пункт $B$ со скоростью 5 км/ч. Каждый час он высылает в пункт $B$ гонцов, которые едут со скоростью 20 км/ч. С каким интервалом они прибывают в $B$?

\s Заметим, что через час после выезда гонца, он будет в 20 км от точки его выезда, а король --- в 5 км, т.е. расстояние между двумя гонцами 15 км, и гонцы проезжают его за 45 минут. Ответ: 45 минут.\QEDA\\

\z На столе лежат четыре карточки: <<А>>, <<Б>>, <<4>>, <<5>>. На каждой карточке с одной стороны --- буква, а с другой --- натуральное число. Какие карточки надо перевернуть, чтобы узнать, правда ли, что если на карточке написано чётное число, то с другой стороны --- гласная буква?

\s Заметим, что опровергнуть условие могут только карточки, у которых с одной стороны чётное число, а с другой --- согласная буква. Тогда <<А>> и <<5>> нам навредить не могут, а <<Б>> и <<4>> нужно всё-таки проверить. \QEDA\\

\z Петя спускается по эскалатору метро, едущему вниз, наступая на каждую ступеньку. Как ему идти, чтобы наступить на большее число ступенек --- помедленнее или побыстрее?

\s Рассмотрим ситуацию относительно системы отсчёта эскалатора. Мы движемся по нему, а к нам навстречу движется край этого эскалатора. Чем медленнее мы идём, тем больше эскалатора перед встречей нас и края эскалатора пройдёт край, и тем меньше эскалатора пройдём мы. Ответ: Побыстрее. \QEDA\\

\z У Белоснежки был клетчатый коврик $9\times 12$. Гном Жулин отрезал от него полоску $8\times 1$ себе на кушак. Сначала Белоснежка решила отрезать ещё полоску $4\times 1$, чтобы коврик снова стал прямоугольным, но потом передумала и, отрезав два куска, сшила из трёх получившихся кусков коврик $10\times 10$. Как она это сделала?

\s Зачёт по рисунку школьника. \QEDA\\

\z Из набора домино выбросили все кости с пустышками. Можно ли все оставшиеся кости выложить в ряд?

\s Предположим, что можно. Рассмотрим, как встречается какое-нибудь число. Например, единица. Она встречается 7 раз. Поскольку внутри ряда она встречается чётное число раз, единица будет и на конце ряда. Аналогично на конце ряда будут и все остальные числа от 2 до 6. Их больше, чем концов ряда. Противоречие. \QEDA\\

\n Дана клетчатая доска размером $10\times 10$ клеток. Играют двое, ходят по очереди; за ход нужно вычеркнуть одну горизонталь или вертикаль, на которой есть хотя бы одна невычеркнутая клетка.Проигрывает тот, кто не может сделать ход. Кто может обеспечить себе победу --- начинающий или его противник, и как ему играть?

\s Выигрывает второй. Когда первый отрезает горизонталь, второй тоже отрезает горизонталь, и наоборот. Тогда в каждый момент времени после хода первого размер доски по какому-то направлению нечётный, значит, он не выиграет.\QEDA

\p Гном Жулин отрезал от доски полоску $10\times 1$ себе на кушак. Кто выиграет теперь?

\s Выиграет первый, стратегия аналогичная.\QEDA\\

\z В турнире участвуют 100 сумоистов разного веса; более тяжелый всегда побеждает. Сумоисты разбились на пары и провели поединки.Затем разбились на пары по-другому и снова провели поединки.Призы получили те, кто выиграл оба поединка. Каково наименьшее возможное количество призеров?

\s 1 призёр обязательно есть, самый тяжёлый сумоист. Но заметим, что матчи могли быть проведены так (1 --- самый лёгкий, 100 --- самый тяжёлый): $1-2, 3-4,\ldots, 99-100$, а во второй день $2-3, 4-5, \ldots , 98-99, 100-1$. \QEDA\\

\z Могут ли произведения всех ненулевых цифр двух последовательных натуральных чисел отличаться ровно в 54 раза?

\s Да, например, 299 и 300. \QEDA\\

\z Могут ли две биссектрисы треугольника пересекаться под прямым углом?

\s Нет, потому что тогда сумма половин двух углов треугольника --- $90^\circ$, а сумма соответствующих углов --- $180^\circ$. \QEDA\\

\z Нарисуйте многоугольник и точку внутри него так, чтобы не было стороны многоугольника, которая видна из этой точки полностью.

\s Зачёт по рисунку школьника. \QEDA\\

\z Как разрезать куб на три равные пирамиды?

\s Например, так. Общая вершина всех пирамид --- вершина куба, а основания --- три грани, которые с этой вершиной не граничат. \QEDA\\

\end{document}
