
\documentclass[12pt,a4paper]{article}
\usepackage{tpl}
\odbegin{Кружок для 7 класса, продолжающие}{Решения занятия №20}

\z Пете позавчера было 12 лет, а в следующем году будет 15. Могло ли так быть?

\s Да, так может быть. Пусть у Пети день рождения 31 декабря, сегодня 1 января. Тогда позавчера ему было 12, вчера 13, 31 декабря в этом году исполнится 14, а 31 декабря в следующем --- 15.\QEDA\\

\z Гном Шоппин нашел на распродаже слона за 1 талер. К несчастью, и у него, и у продавца есть монеты только по 16 и 27 талеров. Сможет ли Шоппин купить слона?

\s Да, сможет. Шоппин даёт 3 монеты по 27 талера и получает 5 монет по 16 талеров в сдачу.\QEDA\\

\z Сколько существует шестизначных чисел, содержащих только нечетные цифры?

\s На каждом месте может быть одна из пяти нечётных цифр, всего $5^6 = 15625$.\QEDA\\

\z Сколько существует шестизначных чисел, в записи которых есть хотя бы одна четная цифра?

\s $900000 - 15625 = 884375$.\QEDA\\ 

\z Выписали все натуральные числа от 1 до 1000. На сколько среди них больше чисел с суммой цифр 14, чем с суммой цифр 13?

\s Заметим, что если число в промежутке от 0 до 999 вычесть из 999, то получим другое число в таком промежутке. Если это проделать с числом с суммой цифр 13, то получим число с суммой цифр 14, то есть чисел с суммой цифр 13 не больше, чем с суммой цифр 14. Аналогично, их и не меньше. Значит в промежутке от 0 до 999 таких чисел равно. Ни 0, ни 1000 ни на что не влияют, поэтому ответ - 0.\QEDA\\

\z В таблице $8\times 8$ угловая клетка покрашена черным цветом, остальные белым. За ход можно одновременно поменять цвета всех клеток в столбце или строке. Можно ли через несколько ходов получить таблицу, покрашенную в белый цвет?

\s Рассмотрим угловые клетки. Среди них нечётное число белых. Заметим, что любая горизонталь или вертикаль проходит через чётное число угловых. Если мы меняем цвет чётного числа угловых, количество угловых белых также меняется на чётное число. Мы не можем получить из нечётного чётное.\QEDA\\

\z Из клетчатой бумаги вырезали квадрат $10\times 10$ и согнули несколько раз по линиям клеток так, что получился квадратик $1\times 1$. Этот квадратик разрезали по линии, соединяющей середины противоположных сторон. Сколько кусочков бумаги при этом могло получиться?

\s Рассмотрим два соседних маленьких квадратика. Они были совмещены сгибом по линии между ними. Следовательно, разрезы в них симметричны относительно линии между ними. Как-нибудь разрежем нижний левый квадрат, и это определит всю картинку. Ответ: 11.\QEDA\\

\z Гном Жорин за десять дней съел 150 груш. Каждый следующий день он съедал больше груш, чем в предыдущий, причем в первый день он съел не менее половины того, что в десятый. Сколько груш он съел в шестой день?

\s Предположим, в последний день гном Жорин съел не больше 19 груш. Тогда в предпоследний день он съел не больше 18 груш, и так далее, и в первый день он съел не более 10 груш, то есть всего не больше, чем $\frac{29\cdot 30}{2}=145$ груш.

Предположим теперь, что в последний день Жорин съел не меньше 21 груш. Тогда в первый день он съел не меньше 11 груш, во второй - не меньше 12 груш, и так далее, и в девятый не меньше 19 груш. Всего он съел не меньше $\frac{31\cdot 10}{2}+1=156$ груш.

Следовательно, в последний день Жорин съел 20 груш. В первый день Жорин съел 10 или 11, во второй - 11 или 12 и так далее, в девятый съел 18 или 19. $10+11+\ldots +18+20=\frac{29\cdot 30}{2}+1=146$ груш, значит, нам нужно добавить ещё четыре. С шестого по девятый дни Жорин ел большее число, с первого по пятый - меньшее. Соответственно, в шестой день Жорин съел 16.\QEDA\\

\z Барон Мюнхгаузен утверждает, что смог разрезать квадратный ковер на квадратные коврики двух разных размеров так, что больших ковриков получилось столько же, сколько и маленьких. Не хвастает ли барон?

\s Удивительно, но не хвастает. Пусть у нас есть коврик $5\times 5$. 9 ковриков $1\times 1$ мы разделим на 16 частей каждый, а 16 ковриков $1\times 1$ --- на 9 частей каждый. И маленьких, и больших ковриков по 144.\QEDA\\

\z Белоснежке на Новый Год подарили набор из 27 одинаковых кубиков. Она хочет склеить из них кубик $3\times 3\times 3$. Для этого она капает по капельке меда между некоторыми двумя кубиками, которые соприкасаются поверхностями, стараясь выбирать их так, чтобы большой кубик не развалился. Доказать, что ей:

\p хватит 26 капелек;

\p хватит 25 капелек;

\p* хватит 24 капелек;

\s В этих трёх пунктах можно проверять рисунки школьников.\QEDA\\

\p** но не хватит 23 капелек.

\s Ну не хватит, нормального доказательства я не знаю. Вряд ли кто-то будет это сдавать, с очень низкой долей вероятности будут сдавать перебором (методов лучше нет), тогда надо будет его проверить.\QEDA\\

\z Коробка из-под кубиков имеет форму куба без крышки. Сможет ли Белоснежка разрезать ее на 3 части так, чтобы потом сложить из них квадратный коврик?

\s Сможет, проверяется по рисунку школьника.\QEDA\\

\z Можно ли в кружках (см. рисунок) разместить различные натуральные числа таким образом, чтобы суммы трех чисел вдоль каждого отрезка оказались равными?\label{circle}

\s Нет. Пусть сумма чисел на каждой из линий $a$, а во всём треугольнике --- $b$. Тогда каждое из чисел в вершинах треугольника равно $\frac{3a-b}{2}$, поскольку мы можем сложить три линии, проходящие через них, и вычесть весь треугольник.\QEDA\\

\begin{figure}[!htb]
	\begin{minipage}{0.44\textwidth}\centering
		\inkscapeinclude{20n-triangles}{20n-triangles}
		\caption{Треугольник в задаче \ref{circle}}
	\end{minipage}
\end{figure}

\end{document}
