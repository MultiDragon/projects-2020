\documentclass[12pt,a4paper]{article}
\usepackage{tpl}
\odbegin{Кружок 7 класса, продолжающие, школа 179}{Решения занятия №23}

\z <<А это вам видеть пока рано>>, сказала Баба-Яга своим 33 ученикам и скомандовала: <<Закройте глаза!>> Правый глаз закрыли все мальчики и треть девочек. Левый глаз закрыли все девочки и треть мальчиков. Сколько учеников всё-таки увидели то, что видеть пока рано?

\s Заметим, что две трети мальчиков увидели то, что им видеть рано, поскольку правый глаз закрыли все, а левый только треть. Аналогично, две трети девочек тоже увидели то, что им пока рано. \textbf{Ответ}: 22 ученика.\QEDA\\

\z Разрежьте квадрат $6\times 6$ клеточек на трёхклеточные уголки, чтобы никакие два уголка не образовывали прямоугольник $2\times 3$.

\s Проверяется по рисунку школьника. \QEDA\\

\z Два автобуса ехали навстречу друг другу с постоянными скоростями. Первый выехал из Москвы в 11 часов утра и прибыл в Ярославль в 16 часов, а второй выехал из Ярославля в 12 часов и прибыл в Москву в 17 часов. В котором часу они встретились? 

\s Пусть они встретятся через $t$ часов после 11 часов утра. Пусть расстояние от Москвы до Ярославля $S$. Расстояние первого до Москвы будет $\frac{St}{5}$. Расстояние второго --- $S-\frac{t-1}{5}S$, и они равны, откуда $t=3$. Другое, более короткое и идейное решение состоит в том, чтобы нарисовать по клеточкам графики расстояния до Москвы от времени, заметить что они отрезки и посчитать время по клеточкам. Ответ: 14 часов. \QEDA\\

\z В четырёхугольнике $ABCD$ биссектрисы $AE$ и $CF$ углов $\angle BAD$ и $\angle BCD$ параллельны. Докажите, что $\angle ABC=\angle ADC$.

\s $\angle ABC=\angle ABE=180^\circ-\angle BEA-\angle EAB=180^\circ-\angle BCF-\angle EAB=180^\circ-\frac{\angle DCB+\angle DAB}{2}$, аналогично тому же равен $\angle ADC$. \QEDA\\

\z Можно ли разрезать квадрат $5\times 5$ на 7 прямоугольников вида $1\times 3$ и $1\times 4$?

\s Можно, проверяется по рисунку школьника. \QEDA\\

\z Каких прямоугольников с целыми сторонами больше: с периметром $2020$ или $2022$?

\s Заметим, что сумма двух соседних сторон у прямоугольника равна половине периметра. В одном случае это 1010, а в другом 1011. Нам нужно рассмотреть всевозможные пары целых (натуральных) сторон у прямоугольников. В первом это от 1, 1009 до 505, 505, а во втором --- от 1, 1010 до 505, 506. \textbf{Ответ:} поровну. \QEDA\\

\z Гномы собрали в лесу корзину орехов. Леший разделил между ними орехи поровну, а себе забрал остаток. Каждому гному досталось по 5 орехов, а Лешему --- 4. На следующий день пришло втрое больше гномов, и они собрали втрое больше орехов. И снова Леший разделил между ними орехи поровну, а себе забрал остаток. Сколько орехов получили гномы и Леший на второй день?

\s Заметим, что у Лешего в первом случае осталось 4 ореха, значит, гномов было не меньше 5. Пусть гномов было $k$. Тогда орехов $5k+4$. Потом гномов стало $3k$, а орехов $15k+12$. Нам нужно поделить нацело $15k+12$ на $3k$, но $3k$ минимум 15, следовательно, получаем 5. Гномы получили по 5 орехов, а Леший получил 12. \QEDA\\

\z В некоторый момент угол между часовой и минутной стрелкой равен $\alpha$, а через час снова равен $\alpha$. Найдите все возможные значения $\alpha$.

\s За час часовая стрелка повернулась на $30^\circ$, а минутная не подвинулась. \textbf{Ответ}: $15^\circ,165^\circ$. \QEDA\\

\newpage
\z На гипотенузе $AB$ прямоугольного треугольника $ABC$ выбрана такая точка $D$, что $BD=BC$, а на катете $BC$ --- такая точка $E$, что $DE=BE$. Докажите, что $AD+CE=DE$.

\s Построим точку $X$ так, что $CX=AD$ (см. рис.) Тогда $\triangle XDB=\triangle ABC$, откуда $\angle XDB=90^\circ$, т.е. $DE= \frac{XB}{2}=\frac{AB}{2}$. Тогда $AD+CE=AB-BD+BC-BE=AB-DE=DE$.\QEDA\\

\begin{figure}[!htb]
	\begin{minipage}{0.32\textwidth}\centering
		\begin{tikzpicture}
			\qsetscale{1}
			\qcoord C40
			\qcoord B{10}0
			\qcoord A48
			\qcoord X00
			\qcgHeight DXAB
			\qcgCenter EDBX

			\qcspoint A
			\qcspoint B
			\qcspoint[above left] C
			\qcspoint[blue,above] X
			\qcspoint D
			\qcspoint[above left] E
			\qctriangle[blue] XBD
			\qctriangle ABC
			\qcsegment DE
		\end{tikzpicture}
	\end{minipage}
\end{figure}

\z Верно ли, что изменив одну цифру в десятичной записи любого натурального числа, можно получить простое число?

\s Нет. Например, рассмотрим $n=200$. Если изменить не последнюю цифру, число будет делиться на 10; если заменить её на чётную или 5, оно будет делиться на 2 или 5; наконец, 201, 203, 207, 209 делятся соответственно на 3, 7, 3 и 11 соответственно.

\textit{Примечание.} Это не единственное число с таким свойством. Также подходят 320, 510, 840, $10!$, $10!^3$, $19!+10$ и т.п.\QEDA\\

\z Дракон запер в пещере шестерых гномов и сказал: <<У меня есть семь колпаков семи цветов радуги. Завтра утром я завяжу вам глаза и надену на каждого по колпаку, а один колпак спрячу. Затем сниму повязки, и вы сможете увидеть колпаки на головах у других, но общаться я вам уже не позволю. После этого каждый втайне от других скажет мне цвет спрятанного колпака. Если угадают хотя бы трое, всех отпущу. Если меньше --- съем на обед>>. Как гномам заранее договориться, чтобы спастись?

\s Пронумеруем колпаки от 0 до 6 в любом порядке (причём можно считать, что это остатки, т.е. колпак 7 --- это колпак 0 и т.п.) Заметим, что каждый гном может сказать одно из двух чисел --- его колпак и спрятанный. Тогда один из этих двух колпаков больше другого не больше чем на 3 (например, если колпаки 0 и 6, то $0-6=1<3$). Назовём его \textit{большим}. Пусть все гномы будут считать, что большой колпак на них. Тогда ровно три из них угадают.\QEDA\\

\z Сережа вырезал из картона две одинаковые фигуры. Он положил их с нахлестом на дно прямоугольного ящика. Дно оказалось полностью покрыто. В центр дна вбили гвоздь. Мог ли гвоздь проткнуть одну картонку и не проткнуть другую?

\s Мог, проверяется по рисунку школьника.\QEDA\\

\end{document}
