\documentclass[12pt,a4paper]{article}
\usepackage{tpl}
\dbegin{Школа 179}{$q$-теорема Куммера}

Обозначим $\Phi_n(q)$ --- многочлены деления круга, т.е. такие, что $\prod\limits_{d\mid n}\Phi_d=[n]$.

\lemma $[n]!=\prod\limits_{d\leq n}\Phi_d^\floor{\frac nd}$.\label{qfac}

\proof Заметим, что $[n]!$ разбивается на множители, каждый из которых разбивается на $\Phi_d$, и каждый множитель вида $\Phi_d$ входит в следующие $q$-числа и только в них: $[d],[2d],\ldots,[\floor{\frac nd}d]$.\QEDA\\

\theorem $Q^n_{n+m}=\prod\limits_{d\leq m+n}\Phi_d^{\floor{\frac{m+n}d}-\floor{\frac md}-\floor{\frac nd}}$.\label{qkummer}

\proof Очевидно из \ref{qfac}. \QEDA\\

\lemma $\Phi_d(1)$ равно $p$, если $d=p^n$, и $1$ в противном случае.\label{unquant} 

\proof Посмотрим на $[p^{N+1}]=[p^N]\Phi_{p^{N+1}}$. Левая часть в точке $q=1$ равна $p^{N+1}$, а левый множитель --- $p^N$, значит, правая часть равна $p$. С другой стороны, если $m$ не степень простого и равно $\prod\limits_i p_i^{\alpha_i}$, то $[m]$ делится на произведение $[p_i^{\alpha_i}]$, которое равно $m$ в точке $q=1$, значит, $\Phi_m(1)=1$.\QEDA\\

\theoremn{Куммер} Степень вхождения $p$ в $C^n_{n+m}$ равно количеству переносов при сложении $n$ и $m$ в столбик в $p$-ричной системе исчисления.

\proof Применим \ref{qkummer} и подставим $q=1$. Тогда по \ref{unquant} степень вхождения $p$ в это выражение равна количеству множителей вида $\Phi_{p^k}$, которое равно количеству переносов. \QEDA\\

\lemman{$q$-числа Каталана} $Q^n_{2n}$ делится на $[n+1]$.

\proof Заметим, что для всех $d|n+1$ выполняется $\floor{2\frac nd}>2\floor{\frac nd}$. Действительно, при увеличении $n$ на 1 левое выражение увеличится на 1, правое на 2, и они оба станут равны $2\frac{n+1}d$ (и, соответственно, друг другу). Отсюда и следует утверждение задачи. \QEDA\\

\end{document}
