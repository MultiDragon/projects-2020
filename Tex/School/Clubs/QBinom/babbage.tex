\documentclass[12pt,a4paper]{article}
\usepackage{tpl}
\dbegin[14-\today]{Школа 179}{q-Babbage Theorem}

\definition{$q$-биномиальный коэффициент $Q^k_n(q)$} --- многочлен от $q$, коэффициент при $q^l$ которого равен количеству правильных путей из $(0,0)$ в $(k,n-k)$, площадь под которыми равна $l$. Если переменная (<<вес клетки>>) не указана, то используется $q$. Также можно брать $q$-бином от множества, тогда считается количество путей из этого множества.\\ 

\lemma Степень вхождения $[p]$ в $[r]$ равна 1, если $p\mid r$, иначе $([p],[r])=1$.

\proof Если $p\nmid r$, то у $[r]$ нет корня, являющегося корнем $p$-й степени из 1, откуда следует вторая часть утверждения. Если $r=kp$, то выполняется $[r]=[p](q^{p(k-1)}+q^{p(k-2)}+\ldots)\equiv[p]k\mod[p]^2$, откуда следует первая часть. \QEDA\\

\lemma Степень вхождения $[p]$ в $[r]!$ равна $\floor{\frac rp}$.

\proof Все множители в $[r]!$ вида $[k]$, где $p\nmid k$, взаимно просты с $[p]$ и не влияют на степень вхождения. Также $\prod\limits_k\frac{[kp]}{[p]}\equiv\floor{\frac rp}!\mod [p]$, т.е. степень вхождения равна количеству этих множителей. \QEDA\\

\lemma $Q^k_{bp}$ делится на $[p]$ и не делится на $[p]^2$ при $p\nmid k$.\label{vhozd}

\proof Заметим, что $Q^k_{bp}=\frac{[bp]!}{[k]![bp-k]!}$ и степень вхождения $[p]$ в числитель равна $b$, а в знаменатель --- $b-1$. \QEDA\\

\theorem $Q^{bp}_{ap}(q)\equiv Q^b_a(q^{p^2})\mod [p]^2$.

\proof Доказываем по индукции по $a$. База при $a=0$ верна.

\textbf{Шаг индукции.} Посмотрим на какой-то правильный путь $\mathcal P$ в прямоугольнике $bp\times (a-b)p$. Пусть, не умаляя общности, $a\geq 2b$, и прямая через точки $(0,bp)$ и $(bp,0)$ пересекается с $\mathcal P$ в точке $(x,bp-x)$. Заметим, что $q$-бином от множества $\{\mathcal P: x=x_0\}$ равен $Q^{x_0}_{bp}q^{(bp-x_0)^2}Q^{bp-x_0}_{(a-b)p}$. Тогда по \ref{vhozd} этот $q$-бином делится на $[p]^2$ при $p\nmid x_0$, нас интересуют только пути, проходящие через какую-то (причём ровно 1) точку вида $(xp,(b-x)p)$. Для таких путей работает предположение индукции, т.е. \[
	Q^{bp}_{ap}(q)\equiv\sum\limits_{x=0}^b Q^{bp-xp}_{ap-bp}(q)Q^{xp}_{bp}(q)\cdot q^{(bp-xp)^2}\equiv\sum\limits_{x=0}^b Q^{b-x}_{a-b}(q^{p^2})Q^x_{b}(q^{p^2})\cdot(q^{p^2})^{(b-x)^2}
.\]

Заметим, что эта сумма равна $Q^b_a(q^{p^2})$ по тождеству Вандермонда.\QEDA\\

\theorem $Q^{bp}_{ap}\equiv C^b_a\cdot(1+\frac12pb(a-b)(q^p-1))\mod[p]^2$.

\end{document}
