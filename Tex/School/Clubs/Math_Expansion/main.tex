\documentclass[12pt,a4paper]{article}
\usepackage{tpl}
\usepackage{xcolor}
\newcommand{\us}{\textcolor{yellow}{\bf 3 Призёра я дам, всерос я не дам}

\z}
\dbegin[23 апреля 2020 г.]{}{Предмет, 11 класс}{Автор курса}

\us Решите систему уравнений в действительных числах:
\begin{align*}
	\begin{cases}
		y^x=x^y;\\
		x^{228}=y^{100500}.
	\end{cases}
\end{align*}\vskip10pt

\us Дана бесконечная клетчатая плоскость, в некоторых клетках нижней полуплоскости (т.е. на точках с $y\leq 0$; в конечном числе) стоят фишки. За ход можно удалить две фишки в соседних клетках и поставить фишку на соседнюю клетку на той же прямой. При каком максимальном $n$ можно поставить фишку на координаты $(x,n)$ при каком-то $x$? (с решением)\\

\us На двух клетках шахматной доски стоят черная и белая фишки. За один ход можно передвинуть любую из них на соседнюю по вертикали или по горизонтали клетку. (Две фишки не могут стоять на одной клетке.) Могут ли в результате таких ходов встретиться все возможные варианты расположения этих двух фишек, причем ровно по одному разу? (с решением)\\

\us Дана короткая односторонняя линейка и точки $A,B$. Можно ли построить отрезок $AB$? (с решением)\\

\us В круг написаны числа от 1 до 30 именно в таком порядке. За ход можно выбрать 3 соседних положительных числа $a,b,c$ и поменять их на $ b-1,c-1,a-1$ именно в таком порядке. Какое максимальное число ходов можно сделать? (с решением)\\

\end{document}
