\documentclass[12pt,a4paper]{article}
\usepackage{tpl}
\pagenumbering{gobble}
\relpenalty=10000 \binoppenalty=10000
\newcommand{\pr}[3]{\dup[6]{\textbf{#1-#2.} #3\vfill\nnn\vfill}}
\newcommand{\nnn}{\vskip8pt\hrule\vskip8pt}
\newcommand{\ans}[3]{\mbox{\textbf{#1-#2}. #3.}\hskip18pt}
\newcommand{\aans}{
\vskip10pt\hdr{Ответы}\vskip10pt

\noindent\ans{1}{1}{2024}\ans{1}{2}{1010}\ans{1}{3}{7}\ans{1}{4}{1515}\ans{1}{5}{14}\ans{1}{6}{8}\ans22{$3\sqrt3$}\ans23{21}\ans246
\vskip5pt
\noindent\ans25{25}\ans26{997}\ans33{$\frac{2}{15}$}\ans34{$\frac{1}{1000\cdot 1001}$}\ans35{12 или 13}\ans36{8090}\ans44{$2^{220}$}
\vskip5pt
\noindent\ans45{$90^\circ,45^\circ,45^\circ$ или $60^\circ,60^\circ,60^\circ$}\ans46{145}\ans556\ans56{58}\ans66{133}
\nnn
}
\begin{document}

\pr{1}{1}{Число $b$ --- среднее арифметическое пяти последовательных целых чисел, наименьшее из которых равно $2020$. Чему равно среднее арифметическое пяти последовательных целых чисел, наименьшее из которых равно $b$?}

\pr{1}{2}{Даны окружности радиуса $1$ с центрами $O_1,O_2$, причём $O_1O_2=2020$. Какую максимальную площадь может иметь треугольник $O_1MO_2$, где $M$ находится на какой-то из этих окружностей?}\newpage

\pr{1}{3}{Какое наименьшее количество точек надо взять на плоскости так, чтобы среди их попарных расстояний были числа $1,2,4,8,16,32,47$?}

\pr{1}{5}{Шахматный король попал с поля b8 на поле g5, сделав наименьшее возможное число ходов. Сколькими способами он мог это сделать?}

\pr{1}{6}{Для скольких натуральных $n$ таких, что $100<n\leq 10000$, число $n$ делится на $\sqrt{n-100}$?}\newpage


\pr{2}{2}{В равнобедренной трапеции $ABCD$ боковое ребро $BC$ и маленькое основание $CD$ равны $2$ см. Кроме того, $BC$ перпендикулярно $AC$. Найдите площадь трапеции.}

\pr{2}{3}{В классе 15 девочек, 16 учеников имеют темные волосы, 17 --- кареглазые и 18 отличников. Новый учитель математики, зная, сколько учеников в классе, но не зная класс, смог точно сказать, что там учится кареглазая темноволосая девочка-отличница. Какое наибольшее количество человек может учиться в этом классе?}\newpage

\pr{2}{4}{На какое наименьшее число прямоугольных треугольников можно разрезать правильный пятиугольник?}

\pr{3}{4}{$a_1=1$ и $a_1+\ldots +a_n=n^2\cdot a_n$ для всех натуральных $n$. Найдите $a_{1000}$.}

\pr{2}{6}{Найдите наименьшее простое число, сумма цифр которого является нечетным составным числом.}\newpage

\pr{1}{4}{Сколько существует таких натуральных $n$, что  $n\leq 2020$ и $1^n+2^n+3^n+4^n$ кратно 10?}

\pr{5}{5}{Мы разбиваем доску $8\times 8$ на квадратики $2\times 2$ и трёхклеточные уголки. Какое наименьшее количество квадратиков может получиться?}
\pr{2}{5}{В цехе работало несколько станков. После реконструкции количество станков уменьшилось, причем число процентов, на которое оно уменьшилось, оказалось равно числу оставшихся станков. Какое наименьшее число станков могло быть до реконструкции?}\newpage

\pr{3}{3}{В классе 10 учеников. Для дежурства наугад выбирают двоих. Вероятность того, что оба дежурных окажутся мальчиками, равна $\frac{1}{3}$. Какова вероятность того, что это будут две девочки?}


\pr{3}{6}{Числа от 1 до 2020 разбиты на две группы. В первой группе находятся числа, ближайший к которым квадрат чётный, а в другой --- числа, ближайший к которым квадрат нечётный. Чему равна разность между суммами в этих группах?}\newpage

\pr{3}{5}{Петя заметил, что у всех его 25 одноклассников различное число друзей в этом классе. Сколько друзей в классе могло быть у Пети?}

\pr{4}{4}{По кругу висит гирлянда из 222 лампочек. Мы можем постучать по любой из лампочек, и она включится, если была выключена, и выключится, если была включена. То же произойдёт и с двумя соседними лампочками. Сколько различных состояний гирлянды мы можем получить?}\newpage

\pr{4}{5}{На сторонах треугольника во внешнюю сторону построены квадраты, 6 вершин квадратов, не являющиеся вершинами исходного треугольника лежат на одной окружности. Найдите углы треугольника.}

\pr{4}{6}{На доске написаны числа от 1 до 30. За одну операцию разрешается взять три положительных числа $a, b, c$, идущих подряд, и поменять их на $b-1, c-1, a-1$ в указанном порядке. Какое максимальное число таких операций можно сделать?}\newpage


\pr{5}{6}{На белом столе лежит стопка из 20 монет. Кроме того, есть два пустых чёрных стола. Каждым ходом можно переложить верхнюю монету с любого стола на верхнюю монету другого стола (или на сам другой стол, если он пуст). За какое наименьшее число ходов можно переложить все монеты на белом столе в порядке, противоположном исходному?}
\aans\aans

\newpage

\pr{6}{6}{У воинов есть три характеристики от 0 до 100: сила, здоровье и скорость. Когда два героя сражаются, первым ходит тот, у кого больше скорость (если скорости равны, они ходят одновременно). Воины по очереди уменьшают здоровье оппонента на свою силу. Если у война становится 0 здоровья, он проигрывает. У всех героев на арене сумма характеристик 100. Какая наименьшая сумма характеристик у героя, который побеждает всех героев на арене?}
\aans\aans
\vskip-10pt

\dup[3]{
\newpage
\dup[2]{
\hdr{Правила кровавого домино}

В игре участвуют 6 команд. Изначально у каждой из них 20 баллов.\\

Игра состоит из 3 туров. В начале каждого тура вы берёте себе 7 задач, которые вы ещё не брали. Каждая задача соответствует какой-то доминошке (1-1, 1-2, \ldots, 6-6 --- всего 21 задача), чем больше сумма номеров доминошек, тем задача сложнее. Затем вы можете решать эти задачи (а также те, которые вы получили раньше) в течение 25 минут тура. Кроме того, по каждой задаче вы можете один раз узнать, верный ли ваш ответ у жюри.\\

После решения задач каждого тура происходит битва. Битва состоит из 5 боёв, в результате которых вы сразитесь с каждой другой командой. Для сражения вы сдаёте лист с номером (доминошкой) задачи, вашим ответом, и подписью, какое число из доминошки --- атака, а какое --- защита (например, если вы сдаёте задачу 2-4, вы можете сделать атаку 2, а защиту 4, или наоборот), ваш соперник делает то же самое. Затем если атака вашей доминошки больше, чем защита соперника, вы получаете очки, равные разности этих чисел, а если защита вашей доминошки меньше, чем атака соперника, вы теряете очки, равные разности этих чисел (очки могут быть отрицательными). \textbf{Если ответ неверный, и атака, и защита считаются равными 0. Кроме того, каждую доминошку можно использовать не больше 1 раза.}\\

В битве после 3-го тура можно использовать неиспользованные доминошки. Для этого вы дополнительно дописываете к сражению сколько угодно номеров задач, ответов к ним, а так же <<атаку>> или <<защиту>>. За каждую задачу с верным ответом атака или защита (по вашему выбору) у вашей доминошки увеличивается на 1 (даже если ответ на исходную задачу неверен), иначе не изменяется.\\
\vfill\hrule\vfill
}
}

\end{document}
