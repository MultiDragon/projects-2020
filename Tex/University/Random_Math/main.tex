\documentclass[12pt,a4paper]{article}
\usepackage{tpl}
\begin{document}
\dbegin[17 сентября 2020 г.]{Матфак ВШЭ}{НИС <<Основные понятия математики>>}{Бурман Юрий Михайлович}

\hdr{Квадратичные вычеты}

Пусть зафиксировано число $m$. Будем пытаться понять, какие остатки по модулю $m$ являются полными квадратами. Причём можно считать, что $m=p$ простое и $p>2$.\\

\definition{Квадратичный вычет} такое число $a$, что существует такое $b$, что $b^2\equiv a\mod p$.

\theorem Если $p\equiv q\mod 4a$, то $a$ --- одновременно вычет или невычет по модулям $p$ и $q$.\label{main}

\lemma~\ref{main} выполняется для $a=-1$, т.е. $-1$ --- вычет при $p=4k+3$ и невычет при $p=4k+1$.\label{minusone}

\definition{Символ Лежандра} выражение $\lrp{\frac{a}{p}} $, равное 1, если $a$ --- квадратичный вычет по модулю $p$ и $-1$, если невычет (и не определённое, если $p\mid a$).

\lemma Выполняется $\lrp{\frac{a}{p}}=a^{\frac{p-1}{2}}$ (в частности, отсюда будет следовать~\ref{minusone}).

\theorem $\mathbb F _p^*=\{1,2,\ldots ,p-1\}$ циклическая. То есть, существует $\epsilon\in \mathbb F _p^*$ такое, что $\ord\epsilon=p-1$. Оно называется \textit{первообразным корнем}.

\proof Пусть $\ord a=k$. Тогда у всех чисел вида $a^l$ при $(l,p-1)=1$ порядок $k$. Заметим, что других чисел с порядком $k$ нет. Действительно, рассмотрим многочлен $x^k-1$. Его корни --- это числа вида $a^t$ и только они, потому что они подходят, а других нет по теореме Безу. Обозначим $\psi(d)$ --- количество чисел с порядком $d$. Мы знаем, что если $\psi(d)\neq 0$ (и $d\mid p-1$), то $\psi(d)=\phi(d)$. Кроме того, $\sum_{k\mid p-1}\psi(k)=p-1$, и $\sum_{k\mid n}\phi(k)=p-1$ для любого $n$. Это значит, что на самом деле $\psi(d)=\phi(d)$ для всех $d\mid p-1$.\QEDA\\

\end{document}
