\documentclass[12pt,a4paper]{article}
\usepackage{tpl}
\usepackage{youngtab}
\begin{document}
\dbegin[18 сентября 2020 г.]{Матфак ВШЭ}{Проект по парковке}{Кудрявцев А.А.}

\textbf{Задача 0.} На парковке есть по одному месту со стоимостями от 1 до $n$ долларов (т.е. всего $n$ мест). Кроме того, есть $n$ машин, у каждой из которых есть $m_i\leq n$ долларов. Машины приезжают по очереди, и каждая из них встаёт на самое дорогое свободное место, цену которого она может заплатить, а если не может встать, то уезжает. Последовательность $m_i$ называется хорошей, если все машины могут встать на какие-то места. Доказать, что если $m_i$ хорошая, то любая перестановка $m_i$ тоже.

% \proof \textbf{(набросок)} Будем называть \textit{$F(a_i)$ --- диаграммой последовательности $a_i$} такую диаграмму Юнга: в $j$-й строке стоит столько элементов, сколько элементов $a_i$ равны $j$ или меньше, а высота диаграммы --- максимум этой последовательности (то есть диаграмма $m_i$ из условия будет иметь $n$ элементов на $n$-й строке). Докажем такой факт: последовательность $m_i$ хорошая тогда и только тогда, когда $F(m_i)$ лежит целиком внутри $F(c_i)$, где $c_i$ --- это последовательность стоимостей парковок (т.е. в условии $c_i=i$ для $i\leq n$), причём $|m_i|=|c_i|$ (скорее всего, можно модифицировать доказательство, чтобы это стало несущественно). Очевидно, это решает задачу, т.к. диаграмма инвариантна относительно перестановок.
% Очевидно, что если $F(m_i)$ не лежит целиком внутри, то последовательность плохая. Действительно, пусть $(i,j)\in F(m)\setminus F(c)$. Тогда у нас есть хотя бы $j$ машин, у каждой из которых $i$ или меньше долларов, а <<дешёвых мест>> на парковке не больше $j-1$. Поэтому мы будем доказывать, что если $F(m_i)$ лежит целиком внутри, то последовательность хорошая.
% Будем доказывать этот факт индукцией по $n$. База для $n=1$ очевидна. Назовём точку $(i,j)\in F(m_i)$ критической, если $(i+1,j)\not\in F(c_i)$. Заметим, что хотя бы одна критическая точка у нас есть --- это $(n,t)$ (где $n$ --- количество машин, а $t$ --- максимальная цена парковки). Теперь можно рассмотреть два случая:
% \begin{itemize}
	% \item Критическая точка ровно одна. Это значит, что для любого $k<t$ машин с не более чем $k$ долларами меньше, чем парковок со стоимостью не более чем $k$. Рассмотрим все машины, кроме машины с номером $n$. Если забрать у каждой из них по 1 доллару, и рассмотреть самую дешёвую $n-1$ парковку, то по предположению индукции им удастся разъехаться. Тогда последняя машина сможет занять место с номером $n$.
	%\item Есть критические точки, помимо $(n,t)$. Пусть это точки $(i_1,j_1),(i_2,j_2),\ldots ,(i_s,j_s)=(n,t)$. Очевидно, что в итоге самые дешёвые $i_k$ машин должны занять все парковки со стоимостью до $j_k$ для любого $k$, и это получится сделать по предположению индукции для срезов диаграмм $F(m_i),F(c_i)$, то есть для диаграмм вида $\{(i,j)\in F(m_i)\mid i_{k-1}<i\leq i_k\}$.
%\end{itemize}

%В обоих случаях мы смогли уменьшить $n$.\QEDA\\

\textbf{Набросок решения.} Пусть $n$ --- количество машин, $t$ --- максимум из всех стоимостей парковок и всех стоимостей машин. Нам понадобятся диаграмма от множества стоимостей парковок (обозначим её $X$; определение написано под рисунком) и диаграмма от множества стоимостей машин (обозначим её $Y$). Например, диаграмма от множества $\{1,2,2,4,5,5,5,8,9\}$ будет выглядеть так (она соответствует 9 парковкам с этими стоимостями):

\yng(1,2,2,2,5,6,6,8,9)

Столбцы каждой из таких диаграмм обозначают машины (или парковочные места), и высота столбца --- это $t+1-k$, где $k$ --- стоимость парковки (или количество денег у владельца этой машины). Другое определение: точка $(x,y)$ лежит в диаграмме мультимножества $X$ тогда и только тогда, когда $\#\{q\in X\mid q\geq t+1-y\}\geq x$ (нумерация слева направо, сверху вниз, т.е. верхний левый угол --- это $(1,1)$).

Будем доказывать, что $m_i$ хорошее тогда и только тогда, когда $Y\subset X$ (например, для множества выше это будет выполняться для парковки, данной в условии --- её диаграмма <<диагональная>>).

В сторону <<только тогда>> это очевидно. Действительно, если $(i,j)\in Y\setminus X$, то есть хотя бы $t+1-i$ машин, у каждой из которых $j$ или меньше долларов, а мест стоимостью $j$ долларов или меньше, не больше, чем $t-i$.

Докажем в сторону <<тогда>> по индукции по $n$. База для $n=1$ очевидна.

Назовём $(i,j)$ критической точкой, если $(i,j)\in Y,(i+1,j)\not\in X,(i,j-1)\not\in X$. То есть $(i,j)$ лежит на верхней правой границе как диаграммы парковок, так и диаграммы машин. Пусть существует критическая точка с $j\neq t$ (то есть не лежащая в самой дорогой машине). Пусть это точка $(i_0,j_0)$. Тогда можно рассмотреть в качестве отдельных парковок <<$P_1=$ все места стоимостью $\geq j_0$>> и <<$P_2=$ все места стоимостью $<j_0$>> (и то же самое деление на $M_1,M_2$ для множества машин). Тогда $M_1$ хорошее (для парковки $P_1$) по предположению индукции, значит, машины из $M_1$ все поместятся на парковку $P_1$ (в $P_2$ они не попадут, потому что там слишком дёшево). $M_2$ тоже хорошее (для парковки $P_2$), значит, все машины поместятся на парковку.

Вторая часть: пусть критических точек с $j\neq t$ нет. Тогда сделаем такую операцию: удалим из $Y$ столбец, соответствующий самой последней машине в очереди, а из $X$ удалим самый левый (т.е. самый большой) столбец. Получится, что для новых диаграмм снова выполняется $Y'\subset X'$, потому что после удаления столбца машины всё, что справа от него, сдвинулось влево на 1, значит, соприкосновений с <<диагональю>> нет, т.е. эту диагональ можно убрать. Для новой пары диаграмм верно предположение индукции, т.е. на этой новой парковке могут разъехаться машины, а последнюю машину можно припарковать на самое левое, т.е. дешёвое место (или на более дорогое, если такое осталось).\QEDA\\

\end{document}
