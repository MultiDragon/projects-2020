\documentclass[12pt,a4paper]{article}
\usepackage{tpl}
\begin{document}
\dbegin[2 сентября 2020 г.]{Матфак ВШЭ}{Алгебра, 1 курс}{Фейгин Евгений Борисович}

\textit{Формула оценки: $\dfrac{D+C+K+2E}{5}$, где $D,C,K,E$ --- оценки за д/з, КР, коллоквиум и экзамен соответственно.}\\

\definition{Абелева группа} множество $A$ с определённой на нём операцией $+$ со следующими свойствами:

\begin{itemize}
	\item $\forall a,b:a+b=b+c$;
	\item $\forall a,b,c:(a+b)+c=a+(b+c)$;
	\item $\exists 0\forall a:a+0=a$;
	\item $\forall a\exists (-a):a+(-a)=0$.
\end{itemize}

\definition{Кольцо} множество $A$ с операциями $+$ и $\times $ со следующими свойствами:

\begin{itemize}
	\item $(A,+)$ --- группа;
	\item $a\times (b+c)=a\times b+b\times c$;
	\item $(b+c)\times a=b\times a+c\times a$.
\end{itemize}

Кроме того, у $\times $ могут быть такие дополнительные свойства:

\begin{itemize}
	\item $\exists 1:\forall a:a\times 1=1\times a=a$ (если есть единица);
	\item $\forall a,b:a\times b=b\times a$ (если коммутативное кольцо);
	\item $\forall a,b,c:a\times (b\times c)=(a\times b)\times c$ (если ассоциативное кольцо);
	\item $\forall a,b:a\times b=0\implies a=0\lor b=0$ (если нет делителей нуля).
\end{itemize}

\definition{Целостное кольцо} ассоциативное коммутативное кольцо с единицей без делителей нуля.

\definition{Поле} коммутативное ассоциативное кольцо с $1$, такое, что $0\neq 1$ и $\forall a\neq 0\exists a^{-1}:aa^{-1}=1$.

\textit{Замечание.} Отсутствие делителей нуля в кольце не гарантирует, что это поле.

\definition{Подгруппа абелевой группы $A$} множество $B\subset A$, со следующими свойствами:

\begin{itemize}
	\item $0\in B$;
	\item $a\in B\implies (-a)\in B$;
	\item $a,b\in B\implies a+b\in B$.
\end{itemize}

\definition{Подкольцо} подгруппа $B\subset A$ такая, что $a,b\in B\implies a\times b\in B$.

\definition{Подполе} подкольцо $B\subset A$ такое, что $1\in B$ и $a\in B\implies a^{-1}\in B$.

\definition{Комплексные числа} множество $\mathbb C=\{a+bi\mid a,b\in \mathbb R\}$ (здесь $i$ --- формальный символ) с операциями сложения и умножения, определёнными следующим образом:

\begin{itemize}
	\item $(a+bi)+(c+di)=(a+c)+(b+d)i$;
	\item $(a+bi)\times (c+di)=(ac-bd)+(bc+ad)i$.
\end{itemize}

\theorem $\mathbb C$ --- поле.

\proof Вначале докажем, что $\mathbb C$ --- кольцо (это очевидно). Кроме того, \[
	a^2+b^2\neq 0\implies (a+bi)\lrp{\frac{a}{a^2+b^2}-\frac{b}{a^2+b^2}i}=1
,\] значит, это поле.\QEDA
\newpage

\definition{Вещественная часть} число $\Re(a+bi)=a$.

\definition{Мнимая часть} число $\Im(a+bi)=b$.

\definition{Модуль комплексного числа} число $N(a+bi)=\sqrt{a^2+b^2}$.

\definition{Аргумент комплексного числа} множество $\Arg(a+bi)$ чисел $\varphi$ таких, что $a+bi=N(a+bi)(\cos\varphi+i\sin\varphi)$.

\textbf{Тригонометрическая запись числа.} Будем записывать \[
	z=a+bi=N(z)(\cos \Arg(z)+i\sin \Arg(z))
.\]
Тогда получится, что \[
	z_1z_2=(N(z_1)N(z_2))\lrp{\cos (\Arg(z_1)+\Arg(z_2))+i\sin(\Arg(z_1)+\Arg(z_2))}
.\]

\definition{Автоморфизм поля} отображение $f:K\to K$ такое, что $f(a)+f(b)=f(a+b)$ и $f(a)f(b)=f(ab)$. Автоморфизмы кольца и абелевой группы определяются аналогично.\\

\definition{Изоморфизм групп} отображение $f:A\to B$ такое, что $f(0_A)=f(0_B)$ и $f(a_1+_A a_2)=f(a_1)+_B f(a_2)$. Если такое отображение существует, то $A$ и $B$ называются изоморфными.\\

Заметим, что $a+bi\mapsto \overline{a+bi}:=a-bi$ --- автоморфизм. Множество его фиксированных точек --- это $\mathbb R$, и легко доказать, что $z\overline{z},z+\overline{z}\in \mathbb R$.\\

Рассмотрим уравнение $z^n=1$. Если $z=\cos\varphi+i\sin\varphi$, то $z^n=\cos n\varphi+i\sin n\varphi=1$, т.е. $\varphi=\frac{2\pi k}{n}$. Это будет $n$ корней (для $k=0,\ldots ,n-1$; будем обозначать $\xi_r=\cos\frac{2\pi r}{n}+i\sin\frac{2\pi r}{n}$), и они делят окружность $N(z)=1$ на $n$ равных частей. Понятно, что если $z_1,z_2$ --- корни, то и $z_1z_2$ тоже, кроме того, $1$ --- корень. Тогда это абелева группа по умножению, которая изоморфна $\mathbb Z / n\mathbb Z$ (по сложению).\\

\definition{Первообразный корень из 1} такой корень $\xi_k$, что $\forall n\exists m:(\xi_k)^m=\xi_n$.

\lemma $\xi_k$ ($r$-и степени) первообразный тогда и только тогда, когда $(k,r)=1$.\\

\definition{Фактор-множество} $M /R$ множество классов эквивалентности на множестве $M$ по отношению эквивалентности $R$.

\definition{Отображение проекции} функция $\pi:M\to M /R$, переводящая любой элемент $a$ в множество $R(a)$ элементов, эквивалентных $a$.

\lemma $\pi$ --- сюръекция и $\pi^{-1}(x)=\{a\in M,a\sim x\}$.\\

Пусть на $M$ есть операция $*$. Будем говорить, что $*$ согласована с отношением $R$, если из того, что $a\sim a',b\sim b'$ следует, что $a*b\sim a'*b$. Тогда на $M /R$ возникает индуцированная операция $*$.

\textit{Если $*$ согласована с $R$, то индуцированная операция наследует многие свойства $*$, в частности: ассоциативность, коммутативность, наличие нейтрального элемента.}\\

\theorem $ \mathbb Z /n\mathbb Z$ поле тогда и только тогда, когда $n$ простое.

\proof Пусть $n=n_1*n_2$. Тогда $[n_1]_n[n_2]_n=[n]_n=[0]_n$.

С другой стороны, пусть $n$ простое. Тогда для любого $m=1,2,\ldots ,n-1$ выполняется $(m,n)=1$, тогда $\exists u,v:um+vn=1\iff [m]_n [u]_n= [1]_n$. Тогда $u$ обратно к $m$.\QEDA\\

\definition{Характеристика поля} минимальное такое $k\in \mathbb N$, что $\underbrace{1+\ldots +1}_k=0$ (имеются в виду 0 и 1 из этого поля). Если такого $k$ нет, то характеристика равна 0.

\lemma Если $\mathbb K$ --- поле, то $\chr\mathbb K$ --- 0 или простое число.

\proof Пусть $\chr\mathbb K=n=n_1n_2$. Тогда $0=\underbrace{1+\ldots +1}_n=(\underbrace{1+\ldots +1}_{n_1})(\underbrace{1+\ldots +1}_{n_2})$, значит, у нас есть делители нуля.\QEDA\\

\newpage

\definition{Евклидово кольцо} целостное кольцо $K$ с функцией нормы $N:K\setminus 0\to Z_{\geq 0}$ со следующими свойствами:

\begin{itemize}
	\item $N(ab)\geq N(a)$, причём равенство только если $b$ обратим.
	\item $\forall a,b\in K,b\neq 0\exists q,r\in K:a=bq+r,N(r)<N(b)$.
\end{itemize}

\textbf{Примеры.}

\begin{itemize}
	\item $\mathbb Z;N(x)=|x|$.
	\item Пусть $F$ --- поле, тогда $F[x]$ с функцией $N(P)=\deg P$ подходит.
\end{itemize}

\lemma $F[x]$ --- евклидово кольцо.\label{polyeuclid}

\proof Очевидно, что это целостное кольцо. Очевидно также, что $\deg fg\geq \deg f\deg g$ и равенство, только если какой-то из многочленов 0 степени, т.е. обратим. Докажем деление с остатком. Пусть $f=\sum f_ix^i,g=\sum g_ix^i,n=\deg f\geq \deg g=m$. Тогда рассмотрим $k=f-g\frac{f_n}{g_m}x^{n-m}$. Его степень меньше $n$, кроме того, $k\equiv f\mod g$, значит, можно проделать алгоритм Евклида. \QEDA

\textit{Замечание.} В этой лемме необходимо, чтобы $F$ было полем. Например, в $\mathbb Z[x]$ не получится разделить $3x$ на $2x$ с остатком.\\

\theoremn{Безу} остаток от деления $f(x)$ на $x-c$ равен $f(c)$.

\proof Следует из \ref{polyeuclid}.\QEDA\\

\theorem Многочлен $f(x)\in F[x]$ не может иметь в $F$ более $\deg f$ корней.

\proof Пусть $c_1,c_2$ --- корни этого многочлена. Тогда $f=(x-c_1)f_1$ и $f(c_2)=(c_2-c_1)f_1(c_2)$. Так как $c_1-c_2\neq 0$, то $f_1(c_2)$ имеет корень $c_2$. Индукция по $\deg f$.\QEDA\\

\lemma Пусть $F$ --- бесконечное поле. Тогда разные многочлены в $F[x]$ определяют разные функции на $F$.

\proof Пусть $f_1,f_2\in F[x]$ определяют одну и ту же функцию. Тогда $f_1-f_2=0\forall x$. Но $f_1-f_2$ имеет конечную степень, а $F$ бесконечное. Противоречие. \QEDA\\

\definition{Кольцо формальных степенных рядов} множество сумм вида \[
	K[[x]]=\{a_0+a_1x+a_2x^2+\ldots \mid x_i\in K\}
.\]

\definition{Кольцо рядов Лорана} множество сумм вида \[
	K((x))=\{x^{-r}(a_0+a_1x+a_2x^2+\ldots) \mid x_i\in K,r\in \mathbb N\}
.\]

Вернёмся к $K[x]$, причём будем считать, что $K$ --- это поле. 

\definition{Неприводимый многочлен} простой элемент в кольце $K[x]$, т.е. такой многочлен $P$, что $P=fg\implies \deg f\cdot \deg g=0$.

\definition{Факториальное кольцо} кольцо, в котором выполняется основная теорема арифметики, т.е. в котором каждый элемент раскладывается в конечное произведение простых единственным способом с точностью до перестановки и умножения на обратимые.

\lemma Любое евклидово кольцо факториальное.

\proof Разложение есть: пусть $x\in K$ не простое, тогда $\exists p,q:x=pq$ и $p,q$ необратимые. Тогда $N(x)>N(p)$ и $N(x)>N(q)$, индукция по норме $x$.

Разложение единственно: линейное представление для НОД позволяет доказать, что если $q\mid ab$ и $(a,q)=1$, то $q\mid b$, откуда и следует утверждение.\QEDA\\

\definition{Гомоморфизм колец} функция $\phi:K\to L$ такая, что $\phi(k_1)+\phi(k_2)=\phi(k_1+k_2)$ и $\phi(k_1)\phi(k_2)=\phi(k_1k_2)$.

\definition{Изоморфизм колец} биективный гомоморфизм.

\textbf{Пример} $\mathbb R[x] /(x^2+1)$ изоморфно $\mathbb C$ с $\phi([ax+b])=ai+b$.

\newpage

\theorem $K[x] /f$ является полем тогда и только тогда, когда $f$ неприводим.

\proof Пусть $f$ приводим. Тогда $f=pq,\deg f>\deg p,\deg f>\deg q$. Тогда $[p]\neq [0]\neq [q]$, но $[pq]=0$, т.е. в этом кольце есть делители 0.

Теперь пусть $f$ неприводим. Докажем, что у любого класса $[g]_f\neq [0]$ есть обратный. Это так, т.к. $\exists u,v:gu+fv=1$ (т.к. $f$ неприводим $\implies (f,g)=1$), тогда $[gu]_f=[1]_f$. \QEDA\\

\theorem Для любых $n,p$ существует неприводимый многочлен степени $n$ над $\mathbb Z /p\mathbb Z$, т.е. существует поле из $p^n$ элементов. Кроме того, все поля, получающиеся таким образом, изоморфны.\\

\definition{Произведение колец} кольцо $K\times L=\{(a,b)\mid a\in K,b\in L\}$ с операциями $(a,b)+(c,d)=(a+c,b+d)$ и $(b,c)\cdot (c,d)=(ac,bd)$.

\theoremn{Китайская теорема об остатках} Пусть даны $n_1,\ldots ,n_k,r_1,\ldots ,r_k\in \mathbb N$, причём $(n_i,n_j)=1$ и $0\leq r_i<n_i$. Тогда $\exists N:R\equiv r_i\mod n_i$, и если $N_1,N_2$ удовлетворяют этому свойству, то $N_1\equiv N_2\mod n_1\ldots n_k$.\label{crt}

\textbf{Алгебраическая переформулировка.} Пусть $n_1,\ldots ,n_k\in \mathbb N$ и $(n_i,n_j)=1$. Тогда \[
	\mathbb Z /(n_1n_2\ldots n_k\mathbb Z)\simeq (\mathbb Z /n_1Z)\times \ldots \times (\mathbb Z /n_k\mathbb Z).
\]
\textbf{Почему отсюда следует~\ref{crt}.} Рассмотрим $[R]_{n_1\ldots n_k}=\phi^{-1}([r_1]_{n_1}\times [r_2]_{n_2}\ldots \times [r_k]_{n_k})$.

\proof Рассмотрим $\phi(t)=(t\mod n_1,t\mod n_2,\ldots ,t\mod n_k)$. Заметим, что $\phi(p+q)=\phi(p)+\phi(q)$ и $\phi(pq)=\phi(p)\phi(q)$. Докажем, что $\phi$ инъективно и сюръективно.

\textbf{Инъективность.} Пусть $\phi(a)=\phi(b)$. Тогда $a\equiv b\mod n_i$. Т.к. $(n_i,n_j)=1$, то $a\equiv b\mod n_1\ldots n_k$.

\textbf{Сюръективность.} Следует из количества элементов и инъективности.\QEDA\\

\theoremn{КТО для многочленов} Пусть $K$ --- евклидово кольцо (в частности, работает для многочленов), $f_1,\ldots ,f_k\in K$ и $(f_i,f_j)=1$. Тогда $K /(f_1f_2\ldots f_k)\simeq \prod_j K / (f_j)$.

\proof Рассмотрим ту же самую функцию $\phi$, как и в~\ref{crt}. Она является инъективным гомоморфизмом по той же самой причине. Докажем сюръективность. Пусть есть многочлены $r_i$. Определим $F_i=\prod_{j\neq i} f_j$. Мы знаем, что $(F_i,f_i)=1$, значит, $\exists a_i,b_i:a_iF_i+b_if_i=1$. Рассмотрим $P=\sum v_ir_iF_i$. Тогда $P= f_i(\ldots )+F_iv_ir_i\equiv r_i\mod f_i$.\QEDA\\

% \definition{Циклическая группа} группа $C_n$, изоморфная $\mathbb Z /n\mathbb Z $ по сложению.
% 
% \definition{Группа преобразований} группа всех биекций $M\to M$ с операцией композиции.
% 
% \definition{Группа перестановок} группа $S_n$, изоморфная группе преобразований конечного множества.

\definition{Циклическая группа} группа $S$, т.ч. $\exists a\in S\forall b\in S\exists n\in \mathbb Z :b=a^n$. Любая циклическая группа изоморфна либо $\mathbb Z $, либо $\mathbb Z /n\mathbb Z $ по сложению.

\definition{Гомоморфизм групп} отображение $\phi:A\to B$ такое, что, $\phi(e)=e$ и $\phi(a_1+a_2)=\phi(a_1)+\phi(a_2)$. Множество всех гомоморфизмов обозначается $\Hom(A,B)$ (которое является группой по сложению, т.е. $(\phi+\psi)(a)=\phi(a)+\psi(a)$).

\definition{Ядро гомоморфизма} множество $\Ker\phi=\{a\in A\mid \phi(a)=e\}$.

\definition{Подгруппа} подмножество группы, которое содержит $e$ и замкнуто относительно сложения и взятия обратного.

\definition{Порядок элемента} такое минимальное число $k$, что $\underbrace{a+\ldots +a}_k=e$.

\definition{Произведение групп} группа $K_1\times \ldots \times K_n$, состоящая из элементов $(k_1,\ldots ,k_n),k_i\in K_i$, с поэлементным сложением.

\lemma Пусть $L$ целостное и $\phi:K\to L$ --- гомоморфизм. Тогда $\phi\equiv 0$ или $\phi(1)=1$.

\proof $\phi(1)=\phi(1\cdot 1)=\phi(1)^2$. Отсюда (т.к. кольцо целостное) либо $\phi(1)=1$, либо $\phi(1)=0$. В первом случае утверждение доказано, во втором $f(k)=f(1)f(k)=0$.\QEDA\\

\lemma Пусть $G$ циклическая и $H\subset G$ --- подгруппа. Тогда $H$ циклическая.

\newpage 

\theorem Пусть $\mathbb F $ --- поле, и $A\subset \mathbb F^*$ --- конечная подгруппа. Тогда $A$ циклическая.\label{primitive}

\lemma Пусть $A$ --- абелева группа, $b_1,b_2\in A,\ord b_1=m_1,\ord b_2=m_2,(m_1,m_2)=1$. Тогда $\ord(b_1b_2)=m_1m_2$.\label{ordshift}

\proof Пусть $(b_1b_2)^s=e$. Тогда $b_1^s=b_2^{-s}\implies b_1^{sm_2}=b_2^{-sm_2}=e$, откуда $sm_2$ делится на $m_1$. Значит, $s$ делится на $m_1$. Аналогично $s$ делится на $m_2$. С другой стороны, $(b_1b_2)^{m_1m_2}=1$.\QEDA\\

\lemma Пусть $A\subset \mathbb F ^*$ и $\max_{a\in A}\ord a=m$, кроме того, $\forall b\in A:\ord b\mid m$. Тогда $A\simeq C_m$.\label{cyclic}

\proof Пусть все элементы $A$ являются корнями $x^m-1=0$. Отсюда следует, что $m\geq |A|$. Но мы знаем, что $m\leq |A|$, значит, $m=|A|$. Тогда $A=\{1,a,\ldots ,a^{m-1},a^m=1\}$ (где $a$ --- элемент с максимальным порядком), т.е. циклическая.\QEDA\\

\prooft{primitive} Докажем, что выполняется условие~\ref{cyclic}. Для этого достаточно доказать, что $\forall x,y\in A\exists z\in A:\ord z=[\ord x,\ord y]$. Пусть \[
	x=\prod_{i=1}^s p_i^{u_i},y=\prod_{i=1}^s p_i^{v_i};p_i\neq p_j;u_i,v_i\in \mathbb Z _{\geq 0};u_i+v_i>0.
\]

Обозначим \[
	l_1=\prod_{i:u_i>v_i}p_i^{u_i};l_2=\prod_{i:u_i\leq v_i}p_i^{v_i};m_1=l_1k_1;m_2=l_2k_2.
\]

Тогда $(l_1,l_2)=1,[m_1,m_2]=l_1l_2$. Рассмотрим $a_3=a_1^{k_1}\cdot a_2^{k_2}$. Тогда по~\ref{ordshift} $\ord a_3=[m_1,m_2]$.\QEDA\\

\end{document}
