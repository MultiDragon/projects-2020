\documentclass[12pt,a4paper]{article}
\usepackage{tpl}
\begin{document}
\dbegin[16 сентября 2020 г.]{Матфак ВШЭ}{Решения алгебры}{Кудрявцев Александр, 1 курс}

\hdr{Лист 1, 16 сентября 2020 г.}\vskip20pt

\z Нарисуем это число на комплексной плоскости:

\begin{tikzpicture}
	\qsetscale{0.5}
	% \qgrid{10}{10}
	\qcoord O55
	\qcspoint O
	\qcoord A89
	\qccircle OA
	\qcoord F{10}5
	\qcoord G5{10}
	\qcoord X{13}9
	\qctriangle[fill,red] OXF
	\qctriangle[fill,blue] OXA
	\qcpoint[right] F{$F(1)$}
	\qcpoint[right] X{$X(x)$}
	\qcgMidpoint MOX
	\qcpoint[left] A{$A(\cos\varphi+i\sin\varphi)$}
	\qcpoint G{$G(i)$}
	\qcspoint M
\end{tikzpicture}

Заметим, что $\triangle OFX=\triangle OAX$, откуда $\arg x=\frac{\varphi}{2}$. Кроме того, $|x|=OX=2OM=2OF\cos \frac{\varphi}{2}=2\cos \frac{\varphi}{2}$.\QEDA\\

\z $z^n=i\iff r^n(\cos\varphi+i\sin\varphi)^n=i\iff r^n(\cos n\varphi+i\sin n\varphi)=i$. Так как $|i|=1$, то $r=1$. Тогда $\cos n\varphi+i\sin n\varphi=i\iff n\varphi=\frac{\pi}{2}+2\pi k\iff \varphi=\frac{\pi}{2n}+2\pi \frac{k}{n}$.\QEDA\\

\z Рассмотрим число $z_0=\sum_{k=1}^n(\cos(kx)+i\sin(kx))$. С одной стороны, нам нужно найти $\Im(z_0)$. С другой стороны,

\begin{equation*}
	z_0=\sum_{k=1}^n((z_1)^k)=z_1\cdot \frac{z_1^k-1}{z_1-1}=z_1\cdot (z_1^k-1)\cdot (\cos x-i\sin x-1)\cdot \frac1{(\cos x+1)^2+(\sin x)^2}
\end{equation*}
где $z_1=\cos x+i\sin x$. Тогда

\begin{align*}
	\Im z_0=\frac{1}{2-2\cos x}\cdot \Im(\cos x+i\sin x)(\cos nx+i\sin nx-1)(\cos x-i\sin x-1)=\\
	=\frac{1}{2-2\cos x}\cdot\Im (1-\cos x-i\sin x)(\cos nx+i\sin nx-1)=\\
	=\frac{1}{2-2\cos x}\cdot\Im(\ldots +i(\sin nx-\cos x\sin nx-\sin x\cos nx+\sin x))=\frac{\sin x+\sin nx-\sin(n+1)x}{2-2\cos x}.\QEDA
\end{align*}

\z Если $n$ нечётно, то множество квадратов корней $n$-й степени из 1 совпадает с множеством корней, а их сумма равна 0 (т.к. эти корни соответствуют векторам из $O$ в вершины правильного $n$-угольника). Если чётно, то множество этих квадратов --- это множество корней $\frac{n}{2}$-й степени, взятое два раза. Их сумма тоже равна 0. Ответ: 0.\QEDA\\

\z Заметим, что множество чисел вида $(5k+3)+i(5l+4),k,l\in \mathbb Z$, замкнуто относительно умножения, т.к. $(5k_1+3+i(5l_1+4))(5k_2+3+i(5l_2+4))=5k'+9+5k''-16+i(5l'+24)=5(k'+k''-2)+3+i(5(l'+4)+4)$. Кроме того, $\frac{2+i}{2-i}=\frac{(2+i)^2}{5}=\frac{3+4i}{5}$. Значит, если $(\frac{2+i}{2-i})^n=1$, то $\frac(3+4i)^n=5^n$, но левая часть вида $5k+3+i(5l+4)$, т.е. её мнимая часть ненулевая.\QEDA\\

\newpage
\hdr{Лист 2, 23 сентября 2020 г.}\vskip20pt

\setcounter{probs}{0}
\z Обозначим $\varphi=\frac{2\pi}{5}$. Мы знаем, что $1=(\cos\varphi+i\sin\varphi)^5=z^5$. То есть либо $z=1$ (что неправда), либо $z^4+z^3+z^2+z+1=0$. Сделаем замену $w=z+\frac{1}{z}$. Тогда $z^2+\frac{1}{z^2}=w^2-2$, т.е. $w^2+w-1=0$. Корни этого уравнения --- $-\frac{1}{2}\pm\frac{\sqrt5}{2}$. Мы знаем, что $z+\frac{1}{z}>0$. Значит, $z+\frac{1}{z}=\frac{\sqrt5-1}{2}$. Нам нужна только вещественная часть корня, которая равна $\frac{\sqrt5-1}{4}$.\QEDA\\

\z Пусть $y^3=-i$, а $S$ --- множество корней уравнения $x^3=1$. Тогда очевидно, что множество корней нашего уравнения --- это $\{ys\mid s\in S\}$. С другой стороны, можно взять $y=i$. Тогда корни этого уравнения --- $\{i,-\frac{i}{2}\pm \frac{\sqrt 3}{2}\}$.\QEDA\\

\z Пусть $z=i+\cos\varphi+i\sin\varphi$. Тогда \[
	z^2=\cos^2\varphi-(1+\sin\varphi)^2+i(\cos\varphi+\sin\varphi\cos\varphi)=\cos2\varphi-1-2\sin\varphi+\frac{i}{2}(2\cos\varphi+\sin2\varphi).
\QEDA\]

\z Пусть $z=-i+\cos\varphi+i\sin\varphi$. Тогда \[
	z^{-1}=\frac{\cos\varphi+i(1-\sin\varphi)}{\cos^2\varphi+\sin^2\varphi-2\sin\varphi+1}=\frac{1}{2}\cdot\frac{\cos\varphi+i(1-\sin\varphi)}{1-\sin\varphi}=\frac{i}{2}+\frac{\cos\varphi}{2-2\sin\varphi}.
\]

Это прямая $\Im(z)=\frac{1}{2}$.\QEDA\\

\z Пусть $z=\cos\varphi+i\sin\varphi$. Рассмотрим $t=\tg\frac{\varphi}{2}$ (оно существует при $\varphi\neq \pi+2\pi k$, т.е. $z\neq -1$). Тогда $\cos\varphi=\frac{1-t^2}{1+t^2}$ и $\sin\varphi=\frac{2t}{1+t^2}$. Кроме того, $\frac{1+ti}{1-ti}=\frac{(1+ti)^2}{1+t^2}=\frac{1-t^2+2ti}{1+t^2}$.\QEDA\\

\newpage
\hdr{Лист 3, 29 сентября 2020 г.}\vskip20pt

\setcounter{probs}{0}
\z Если $n\leq 1$, то ответ, очевидно, 0. Пусть $n>1$. Выберем старшие $n-2$ коэффициента как угодно (у этого $3^{n-2}$ способов), у нас получится многочлен $P_1(x)\cdot x^3\equiv Q(x)\mod x^2+x+2$, где $\deg Q\leq 1$. Тогда младшие 3 коэффициента для данного $P_1$ можно выбрать единственным образом, а именно, так, чтобы $P(x)\equiv x^2+x+2-Q(x)\mod x^3$. Ответ: $3^{n-2}$.\label{x2x2}\QEDA\\

\z Пусть $x$ --- нетривиальный идемпотент в $\mathbb Z /36\mathbb Z$. Тогда $x(x-1)\vdots 36$. Заметим, что $(x,x-1)=1$, тогда либо $x\vdots 9,x-1\vdots 4$, либо наоборот. В первом случае получаем $x=9$, во втором $x=28$. Ответ: $\{0,1,9,28\}$.\QEDA\\

\z $(x^4-4x^3+1,x^3-3x^2+1)=(x^3-3x^2+1,x^4-x^4-4x^3+3x^3-x+1)=(x^3-3x^2+1,x^3+x-1)=(x^3+x-1,3x^2+x-2)=(3x^2+x-2,x^3+x-1-x^3-\frac{x^2}{3}-\frac{2x}{3})=(3x^2+x-2,\frac{x^2}{3}-\frac{x}{3}-1)=(x^2-x-3,4x+7)=1.\QEDA$\\

\z Пусть $p=qr$, где $\deg p=4$ и $\deg q\geq \deg r$, а $r$ неприводим. Тогда $\deg r\leq 2$. Из многочленов 2 степени неприводим только $x^2+x+1$, значит, если $\deg r=2$, то $p=(x^2+x+1)^2=x^4+x^2+1$. Все остальные многочлены 4 степени (приводимые) делятся или на $x$ (т.е. у них коэффициент 0 при $x^0$), или на $x+1$ (т.е. сумма коэффициентов чётная). Значит, неприводимые многочлены --- это $x^4+x^3+1,x^4+x+1,x^4+x^3+x^2+x+1$.\QEDA\\

\z Пусть $p\in \mathbb Z /2\mathbb Z[x]$ не имеет корней. Тогда у $p+1$ есть и корень 0, и корень 1, значит, $p\vdots x(x+1)$. Рассуждая аналогично \ref{x2x2}, получаем, что таких многочленов $2^{n-3}$ (там ещё есть условие, что старший член равен 1).\QEDA\\

\newpage
\hdr{Лист 4, 30 сентября 2020 г.}\vskip20pt

\setcounter{probs}{0}
\z Заметим, что $f=x^3+x^2+1$ неприводим над $\mathbb Z /2\mathbb Z$. Действительно, пусть $f=pq,\deg p\geq \deg q>0$. Тогда $\deg q=1$, значит, $q=x$ или $q=x+1$. Но $f(0)=f(1)=1$, значит, это не так. Тогда по теореме из лекции $(\mathbb Z /2\mathbb Z)[x] /(x^3+x^2+1)$ это поле.\QEDA\\

\z $(x+1)^2=x,x(x+1)=1$. Ответ: 3.\QEDA\\

\z Будем считать, что $M\neq\emptyset$, иначе это не кольцо. Тогда:

\begin{itemize}
	\item Пусть $|M|>1$. Тогда делители нуля --- все функции, которые дают 0 хотя бы в 1 точке, но не во всех (если $f(x_0)=0$, то можно рассмотреть функцию $g$ такую, что $g(x_0)=1$ и $g(y)=0$ при $y\neq x_0$), а обратимые --- те, для которых $\forall x:f(x)\neq 0$ (можно рассмотреть функцию $g(x)=f(x)^{-1}$). Это не поле.
	\item Если в $M$ один элемент, то это кольцо изоморфно $\mathbb R$. Тогда там нет делителей 0, все элементы обратимые, и это поле.\QEDA
\end{itemize}

\z $(a+b)^p=a^p+b^p+\sum_{i=1}^{p-1} \binom{p}{i} a^ib^{p-i}$, и все слагаемые в этой сумме равны $0$, потому что $\binom{p}{i}=\frac{p!}{i!(p-i)!}$ и числитель делится на $p$, а знаменатель нет.\QEDA\\

\end{document}
