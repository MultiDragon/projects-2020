\documentclass[12pt,a4paper]{article}
\usepackage{tpl}
\begin{document}
\dbegin[10 сентября 2020 г.]{Матфак ВШЭ}{Решения дискретной математики}{Кудрявцев Александр}

\hdr{Лист 1 (16 задач)}

\z Докажем, что $\overline{a}\land \overline{b}=\overline{a\lor b}$. Действительно, левая часть равна 1 тогда и только тогда, когда $a=b=0$, а $a\lor b$ равно 1 во всех остальных случаях.\QEDA\\

\z

\begin{itemize}
	\item $\forall x\in \mathbb Z\exists y\in \mathbb Z:x+y=0$.
	\item $\forall N\in \mathbb N\exists p>N\in \mathbb N:p,p+2$ --- простые. (При этом $p$ простое означает, что $\forall x\in \mathbb N,x<p:p\equiv 0\mod x\iff x=1$)\QEDA\\
\end{itemize}

\n Составим таблицы истинности:

\begin{tabular}{|c|c|c|c|c|}
	\hline
	$x$ & $y$ & $z$ & $x\overline{y}\lor y\overline{z}\lor z\overline{x}$ & $x\overline{z}\lor z\overline{y}\lor y\overline{x}$\\
	\hline
	$0$ & $0$ & $0$ & $0$ & $0$\\
	\hline
	$0$ & $0$ & $1$ & $1$ & $1$\\
	\hline
	$0$ & $1$ & $0$ & $1$ & $1$\\
	\hline
	$0$ & $1$ & $1$ & $1$ & $1$\\
	\hline
	$1$ & $0$ & $0$ & $1$ & $1$\\
	\hline
	$1$ & $0$ & $1$ & $1$ & $1$\\
	\hline
	$1$ & $1$ & $0$ & $1$ & $1$\\
	\hline
	$1$ & $1$ & $1$ & $0$ & $0$\\
	\hline
\end{tabular}

\p
{\Large \begin{align*}
		x\overline{y}\lor y\overline{z}\lor z\overline{x}=\\
		= x\overline{y}z\lor x\overline{y}\overline{z}\lor xy\overline{z}\lor \overline{x}y\overline{z}\lor \overline{x}yz\lor \overline{x}\overline{y}z=\\
		= xy\overline{z}\lor x\overline{y}\overline{z}\lor x\overline{y}z\lor \overline{x}\overline{y}z\lor \overline{x}yz\lor \overline{x}y\overline{z}=\\
		=x\overline{z}\lor z\overline{y}\lor y\overline{x},
\end{align*}}
где второе равенство выполнено по ассоциативности и коммутативности $\land,\lor$, а первое и третье --- т.к. $\forall a,b:a=ab\lor a\overline{b}$.\QEDA\\

\setcounter{probs}{5}

\n Докажем, что у каждого элемента $\chi_M\in {\{0,1\}}^\Omega$ есть единственный прообраз. Действительно, $M=\{x\in\Omega\mid \chi_M(x)=1\}$ --- это прообраз, и если другое множество $N$ является прообразом, то пусть $M\Delta N=K\neq\emptyset$ и $x\in K$, тогда $\chi_M(x)\neq \chi_N(x)$, откуда $\chi_M\neq chi_N$.\QEDA\\

\p

\begin{itemize}
	\item $\chi_{A\cap B}(x)$ равно 1, если (и только если) $x$ лежит хотя бы в одном из $A,B$, а $\chi_A(x)\land\chi_B(x)$ тоже этому равно.
	\item $\chi_{A\cup B}(x)$ равно 1, если (и только если) $x$ лежит хотя бы в одном из $A,B$, а $\chi_A(x)\lor\chi_B(x)$ тоже этому равно.
	\item $\chi_A(x)+\chi_B(x)=\chi_{A\Delta B}(x)$. Действительно, каждая из частей равна 1, если и только если $x$ лежит в одном из $A,B$ и не лежит в другом.
	\item $\chi_A(x)\implies \chi_B(x)$ равно 0, если и только если $x\in A,x\not\in B$. $\chi_{A\setminus B}$ равно 1 при том же условии. Следовательно, $(\chi_A(x)\implies\chi_B(x))=\chi_{\Omega\setminus(A\setminus B)}$.
\end{itemize}

\newpage

\n Пусть $a\in Z^{X\cup Y}$ --- какая-то функция, сопоставляющая каждому элементу $X$ элемент $Z$, и каждому элементу $Y$ --- также элемент $Z$. Рассмотрим функции $b:a\restriction_X$ и $c:a\restriction_Y$. Сопоставим $a$ пару $(b,c)$. Это биекция. Действительно, если есть пара $(b,c)$, то можно рассмотреть функцию $f:X\cup Y\to Z;x\mapsto \{b(x), x\in X;c(x), x\in Y\}$, и ей сопоставлена пара $(b,c)$. (других прообразов у $(b,c)$ очевидно нет, потому что функция будет отличаться по какому-то элементу, и если этот элемент в $X$, то первый компонент образов различен, а если в $Y$, то второй)\QEDA\\

\p Пусть $h\in A^{B\times C}$ --- какая-то функция, каждой паре $(b,c)$ сопоставляющая элемент из $A$. Рассмотрим такую функцию $f:C\to A^B$, каждому элементу $c$ сопоставляющую функцию $g_c:B\to A;b\mapsto h(b,c)$. Аналогично предыдущему пункту это биекция (теперь функции $g:C\to A^B$ мы будем сопоставлять функцию $h:B\times C\to A$ такую, что $h(b,c)=(g(c))(b)$).\QEDA\\

\n Это неправда, потому что $g$ должно быть функцией откуда-то в $Y$ (а не в подмножество), и если $y\in Y\setminus f(X)$, то $g^{-1}(y)=\emptyset$, т.е. $g$ не является сюръекцией.\QEDA\\

\p Пусть $S$ --- множество классов эквивалентности, на которые $f$ делит $f(X)$ (т.е. для каждого $y\in f(X)$ в $S$ лежит множество $A_y:\{x\mid f(x)=y\}$. Рассмотрим $q:X\to S,x\mapsto A_{f(x)}$ и $p:S\to Y,A_y\mapsto y$. Первое сюръекция, потому что я так определил $S$, а второе инъекция, потому что каждый класс эквивалентности представлен один раз.\label{equiv}\QEDA\\

\z Пусть $M$ конечно. Тогда у каждой биекции есть обратная (она и так есть), и всего биекций $|M|!$. Тогда в последовательности $\Id_m,f,f^2,\ldots ,f^{|M|!}$ есть два одинаковых элемента. Пусть это $f^k$ и $f^l$, причём $l>k$. Тогда $\Id_m=f^k\circ {f^{-1}}^k=f_l\circ {f^{-1}}^k=f^{l-k}$. Для бесконечных множеств можно рассмотреть такой контрпример: $M=\mathbb Z,f(x)=x+1$. Тогда $f^k(x)=x+k\neq x$ при $k>0$.\QEDA\\

\n Пусть $f:A\to C,g:A\to B,h:B\to C$.

\begin{itemize}
	\item Если $g,h$ инъективны, то $h(b_1)\neq h(b_2)$ и $g(a_1)\neq g(a_2)$ при $a_1\neq a_2$ и $b_1\neq b_2$. Пусть $f(a_1)=f(a_2)$. Это невозможно, т.к. можно обозначить $g(a_1)=b_1,g(a_2)=b_2$ и применить первые два факта.
	\item Если $g,h$ сюръективны, то $\forall c\exists b:h(b)=c$ и $\exists a:g(a)=b$ (для всех $b$, в т.ч. $b$ из предыдущего факта). Тогда $f(a)=c$ существует для любого $a$.\QEDA\\
\end{itemize}

\n Докажем, что $A_{n+1}\subset A_n$. Индукция по $n$, для $n=0$ верно, шаг верен, потому что ограничение $f$ на $A_n$ является инъекцией и не является биекцией --- если $x\in A_n\subset A_{n-1}$ (оно непусто по предположению индукции), то $f^{-1}(x)\notin A_{n-1}$ (иначе $x=f(f^{-1}(x))\in A_{n-1}$). Для функции $f:\mathbb N\to \mathbb N,x\mapsto x+1$ множество $B$ пусто, для функции $f:\mathbb N\to \mathbb N,0\mapsto 0,x\mapsto x+1$ при $x>0$ множество $B=\{0\}$.\QEDA\\

\z Заметим, что если $A\cap B=\emptyset$, то циклы над множествами $A,B$ коммутируют. Разобьём $N$ на классы эквивалентности таким образом: $x\sim y$, если $\exists n:f^n(x)=y$. Это отношение эквивалентности (см. \ref{equiv}), и над каждым классом $s$ можно сделать цикл $c_s$ (то есть цикл из элементов $a\in s,f(a),\ldots ,f^{n-1}(a),f^n(a)=a$) такой, что $c_1\circ c_2\circ \ldots \circ c_k=f$.\QEDA\\

\z Назовём $k$-последовательностью последовательность из $n+1$ числа, первые $n$ из которых принимают значения 0 или 1, а $n+1$-е число равно $a_K$, где $K$ --- последовательность номеров единиц среди первых $n$ чисел, Пусть $A$ --- множество всех $k$-последовательностей и $B=\{a\in A:a_{n+1}=1\}$. Тогда $f=\sum_{a\in B}t(a)$, где $t(a)=\prod_{b=1;a_b=1}^n x_b\cdot \prod_{b=1;a_b=0}^n \overline{x_b}$ (под сложением здесь имеется в виду логическое <<или>>, а под умножением логическое <<и>>). Докажем это. Вначале заметим, что $t(a)(x_1,\ldots ,x_n)$ равно 1 тогда и только тогда, когда $x_i=a_i\forall i$. Тогда их сумма равна 1 тогда и только тогда, когда это выполняется для какого-то $a\in B$.\QEDA\\

\newpage

\z Принцип: предположить факт, обратный тому, который мы доказываем, и прийти к противоречию.

\p Пусть такое рациональное число существует, тогда можно выбрать его числитель и знаменатель взаимно простыми, т.е. $\exists p,q\in \mathbb N:(p,q)=1,(\frac{p}{q})^2=\frac{p^2}{q^2}=2$. Это значит, что $p^2=2q^2$, и по основной теореме арифметики $p$ чётно. Пусть $p=2r$, тогда $q^2=2r^2$, т.е. $q$ тоже чётно, что противоречит несократимости $\frac{p}{q}$.\QEDA\\

\p Пусть простых чисел конечное число. Пусть $p_0$ --- максимальное из них. Рассмотрим число $p_0!+1$. У этого числа нет делителей меньших или равных $p_0$. Значит, у него есть больший простой делитель.\QEDA\\

\z Например, если $A(x)=x\in \mathbb N,B(x)=x\in \mathbb Z$, то $A\implies B$ верно, а $B\implies A$ неверно. С другой стороны, $\overline{A}\implies \overline{B}\iff B\implies A$ (на этом основан принцип доказательства от противного).\QEDA\\

\end{document}
