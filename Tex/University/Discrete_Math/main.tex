\documentclass[12pt,a4paper]{article}
\usepackage{tpl}
\begin{document}
\dbegin[2 сентября 2020 г.]{Матфак ВШЭ}{Дискретная математика, 1 курс}{Артамкин}

\textit{Формула оценки: $avg_i(\min(10,a_i))$, где $a_i$ --- количество задач, сданных из листка $i$ ($i=1\ldots 5$). Если через первый месяц сдано меньше трёх листков, то вместо этого ставится $0.7\cdot avg_i(\min(10,a_i))$ (при этом $i=1\ldots 4$).}\\

\hdr{Наивная теория множеств}

Пусть мы можем определять множества из элементов с любым общим свойством. Тогда возникает \textit{парадокс Рассела:}

Назовём множество $A$ хорошим, если $A\notin A$. Рассмотрим $\Omega$ --- множество всех хороших множеств. Тогда если $\Omega\in\Omega$, то оно плохое, но при этом хорошее, потому что все элементы $\Omega$ хорошее. А если $\Omega\notin\Omega$, то оно хорошее, но при этом не лежит в множестве всех хороших множеств. Противоречие.\QEDA\\

Чтобы таких противоречий не было, была создана аксиоматика Цермело-Франкеля (ZF).\\

\shdr{Разрешённые способы создания множеств}

\begin{itemize}
	\item Декартово произведение: если $A,B$ --- множества, то $A\times B:=\{(a,b)\mid a\in A,b\in B\}$ --- множество.
	\item Если $A$ --- множество, то $2^A$ --- множество подмножеств $A$ --- тоже множество. (Другие обозначения: $\mathcal B(A),\mathcal P(A)$)
\end{itemize}

\definition{Функция или отображение $f:A\to B$} правило, которое каждому $a\in A$ сопоставляет один элемент $b\in B$, этот $b$ называется образом $a$ и обозначает $f(a)$.

\definition{График функции $f$} $\Gamma_f=\{(x,y)\subset A\times B\mid y=f(x)\}$.

\definition{Инъекция} такая функция $f:A\to B$, что $f(x_1)=f(x_2)\implies x_1=x_2$.

\definition{Сюръекция} такая функция $f:A\to B$, что $\forall b\in B\exists a\in A:f(a)=b$.

\definition{$B^A$} множество всех функций из $A$ в $B$.\\

\textbf{Несколько задач про $B^A$.}

\z Пусть $A\cap B=\emptyset$. Тогда между $X^A\cap B$ и $X^A\times X^B$ существует каноническая биекция.

\z Между ${(X^Y)}^Z$ и $X^{Y^Z}$ существует каноническая биекция.

\z Между $2^A$ и ${\{0,1\}}^A$ существует каноническая биекция.

\proof Построим такое отображение из $2^A$ в ${\{0,1\}}^A$. Пусть $X\in 2^A$. Рассмотрим такую функцию $\chi_X\in{\{0,1\}}^A$ --- индикатор $X$, т.е.
\begin{equation*}
	\chi_X(c)=
	\begin{cases}
		1,c\in X\\
		0,c\notin X.
	\end{cases}
\end{equation*}
Заметим, что:

\begin{itemize}
	\item Если $X\neq Y\in 2^A$, то $X\delta Y\neq\emptyset$. Тогда если $a\in X\delta Y$, то $\chi_X(a)\neq\chi_Y(a)$. Значит, $X\to\chi_X$ --- инъекция.
	\item Если $f\in{\{0,1\}}^A$, то у $f$ есть прообраз --- множество элементов, на которых $f$ равна 1.
\end{itemize}

Следовательно, $f$ биекция.\QEDA\\

\end{document}
