\documentclass[12pt,a4paper]{article}
\usepackage{tpl}
\dbegin[1 сентября 2020 г.]{Матфак ВШЭ}{Лекции по математике, 1 курс}{Разные авторы}

\dsection{Ряды Фарея}{Скрипченко Александра Сергеевна}

\theorem $\pi\notin\mathbb Q$.

\proof Заметим, что \[
	\tg x=\dfrac{x}{1-\dfrac{x^2}{3-\dfrac{x^2}{5-\ldots}}}
.\] Кроме того, $\tg\frac{\pi}{4}=1$. Предположим, что $\pi$ рационально, тогда $t=\frac{\pi}{4}$ рационально, и \[
	1=\dfrac{t}{1-\dfrac{t^2}{3-\dfrac{t^2}{5-\ldots}}}
,\] т.е. бесконечная цепная дробь, противоречие.\QEDA\\

\definition{Последовательность Фарея $\mathcal F_n$} подмножество $[0,1]\cap \mathbb Q$, в которое входят все дроби со знаменателем не больше $n$, упорядоченные по возрастанию.

\textbf{Вопрос:} как понять, куда ставить новые дроби, и как устроены расстояния между соседями?\\

\lemma Пусть $\frac{a}{b},\frac{c}{d},\frac{p}{q}$ --- подряд идущие элементы последовательности, и $\frac{a}{b}<\frac{p}{q}<\frac{c}{d}$. Тогда $bp-aq=cq-pd=1$.\\

% \proof Вначале докажем это для медианты (т.е. если $\frac{p}{q}=\frac{a+c}{b+d}$. Для них очевидно. Иначе рассмотрим $\frac{p'}{q'}=M(\frac{a}{b},\frac{c}{d})$. Пусть, не умаляя общности, $\frac{p}{q}<\frac{p'}{q'}$. Тогда площадь параллелограмма, натянутого на $(p',q')$ и $(a,b)$ равна 1.

\theoremn{Дирихле} $\forall x\in \mathbb R,\forall N\in \mathbb N \exists p,q\leq N: |x-\frac{p}{q}|<\frac{1}{Nq}$.

\proof Пусть $x\in[0,1]$. Рассмотрим последовательность $\mathcal F_n$. Пусть $\frac{c}{d}<x<\frac{a}{b}$ --- соседние члены этой последовательности и $\frac{e}{f}$ --- их медианта. Заметим, что $f=b+d>N$; кроме того, длины отрезков, на которые медианта разобьёт $(\frac{c}{d};\frac{a}{b})$ --- это $\frac{1}{fb}$ и $\frac{1}{fd}$. Наш $x$ попал на один из этих отрезков. Пусть, не умаляя общности, это отрезок $(\frac{a}{b};\frac{e}{f})$; тогда возьмём приближение $\frac{a}{b}$, оно подходит.\QEDA

\theoremn{Рот} $\forall \alpha\notin Q,\varepsilon>0$ существует лишь конечное количество приближений вида $\frac{p}{q}$ таких, что $|\alpha-\frac{p}{q}|<\frac{1}{q^{2+\varepsilon}}$.\\

\definition{Дзета-функция $\zeta(s)$} сумма ряда $a_n=n^{-s}$. Определена везде, кроме $s=1$. Если $\Re(s)>1$, то ряд абсолютно сходится; если $s=-2k,k\in \mathbb N$, то $\zeta(s)=0$ (<<тривиальные нули>>); все остальные нули называются нетривиальными.

\textbf{Гипотеза Римана.} Если $s$ --- нетривиальный ноль, то $\Re(s)=\frac{1}{2}$.\\

\definition{Функция Мёбиуса $\mu(n)$}
\begin{align*}
	\mu_n=\begin{cases}
		0,n\text{ несвободно от квадратов}.\\
		(-1)^k,k\text{ --- количество простых множителей в }n.
	\end{cases}
\end{align*}

\definition{Функция Мёртенса $M(n)$} $M(n)=\sum_{i=1}^n \mu(i)$.

\theorem Гипотеза Римана равносильна следующему утверждению: \[
	\forall\varepsilon>0 \exists C>0: M(n)\leq Cn^{\frac{1}{2}+\varepsilon}
.\]

Пусть $L(n)$ --- количество элементов в $\mathcal F_n$ и $\alpha_v$ --- $v$-й элемент этой последовательности. Обозначим $\delta_v=\alpha_v-\frac{v}{L(n)}$.

\theorem Гипотеза Римана равносильна следующему утверждению: \[
	\forall\varepsilon>0 \exists C>0:\sum\limits_{v=1}^{L(n)}|\delta_v|\leq Cn^{\frac{1}{2}+\varepsilon}
.\]

\end{document}
