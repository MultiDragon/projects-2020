\documentclass[12pt,a4paper]{article}
\usepackage{tpl}
\begin{document}
\dbegin[2 сентября 2020 г.]{Матфак ВШЭ}{Матанализ, 1 курс}{Красносельский}

\textit{Формула оценки: $\dfrac{4P+6S+5K+5E}{20}$, где $P,S,K,E$ --- оценки за листки, семинары, коллоквиум и экзамен соответственно.}\\

\hdr{Вещественные числа}

\definition{Бинарная операция на мн-ве $G$} функция $(a,b)\in G\times G\mapsto a\oplus b\in G$, т.е. каждую упорядоченную пару элементов $G$ переводит в какой-то элемент $G$.

\definition{Коммутативная группа} множество $G$ с операцией $\oplus$ со следующими свойствами:

\begin{itemize}
	\item $\exists e\in G\forall x\in G: e\oplus x=x\oplus e=x$.
	\item $\forall x\in G\exists y\in G: x\oplus y=y\oplus x=e$.
	\item $\forall a,b,c\in G: (a\oplus b)\oplus c=a\oplus (b\oplus c)$ (ассоциативность).
	\item $\forall a,b\in G: a\oplus b=b\oplus a$ (коммутативность).
\end{itemize}

\definition{Поле} множество $(G,\oplus,\odot,0)$ со следующими свойствами:

\begin{itemize}
	\item $(G,\oplus)$ --- аддитивная группа;
	\item $(G\setminus\{0\},\odot)$ --- мультипликативная группа;
	\item $\forall a,b,c\in G: (a\oplus b)\odot c=a\odot c\oplus b\odot c$.
\end{itemize}

\definition{Отношение на множестве $G$} подмножество $G\times G$. Например, отношение <<$a<b$>> в множестве $\{1,2,3\}$ --- это $\{(1,2),(1,3),(2,3)\}$.

\definition{Отношение порядка} отношение $\leqslant$ со следующими свойствами:

\begin{itemize}
	\item $\forall a:a\leqslant a$.
	\item $\forall a,b:(a\leqslant b\cap b\leqslant a)\implies a=b$.
	\item $\forall a,b,c:(a\leqslant b\cap b\leqslant c)\implies a\leqslant c$.
	\item $\forall a,b:(a\leqslant b\cup b\leqslant a)$.
\end{itemize}

\definition{Упорядоченное поле} множество $F$ со следующими свойствами:

\begin{itemize}
	\item $F$ --- поле.
	\item На $F$ есть отношение порядка.
	\item $\forall a,b,c\in F: a\leqslant b\implies a+c\leqslant b+c$.
	\item $\forall a,b\in F:0\leqslant a\cap 0\leqslant b\implies \leqslant a\cdot b$.
\end{itemize}

\textbf{Примеры упорядоченных полей: } $\mathbb Q,\mathbb R,\mathbb Q(\sqrt3)$, алгебраические числа, кроме того, рациональные функции над $\mathbb R$ со следующим отношением порядка: $f_1\leqslant f_2$, если у $f_1-f_2$ отношение старших членов числителя и знаменателя меньше или равен 0.

\textbf{Аксиома непрерывности.} Пусть $F$ --- упорядоченное поле, и $A\neq\emptyset, B\neq\emptyset\subset F$. Кроме того, $\forall a\in A,b\in B:a\leqslant b$. Тогда $\exists c\in F:\forall a\in A,b\in B:a\leqslant c\leqslant b$.

\definition{Множество вещественных чисел} упорядоченное поле с аксиомой непрерывности.

\textit{Пример.} $\mathbb Q\neq \mathbb R$, т.к. у множеств $\{r\in \mathbb Q:r>0,r^2<2\}$ и $\{r\in \mathbb Q:r>0;r^2>2\}$ нет разделителя.\\

\shdr{Примеры моделей действительных чисел}

\begin{itemize}
	\item $0,123\ldots $.
	\item Прямая с $0$ и $1$.
	\item Классы эквивалентности фундаментальных последовательностей из $\mathbb Q$.
	\item Сечения Дедекинда.
\end{itemize}

\end{document}
