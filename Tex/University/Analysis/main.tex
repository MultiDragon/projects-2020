\documentclass[12pt,a4paper]{article}
\usepackage{tpl}
\begin{document}
\dbegin[2 сентября 2020 г.]{Матфак ВШЭ}{Матанализ, 1 курс}{Красносельский}

\textit{Формула оценки: $\dfrac{4P+6S+5K+5E}{20}$, где $P,S,K,E$ --- оценки за листки, семинары, коллоквиум и экзамен соответственно.}\\

\hdr{Вещественные числа}

\definition{Бинарная операция на мн-ве $G$} функция $(a,b)\in G\times G\mapsto a\oplus b\in G$, т.е. каждую упорядоченную пару элементов $G$ переводит в какой-то элемент $G$.

\definition{Коммутативная группа} множество $G$ с операцией $\oplus$ со следующими свойствами:

\begin{itemize}
	\item $\exists e\in G\forall x\in G: e\oplus x=x\oplus e=x$.
	\item $\forall x\in G\exists y\in G: x\oplus y=y\oplus x=e$.
	\item $\forall a,b,c\in G: (a\oplus b)\oplus c=a\oplus (b\oplus c)$ (ассоциативность).
	\item $\forall a,b\in G: a\oplus b=b\oplus a$ (коммутативность).
\end{itemize}

\definition{Поле} множество $(G,\oplus,\odot,0)$ со следующими свойствами:

\begin{itemize}
	\item $(G,\oplus)$ --- аддитивная группа;
	\item $(G\setminus\{0\},\odot)$ --- мультипликативная группа;
	\item $\forall a,b,c\in G: (a\oplus b)\odot c=a\odot c\oplus b\odot c$.
\end{itemize}

\definition{Отношение на множестве $G$} подмножество $G\times G$. Например, отношение <<$a<b$>> в множестве $\{1,2,3\}$ --- это $\{(1,2),(1,3),(2,3)\}$.

\definition{Отношение порядка} отношение $\leqslant$ со следующими свойствами:

\begin{itemize}
	\item $\forall a:a\leqslant a$.
	\item $\forall a,b:(a\leqslant b\cap b\leqslant a)\implies a=b$.
	\item $\forall a,b,c:(a\leqslant b\cap b\leqslant c)\implies a\leqslant c$.
	\item $\forall a,b:(a\leqslant b\cup b\leqslant a)$.
\end{itemize}

\definition{Упорядоченное поле} множество $F$ со следующими свойствами:

\begin{itemize}
	\item $F$ --- поле.
	\item На $F$ есть отношение порядка.
	\item $\forall a,b,c\in F: a\leqslant b\implies a+c\leqslant b+c$.
	\item $\forall a,b\in F:0\leqslant a\cap 0\leqslant b\implies \leqslant a\cdot b$.
\end{itemize}

\textbf{Примеры упорядоченных полей: } $\mathbb Q,\mathbb R,\mathbb Q(\sqrt3)$, алгебраические числа, кроме того, рациональные функции над $\mathbb R$ со следующим отношением порядка: $f_1\leqslant f_2$, если у $f_1-f_2$ отношение старших членов числителя и знаменателя меньше или равен 0.

\textbf{Аксиома непрерывности.} Пусть $F$ --- упорядоченное поле, и $A\neq\emptyset, B\neq\emptyset\subset F$. Кроме того, $\forall a\in A,b\in B:a\leqslant b$. Тогда $\exists c\in F:\forall a\in A,b\in B:a\leqslant c\leqslant b$.

\definition{Множество вещественных чисел} упорядоченное поле с аксиомой непрерывности.

\textit{Пример.} $\mathbb Q\neq \mathbb R$, т.к. у множеств $\{r\in \mathbb Q:r>0,r^2<2\}$ и $\{r\in \mathbb Q:r>0;r^2>2\}$ нет разделителя.\\

\shdr{Примеры моделей действительных чисел}

\begin{itemize}
	\item $0,123\ldots $.
	\item Прямая с $0$ и $1$.
	\item Классы эквивалентности фундаментальных последовательностей из $\mathbb Q$.
	\item Сечения Дедекинда.
\end{itemize}

\newpage

\definition{Индуктивное множество} подмножество $K\subset\mathbb R$ такое, что если $x\in K$, то $x+1\in K$.

\definition{Натуральные числа $\mathbb N$} минимальное индуктивное множество, содержащее единицу.

\definition{Целые числа $\mathbb Z$} множество из всех натуральных чисел, нуля и чисел, противоположных натуральным.

\definition{Рациональные числа $\mathbb Q$} такое множество: $\mathbb Z\cup \{mn^{-1}|m\in \mathbb Z,n\in\mathbb Z\setminus 0\}$

\theorem У любого упорядоченного поля есть подполе, изоморфное $\mathbb Q$.\\

\textbf{Аксиома Архимеда.} $\forall a,h>0\exists n\in\mathbb N:an>h$.

\theoremn{Принцип Архимеда} $\forall h>0\exists a\in \mathbb R:\exists n\in \mathbb Z:(n-1)h\leq a<nh$.

\proof Рассмотрим множество $E=\{n\mid n\in\mathbb Z,ah^{-1}<n\}$. Если $a=0$, то это $\mathbb N$, а в противном случае оно непусто по аксиоме Архимеда. Кроме того, оно ограничено снизу нулём. Тогда у него есть минимальный элемент, он подходит в качестве $n$. \QEDA\\

\textbf{Следствие.} $\forall\varepsilon>0\exists n\in \mathbb N:n^{-1}<\varepsilon$.

\textbf{Следствие 2.} $a<b\in \mathbb R\implies \exists r\in \mathbb Q:a<r<b$.

\proof Возьмём $n$ так, что $n^{-1}<b-a$, и $m$ так, что $\frac{m-1}{n}\leq a<\frac{m}{n}$. Тогда $a<\frac{m}{n}<b$. \QEDA

\textbf{Следствие 3.} $\forall x\in \mathbb R\exists[x]$.\\

\definition{Последовательность} функция натурального аргумента.

\definition{Вложенная последовательность} последовательность $a_i$ множеств такая, что $a_i\subset a_{i-1}$ для всех $i$.

\definition{Стремящаяся к нулю последовательность} последовательность $a_i$ вещественных чисел такая, что $\forall \varepsilon>0\exists n\in \mathbb N:\forall m>n:|a_m|<\varepsilon$.

\theorem Пусть $\Delta_n$ --- вложенная последовательность отрезков в $\mathbb R$, т.е. множеств вида $\{x\in \mathbb R\mid a\leq x\leq b,a,b\in\mathbb R\}$. Тогда $\bigcap_n \Delta_n\neq\emptyset$. Кроме того, если последовательность $|\Delta_n|$ стремится к $0$, то $|\bigcap_n\Delta_n|=1$.

\proof Пусть $\Delta_n=[a_n,b_n]$. Рассмотрим $A=\{a_n\mid n\in \mathbb N\}$ и $B=\{b_n\mid n\in \mathbb N\}$. У нас верно, что $a_n\leq b_m$, т.к. последовательность вложена, тогда по аксиоме непрерывности $\exists\gamma:a_i<\gamma<b_j$. Тогда $\gamma\in[a_n,b_n]\forall n$. Кроме того, если таких $\gamma$ хотя бы два, то длина каждого отрезка хотя бы $|\gamma_2-\gamma_1|$, значит, последовательность длин не стремится к 0. \QEDA\\

\definition{Окрестность $\mathcal O(x)$} любой интервал, содержащий $x$.

\definition{$\varepsilon$-окрестность $\mathcal O_\varepsilon(x)$} интервал $(x-\varepsilon,x+\varepsilon)$.

\shdr{Свойства}

\begin{itemize}
	\item Если $\mathcal O_1(x),\mathcal O_2(x)$ --- окрестности, то $\mathcal O_1\cap \mathcal O_2$ --- тоже.
	\item $\forall x\neq y\exists \mathcal O(x):y\notin O(x)$.
\end{itemize}

\definition{Предельная точка множества $A$} такое число $x$, если в любой окрестности $\mathcal O(x)$ существует $a\neq x\in A$, такое, что $a\in\mathcal O(x)$ (это то же самое, как если бы в этой окрестности было бесконечно много точек из $A$). Множество предельных точек обозначается $A'$.

\theorem Пусть $A$ --- бесконечное ограниченное множество. Тогда $A'\neq\emptyset$.

\proof Пусть границы $A$ --- это точки $a_1,b_1$. Поделим отрезок $[a_1,b_1]$ пополам, в одном из отрезков лежит бесконечное количество точек $A$. Пусть его границы --- это $a_2,b_2$. Его тоже поделим пополам и т.п. У нас получится вложенная последовательность отрезков, на каждом из которых лежит бесконечное количество $A$. Пусть $\gamma$ --- пересечение этих отрезков. Возьмём любую окрестность $\gamma$. Она целиком содержит какой-то из отрезков $[a_n,b_n]$, в котором бесконечное количество элементов $A$. Значит, $\gamma$ --- предельная точка.\QEDA

\newpage

\definition{Открытое множество} такое множество $G$, что $\forall x\in G\exists \mathcal O(x)\subset G$.

\definition{Замкнутое множество} такое множество $G$, что $G'\subset G$.

\definition{Внутренняя точка множества $A$} такая точка $x$, что $\exists\mathcal O(x)\subset A$.

\shdr{Примеры}

\begin{itemize}
	\item $\emptyset,\mathbb R$ замкнутые и открытые одновременно. Других одновременно замкнутых и открытых множеств нет (это эквивалентно аксиоме непрерывности).
	\item $\mathbb N$ замкнутое --- у него нет предельных точек.
	\item $(a,b)$ открытое.
	\item $[a,b]$ замкнутое.
	\item Канторово множество замкнутое.
	\item $[a,b)$ ни открытое, ни замкнутое.
\end{itemize}

\theorem Пусть $A$ открытое, а $B$ замкнутое. Тогда $A\setminus B$ открытое, а $B\setminus A$ замкнутое.

\end{document}
