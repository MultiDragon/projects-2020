\documentclass[12pt,a4paper]{article}
\usepackage{tpl}
\begin{document}
\dbegin[2 сентября 2020 г.]{Матфак ВШЭ}{Матанализ, 1 курс}{Красносельский}

\textit{Формула оценки: $\dfrac{4P+6S+5K+5E}{20}$, где $P,S,K,E$ --- оценки за листки, семинары, коллоквиум и экзамен соответственно.}\\

\hdr{Вещественные числа}

\definition{Бинарная операция} функция $(a,b)\in G\times G\mapsto a\oplus b\in G$, т.е. каждую упорядоченную пару элементов $G$ переводит в какой-то элемент $G$.

\definition{Коммутативная группа} множество $G$ с операцией $\oplus$ со следующими свойствами:

\begin{itemize}
	\item $\exists e\in G\forall x\in G: e\oplus x=x\oplus e=x$.
	\item $\forall x\in G\exists y\in G: x\oplus y=y\oplus x=e$.
	\item $\forall a,b,c\in G: (a\oplus b)\oplus c=a\oplus (b\oplus c)$ (ассоциативность).
	\item $\forall a,b\in G: a\oplus b=b\oplus a$ (коммутативность).
\end{itemize}

\definition{Поле} множество $(G,\oplus,\odot,0)$ со следующими свойствами:

\begin{itemize}
	\item $(G,\oplus)$ --- аддитивная группа;
	\item $(G\setminus\{0\},\odot)$ --- мультипликативная группа;
	\item $\forall a,b,c\in G: (a\oplus b)\odot c=a\odot c\oplus b\odot c$.
\end{itemize}

\definition{Отношение} подмножество $G\times G$. Например, отношение <<$a<b$>> в множестве $\{1,2,3\}$ --- это $\{(1,2),(1,3),(2,3)\}$.

\definition{Отношение порядка} отношение $\leqslant$ со следующими свойствами:

\begin{itemize}
	\item $\forall a:a\leqslant a$.
	\item $\forall a,b:(a\leqslant b\cap b\leqslant a)\implies a=b$.
	\item $\forall a,b,c:(a\leqslant b\cap b\leqslant c)\implies a\leqslant c$.
	\item $\forall a,b:(a\leqslant b\cup b\leqslant a)$.
\end{itemize}

\definition{Упорядоченное поле} множество $F$ со следующими свойствами:

\begin{itemize}
	\item $F$ --- поле.
	\item На $F$ есть отношение порядка.
	\item $\forall a,b,c\in F: a\leqslant b\implies a+c\leqslant b+c$.
	\item $\forall a,b\in F:0\leqslant a\cap 0\leqslant b\implies \leqslant a\cdot b$.
\end{itemize}

\textbf{Примеры упорядоченных полей: } $\mathbb Q,\mathbb R,\mathbb Q(\sqrt3)$, алгебраические числа, кроме того, рациональные функции над $\mathbb R$ со следующим отношением порядка: $f_1\leqslant f_2$, если у $f_1-f_2$ отношение старших членов числителя и знаменателя меньше или равен 0.

\textbf{Аксиома непрерывности.} Пусть $F$ --- упорядоченное поле, и $A\neq\emptyset, B\neq\emptyset\subset F$. Кроме того, $\forall a\in A,b\in B:a\leqslant b$. Тогда $\exists c\in F:\forall a\in A,b\in B:a\leqslant c\leqslant b$.

\definition{Множество вещественных чисел} упорядоченное поле с аксиомой непрерывности.

\textit{Пример.} $\mathbb Q\neq \mathbb R$, т.к. у множеств $\{r\in \mathbb Q:r>0,r^2<2\}$ и $\{r\in \mathbb Q:r>0;r^2>2\}$ нет разделителя.\\

\shdr{Примеры моделей действительных чисел}

\begin{itemize}
	\item $0,123\ldots $.
	\item Прямая с $0$ и $1$.
	\item Классы эквивалентности фундаментальных последовательностей из $\mathbb Q$.
	\item Сечения Дедекинда.
\end{itemize}

\newpage

\definition{Индуктивное множество} подмножество $K\subset\mathbb R$ такое, что если $x\in K$, то $x+1\in K$.

\definition{Натуральные числа} минимальное индуктивное множество, содержащее единицу.

\definition{Целые числа} множество из всех натуральных чисел, нуля и чисел, противоположных натуральным.

\definition{Рациональные числа} такое множество: $\mathbb Z\cup \{mn^{-1}|m\in \mathbb Z,n\in\mathbb Z\setminus 0\}$

\theorem У любого упорядоченного поля есть подполе, изоморфное $\mathbb Q$.\\

\textbf{Аксиома Архимеда.} $\forall a,h>0\exists n\in\mathbb N:an>h$.

\theoremn{Принцип Архимеда} $\forall h>0\exists a\in \mathbb R:\exists n\in \mathbb Z:(n-1)h\leq a<nh$.

\proof Рассмотрим множество $E=\{n\mid n\in\mathbb Z,ah^{-1}<n\}$. Если $a=0$, то это $\mathbb N$, а в противном случае оно непусто по аксиоме Архимеда. Кроме того, оно ограничено снизу нулём. Тогда у него есть минимальный элемент, он подходит в качестве $n$. \QEDA\\

\textbf{Следствие.} $\forall\varepsilon>0\exists n\in \mathbb N:n^{-1}<\varepsilon$.

\textbf{Следствие 2.} $a<b\in \mathbb R\implies \exists r\in \mathbb Q:a<r<b$.

\proof Возьмём $n$ так, что $n^{-1}<b-a$, и $m$ так, что $\frac{m-1}{n}\leq a<\frac{m}{n}$. Тогда $a<\frac{m}{n}<b$. \QEDA

\textbf{Следствие 3.} $\forall x\in \mathbb R\exists[x]$.\\

\definition{Последовательность} функция натурального аргумента.

\definition{Вложенная последовательность} последовательность $a_i$ множеств такая, что $a_i\subset a_{i-1}$ для всех $i$.

\definition{Стремящаяся к нулю последовательность} последовательность $a_i$ вещественных чисел такая, что $\forall \varepsilon>0\exists n\in \mathbb N:\forall m>n:|a_m|<\varepsilon$.

\theorem Пусть $\Delta_n$ --- вложенная последовательность отрезков в $\mathbb R$, т.е. множеств вида $\{x\in \mathbb R\mid a\leq x\leq b,a,b\in\mathbb R\}$. Тогда $\bigcap_n \Delta_n\neq\emptyset$. Кроме того, если последовательность $|\Delta_n|$ стремится к $0$, то $|\bigcap_n\Delta_n|=1$.

\proof Пусть $\Delta_n=[a_n,b_n]$. Рассмотрим $A=\{a_n\mid n\in \mathbb N\}$ и $B=\{b_n\mid n\in \mathbb N\}$. У нас верно, что $a_n\leq b_m$, т.к. последовательность вложена, тогда по аксиоме непрерывности $\exists\gamma:a_i<\gamma<b_j$. Тогда $\gamma\in[a_n,b_n]\forall n$. Кроме того, если таких $\gamma$ хотя бы два, то длина каждого отрезка хотя бы $|\gamma_2-\gamma_1|$, значит, последовательность длин не стремится к 0. \QEDA\\

\definition{Окрестность} любой интервал, содержащий $x$.

\definition{$\varepsilon$-окрестность} интервал $(x-\varepsilon,x+\varepsilon)$.

\shdr{Свойства}

\begin{itemize}
	\item Если $\mathcal O_1(x),\mathcal O_2(x)$ --- окрестности, то $\mathcal O_1\cap \mathcal O_2$ --- тоже.
	\item $\forall x\neq y\exists \mathcal O(x):y\notin O(x)$.
\end{itemize}

\definition{Предельная точка множества} такое число $x$, если в любой окрестности $\mathcal O(x)$ существует $a\neq x\in A$, такое, что $a\in\mathcal O(x)$ (это то же самое, как если бы в этой окрестности было бесконечно много точек из $A$). Множество предельных точек обозначается $A'$.

\theorem Пусть $A$ --- бесконечное ограниченное множество. Тогда $A'\neq\emptyset$.

\proof Пусть границы $A$ --- это точки $a_1,b_1$. Поделим отрезок $[a_1,b_1]$ пополам, в одном из отрезков лежит бесконечное количество точек $A$. Пусть его границы --- это $a_2,b_2$. Его тоже поделим пополам и т.п. У нас получится вложенная последовательность отрезков, на каждом из которых лежит бесконечное количество $A$. Пусть $\gamma$ --- пересечение этих отрезков. Возьмём любую окрестность $\gamma$. Она целиком содержит какой-то из отрезков $[a_n,b_n]$, в котором бесконечное количество элементов $A$. Значит, $\gamma$ --- предельная точка.\QEDA

\newpage

\definition{Открытое множество} такое множество $G$, что $\forall x\in G\exists \mathcal O(x)\subset G$.

\definition{Замкнутое множество} такое множество $G$, что $G'\subset G$.

\definition{Внутренняя точка множества} такая точка $x$, что $\exists\mathcal O(x)\subset A$.

\shdr{Примеры}

\begin{itemize}
	\item $\emptyset,\mathbb R$ замкнутые и открытые одновременно. Других одновременно замкнутых и открытых множеств нет (это эквивалентно аксиоме непрерывности).
	\item $\mathbb N$ замкнутое --- у него нет предельных точек.
	\item $(a,b)$ открытое.
	\item $[a,b]$ замкнутое.
	\item Канторово множество замкнутое.
	\item $[a,b)$ ни открытое, ни замкнутое.
\end{itemize}

\theorem Пусть $A$ открытое, а $B$ замкнутое. Тогда $A\setminus B$ открытое, а $B\setminus A$ замкнутое. В частности, $\mathbb R\setminus B$ открытое, а $\mathbb R\setminus A$ замкнутое.\label{openclosed}

\proof Первая часть: пусть $x\in A\setminus B$. Тогда существует окрестность $O(x)\subset A$. Кроме того, $x\not\in B$. Тогда $x\not\in B'$, значит, существует окрестность $O*(x)$ такая, что $O*(x)\cap B=\emptyset$. Тогда $O(x)\cap O*(x)\subset A\setminus B$.

Вторая часть: пусть $x\in (B\setminus A)'$. Это значит, что в любой $O(x)$ бесконечно много точек из $B\setminus A$. Так как все эти точки в $B$, и $B$ замкнутое, то $x\in B$. Предположим, что $x\in A$. Тогда есть $O*(x)\subset A$, значит, $O*(x)\cap B\setminus A=\emptyset$, противоречие. Значит, $x\in B\setminus A$.\QEDA\\

\definition{Замыкание множества} множество $\overline{A}=A\cup A'$.\\

\lemma Пусть $A\subset \mathbb R$. Тогда $A'$ замкнуто.

\proof Пусть $x\in A''$. Это значит, что в любой $O(x)$ есть $y\in A'$. Возьмём окрестность $O(y)$, такую, что $O(y)\subset O(x),x\not\in O(y)$. В этой окрестности есть $z\in A$. Но $z\in O(x)$, значит, $x$ --- предельная точка для $A$, т.е. $x\in A'$.\QEDA\\

\lemma Пусть $A\subset \mathbb R$. Тогда $\overline{A}$ замкнуто.

\proof Пусть $x\in (\overline{A})'$. Вначале докажем, что $x\in(A'\cup A'')$. Действительно, $x\in (A\cup B)'$ значит, что есть последовательность в $A\cup B$, которая сходится к $x$, и из неё можно выбрать бесконечную подпоследовательность, лежащую либо в $A$, либо в $B$.

Итак, $x\in (A'\cup A'')$. Но $A'$ замкнуто, значит, $A''\subset A'$. Тогда $x\in A'\subset \overline{A}$, т.е. $\overline{A}$ замкнуто.\QEDA\\

\lemma Пусть $S\subset 2^{\mathbb R}$ и все множества в $S$ открытые. Тогда $A=\bigcup_{s\in S} s$ открытое.\label{opencup}

\proof Пусть $x\in A$. Тогда $x\in s$ для какого-то $s\in S$. Тогда $O(x)\in s\subset A$. \QEDA\\

\lemma Пусть $S\subset 2^{\mathbb R}$ и все множества в $S$ замкнутые. Тогда $A=\bigcap_{s\in S} s$ замкнутое.\label{closedcap}

\proof Следует из~\ref{openclosed} и~\ref{opencup}.\QEDA\\

\lemma Пусть $S\subset 2^{\mathbb R}$ конечно и все множества открытые. Тогда $A=\bigcap_{s\subset S} s$ открытое.\label{opencap}

\proof Докажем для двух множеств (для большего числа по индукции). Пусть $A=S_1\cap S_2$. Возьмём любое число $k\in S_1\cap S_2$. Мы знаем, что $\exists O_1(k)\subset S_1,O_2(k)\subset S_2$. Значит, $O_1(k)\cap O_2(k)\subset S_1\cap S_2$. Значит, $A$ открытое.\QEDA\\

\lemma Пусть $S\subset 2^{\mathbb R}$ конечно и все множества замкнутые. Тогда $A=\bigcup_{s\subset S} s$ замкнутое.

\proof Следует из~\ref{openclosed} и~\ref{opencap}. \QEDA\\

\newpage

\theorem Пусть $A$ открытое. Тогда $A=\bigsqcup_{i=1}^\infty A_n$, где $A_n=(a_n;b_n)$ (при этом $a_n$ может быть равно $-\infty$, а $b_n$ --- $+\infty$).

\proof Пусть $x\in A$. Тогда $O(x)\in A$. Рассмотрим все такие окрестности. Пусть $L_x$ --- множество левых их концов, а $R_x$ --- их правых концов. Мы знаем, что существуют $l_x=\inf L_x,r_x\sup R_x$ (причём они могут быть равны $\pm\infty$). Возьмём все эти интервалы $(l_x,r_x)$ и в каждом из них выберем рациональную точку. Они все различные (если интервалы пересекаются, то они равны), значит, интервалов не более чем счётно.\QEDA\\

\definition{Покрытие множества} набор множеств $B$, т.ч. $A\subset\bigcup_{b\in B} b$.

\definition{Открытое покрытие} покрытие $B$, т.ч. все его элементы открытые.

\definition{Компакт} множество $A$, такое, что из его любого открытого покрытия можно выбрать конечное подпокрытие.

\lemman{Гейне, Борель} Отрезок $[a,b]$ является компактом.

\proof Пусть $A=[a,b]$, и $B$ --- его открытое покрытие. Пусть такого конечного подпокрытия не существует. Разрежем отрезок пополам, у нас получатся отрезки $C_1,D_1$. Тогда хотя бы для одного из них нет конечного подпокрытия (если нет у обоих, выберем другой). Разрежем его пополам на отрезки $C_2,D_2$, хотя бы у одного из них нет конечного подпокрытия и т.п. Рассмотрим систему отрезков $E_i=C_i\cup D_i$. У них есть общая точка $\gamma$. Так как $B$ --- покрытие, то наша общая точка лежит в каком-то множестве $b$. Значит, $O(\gamma)\subset b$ (т.к. покрытие было открытым). Возьмём отрезок $E_k\subset O(\gamma)$ (он есть, т.к. длины этих отрезков стремятся к 0). Мы знаем, что для него нет конечного подпокрытия, но оно есть --- это множество $b$. Противоречие.\QEDA\\

\definition{Всюду плотное множество} такое множество $A\subset[0,1]$, что $\overline{A}=[0,1]$ (первое условие для определённости, в общем случае оно не нужно).

\definition{Нигде не плотное множество} такое множество $A$, что для любого $(a,b)$ существует $(c,d)\subset (a,b)$ такой, что $(c,d)\cap A=\emptyset$.

\theoremn{Бэр} Отрезок $[0,1]$ нельзя представить в виде объединения счётного количества нигде не плотных множеств.

\proof Пусть $[0,1]=\bigcup_{i=1}^\infty B_i$ и все $B_i$ нигде не плотные. Обозначим $C_1=[0,1]$. Возьмём любой интервал на отрезке $C_k$, тогда в нём существует интервал, не пересекающийся с $B_k$, выберем в нём подмножество --- отрезок и назовём его $C_{k+1}$. У нас получится система вложенных отрезков, у которой есть общая точка $\gamma$, которая не лежит ни в одном из наших множеств, т.к. $C_t$ не пересекается с $B_q$ при $t>q$. \QEDA\\

\definition{Канторово множество} результат следующей процедуры. Обозначим $S_0=\emptyset,K_i=[0,1]\setminus S_i$. На каждом шаге $K_i$ будет дизъюнктным объединением конечного количества отрезков вида $[l_j,r_j]$. Тогда $S_{i+1}=S_i\cup\bigsqcup (\frac{2l_j+r_j}{3},\frac{l_j+2r_j}{3})$ будет дизъюнктным объединением конечного количества интервалов. Канторовым множеством называется $\bigcap_{i=1}^\infty K_i$.

\theorem Канторово множество замкнутое, нигде не плотное, и равномощное $[0,1]$.

\proof Первая часть: $K=[0,1]\setminus\bigcap_{i=1}^\infty S_i$, и каждое $S_i$ --- объединение конечного кол-ва интервалов, значит, их объединение --- объединение счётного кол-ва интервалов, т.е. открытое. Тогда $K$ --- дополнение открытого множества, значит, оно замкнутое.

Вторая часть: рассмотрим любой интервал I. Если он пересекается с $(\frac{1}{3},\frac{2}{3})$, то мы победили, иначе он лежит целиком внутри либо $[0,\frac{1}{3}]$, либо $[\frac{2}{3},1]$. Рассмотрим тот, в котором он лежит, и продолжим так делать до тех пор, пока мы не победим (мы победим, т.к. для какого-то $r$ верно $|I|>\frac{1}{3^r}$ и мы сделаем не более $r+1$ шага).

Третья часть: заметим, что канторово множество --- это множество всех таких чисел на отрезке $[0,1]$, что их троичная запись состоит только из $0$ и $2$. Заменим все $2$ на $1$ и рассмотрим число с такой двоичной записью. Это биекция между отрезком и канторовым множеством.\QEDA\\

\end{document}
