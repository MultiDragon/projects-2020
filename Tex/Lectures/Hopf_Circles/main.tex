\documentclass[12pt,a4paper]{article}
\usepackage{tpl}
\dbegin[5 мая 2020 г.]{Лекции 179}{Преобразования Хопфа}{Тиморин Владлен Анатольевич}

\shdr{Параметризация окружности}

\begin{enumerate}
	\item $x^2+y^2=1$;
	\item $x=\sin\varphi,y=\cos\varphi$;
	\item $x=\frac{1-t^2}{1+t^2},y=\frac{2t}{1+t^2}$. Это работает, потому что:
	\begin{itemize}
		\item При таких $x,y$ выполняется $x^2+y^2=1$.
		\item $t=\tg\frac{\varphi}{2}$.
		\item Это проекция прямой $x=0$ из точки $(-1,0)$ на окружность. К сожалению, отсюда следует, что эта точка ничему не соответствует.
	\end{itemize}
\item $x=\frac{s^2-t^2}{s^2+t^2},y=\frac{2st}{s^2+t^2}$.
\end{enumerate}

\lemman{Тождество Диофанта-Брахмагупты-Фибоначчи} Если $x,y$ оба представляются как суммы двух квадратов, то $xy$ тоже.

\proof $(a^2+b^2)(c^2+d^2)=(ac\pm bd)^2+(ad\mp bc)^2$. Кстати, это связано с тем, что $|z_1|\cdot |z_2|=|z_1\cdot z_2|$.\QEDA\\

Кроме того, ещё есть такая формула: \[
	(|z|^2-|w|^2)+(|2z\overline{w}|)^2=(|z|^2+|w|^2)
.\] Когда мы поделим левую часть на правую, у нас получится

\begin{align*}
	\xi+i\eta=\frac{2z\overline{w}}{|z|^2+|w|^2},\\
	\zeta=\frac{|z|^2-|w|^2}{|z|^2+|w|^2},\\
	\xi^2+\eta^2+\zeta^2=1,
\end{align*}
то есть мы записали точки двумерной сферы комплексными числами $z,w$, то есть четырьмя параметрами. Кроме того, мы знаем, что можно $z,w$ умножить на произвольное ненулевое комплексное $k$. Поэтому $w$ в формуле сопряжено --- иначе умножение не работает. (так оно работает, потому что числитель умножится на $k\overline{k}=|k|^2\in \mathbb R$)

Давайте считать, что $\Im w=1$. То есть $z=s+it,w=u+i$.

\definition{Отображение Хопфа} отображение $(s,t,u)\mapsto (\xi,\eta,\zeta)$ из $\mathbb R^3$ в $\mathbb S^2$.

На самом деле, это отображение --- \textit{стереографическая проекция} (см.рис).

\imagefigure[0.45]{stereograph}{Стереографическая проекция}

\newpage

Ещё можно считать, что $|z|^2+|w|^2=1$. То есть $z=x_1^2+ix_2,w=x_3+ix_4,\sum x_i^2=1$.

\definition{Отображение Хопфа} отображение $\mathbb S^3\to\mathbb S^2,(x_1,x_2,x_3,x_4)\mapsto (\xi,\eta,\zeta)$.

Прообраз каждой точки $\mathbb S^2$ в $\mathbb S^3$ --- это окружность.  $\mathbb S^3$ бьётся на торы с общими вертикальными осями (эти торы --- прообразы параллелей на сфере) и слои отображения (т.е. эти окружности)  \textit{зацеплены между собой} (см.рис).

\imagefigure{hopf2-circles}{Окружности на торе}

Можно находить на торе окружности, проводя плоскости через ось или перпендикулярные оси. Но ещё можно проводить касательные плоскости (см.рис), получаются окружности Вилларсо.

\imagefigure{villarso}{Окружности Вилларсо}

\hrule\vskip20pt

\theorem Первое отображение Хопфа переводит прямые в окружности.\label{lines}

\lemma Пусть кривая в $Oxyz$ задана как $x=\frac{q_1(t)}{q_0(t)},y=\frac{q_2(t)}{q_0(t)},z=\frac{q_3(t)}{q_0(t)}$, где $q_i$ --- квадратные формулы. Тогда эта кривая лежит в плоскости.\label{quadr}

\proof Рассмотрим уравнение $\sum\limits_{i=0}^3 \lambda_iq_i(t)=0$. Это уравнение вида $at^2+bt+c=0$, где в $a,b,c$ есть неизвестные. Тогда надо рассмотреть систему $a=0,b=0,c=0$. У неё есть решение, потому что $a=\sum k_i\lambda_i$ и т.п. (она однородная), тогда мы получим уравнение $\lambda_0+\lambda_1x+\lambda_2y+\lambda_3z=0$, то есть уравнение плоскости.\QEDA\\

\prooft{lines} Образы прямых --- это формулы вида из \ref{quadr}. Значит, они лежат в плоскости, а так как отображение Хопфа переводит в сферу, то прямые переходят в окружности.\QEDA\\

\end{document}
