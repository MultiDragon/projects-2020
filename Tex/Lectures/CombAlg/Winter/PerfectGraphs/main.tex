\documentclass[12pt,a4paper]{article}
\usepackage{tpl}
\newcommand{\cmp}{\leqslant}
\dbegin[22 февраля 2020 г.]{Комбинаторика и алгоритмы, зима 2020}{Совершенные графы}{Григорьев Михаил Александрович}

Пусть есть граф $G$ и мы хотим правильно раскрасить его вершины (т.е. так, что концы каждого ребра разных цветов). Назовём $\chi(G)$ --- минимальное число цветов для такой раскраски, $\omega(G)$ --- кликовое число (т.е. размер максимальной клики), $\alpha(G)$ --- число независимости (размер максимальной антиклики). Очевидно, что $\chi(G)\geq \omega(G)$. Кроме того, для многих графов тут равенство, но не для всех (например, для нечётного цикла  $\chi(G)=2,\omega(G)=3$).\\

\textbf{Жадный алгоритм раскраски.} Красим вершины по порядку, на каждом шаге красим вершину в минимальный возможный цвет. Этот алгоритм не оптимальный и зависит от порядка. Однако если упорядочить вершины по номеру, получится оптимальная раскраска. Кроме того, вне зависимости от порядка получится раскраска в $\Delta(G)+1$ цвет, где  $\Delta(G)$ --- максимальная степень вершины.

\lemma Пусть $G$ связен, $\Delta(G)=k$ и есть вершина $u$ степени не больше $k-1$. Тогда $\chi(G)\leq k$.

\proof Удалим $u$. Тогда $\Delta(G)\leq k$ и есть новые вершины степени не больше $k-1$ (соседи $u$). Если граф после этого разбился на компоненты, то такие вершины есть в каждой компоненте, значит, каждая компонента красится в  $k$ цветов. Потом покрасим $u$ в цвет, которого нет у её соседей.\QEDA\\

\lemma Пусть $u$ --- мост между несколькими компонентами, каждая из которых красится в $n$ цветов. Тогда $G$ можно покрасить в $n+1$ цвет.

\proof Покрасим каждую из компонент в одни и те же $n$ цветов, а $u$ в цвет $n+1$.\QEDA\\

\definition{Критическое число $G$} такое $k$, что в любом $H\subset G$ есть $v:\deg v\leq k$. \\

\lemma $k$-критический граф красится в $k+1$ цвет.

\proof Пусть $G$ $k$-критический. Тогда удалим любую вершину $u$ и получаем другой $k$-критический граф. Он красится в  $k+1$ цвет, у $u$ есть максимум $k$ запретов.\QEDA\\

\z Граф, как конечное число ладей бьют друг друга, $2$-критический.

\s Возьмём у подграфа самую верхнюю линию и в ней самую левую ладью.\QEDA\\

\z Пусть на плоскости есть несколько кругов, и проведено ребро между ними, если два круга касаются. Этот граф $6$-критический.

\s Возьмём самый маленький круг любого подграфа, для него каждый угол минимум $60^\circ$, значит, его касается максимум 6 кругов.\QEDA\\

\inkscapefigure[4]{circles-subgraph}{Круги на плоскости и углы между их центрами}
\vskip10pt

\lemma Каждый планарный граф $5$-критический.\\

\definition{Индуцированный подграф} такой подграф $H$ графа $G$, что $\forall v_1,v_2\in V(H):(v_1,v_2)\in E(G)\iff (v_1,v_2)\in E(H)$.

\definition{Совершенный граф $G$} такой граф $G$, что для каждого его индуцированного подграфа $H$ выполняется $\chi(H)=\omega(H)$. Например, полные и двудольные графы совершенные.

\theorem Интервальный граф (т.е. граф пересечения отрезков на прямой) критический.\label{intervals}

\prooftms{intervals}
\prooftm{intervals} Индуцированные подграфы интервального графа тоже интервальные. Докажем, что для интервального графа $\chi(G)=\omega(G)$. Пусть $\omega(G)=n$. Посмотрим на отрезок, у которого правый конец самый левый из возможных. Тогда любой отрезок, который пересекает выбранный, пересекает его в том числе по его последней точке (иначе мы выбрали не самый левый). В этой точке не больше $n-1$ отрезков, значит, у нашего отрезка степень не больше $n-1$. Мы доказывали для подграфов, значит, он $(n-1)$-критический.\QEDA\\

\prooftm{intervals} Жадный алгоритм красит граф, у которого $\omega(G)=n$, в $n-1$ цвет.\QEDA\\

\z Дополнение интервального графа совершенное.\\

\theorem Дополнение совершенного графа совершенное.\label{maintheorem}

\theoremn{Гипотеза Бержа} $G$ совершенный тогда и только тогда, когда в $G$ и в $\overline{G}$ нет нечётных циклов длины больше 3 без других рёбер.

\textit{<<Математика --- это когда вы смотрите на разные вещи и думаете, что это круто. Пока вы так думаете, это и правда круто.>> --- Григорьев}\\

\theoremn{Кёниг} Пусть есть набор клеток и мы хотим расставить туда максимальное количество ладей. Это количество равно минимальному количеству полосок, которые накрывают этот набор клеток.\label{konig}

\textbf{Другая формулировка.} Максимальное паросочетание двудольного графа (т.е. размер максимального набора попарно несмежных рёбер) равно его минимальному вершинному покрытию (т.е. размеру минимального множества вершин, что вне них нет рёбер).

\textbf{Третья формулировка.} Заметим, что $\alpha(G)+\beta(G)=n$. Тогда теорема говорит, что в двудольном графе $\alpha(G)+\text{паросочетание}=n$.\\

\definition{Рёберный граф $H$} граф $G$, у которого вершины --- рёбра $H$, а ребро проведено между двумя вершинами, если в $H$ они были соединены ребром.

\textbf{Четвёртая формулировка \ref{konig}.} Дополнение рёберного графа двуд. графа совершенно.\\

\lemman{Холл} Пусть $G$ --- двудольный граф, $A$ и $B$ --- его доли. Тогда инъекция $f:A\to B$, такая, что $x\sim f(x)$, существует тогда и только тогда, когда $\forall X\subset A:|N(X)|\geq|X|$, где $N(X)=\bigcup_{x\in X}f(x)$.\label{hall}

\proof В сторону <<только тогда>> это очевидно. Пусть $G=(A,B)$ --- минимальный граф, для которого утверждение неверно ($A$ минимальная из возможных, $B$ минимальная из возможных с таким размером $A$). Тогда рассмотрим два случая:
\begin{itemize}
	\item Пусть $\forall X\neq A:|N(X)|>|X|$. Тогда можно удалить любую вершину в $A$, любую смежную ей вершину и получить контрпример меньший, чем $G$.
	\item Иначе $\exists X\neq A:|N(X)|=|X|$. Заметим, что по лемме Холла для $G'=(X,N(X))$ их можно разбить на пары. Допустим, что для $H=(A\setminus X,B\setminus N(X))$ не выполнено условие леммы Холла. Пусть оно не выполнено для $Y\subset A\setminus X$. Тогда очевидно, что оно не выполнено и для $Y\cup X$.\QEDA\\
\end{itemize}

\textbf{Другая формулировка \ref{hall}.} Пусть есть $n$ множеств. Тогда можно выбрать из них по различному элементу тогда и только тогда, когда в объединении любых $k$ множеств есть хотя бы $k$ элементов.

\textbf{Третья формулировка \ref{hall}.} То же самое, но на языке табличек и ладей.\\

\prooft{konig} Пусть $G=(A\cup C,B\cup D)$ двудольный и $A\cup B$ --- вершинное покрытие. Тогда рёбра лежат внутри $AB,BC,AD$. Заметим, что в $G_1=(A,D)$ и $G_2=(B,C)$ выполняется условие леммы Холла, потому что иначе можно уменьшить это вершинное покрытие. Значит, в $(A,D)$ и $(B,C)$ есть максимальное паросочетание размера $\pi(G_1)+\pi(G_2)=|A|+|B|=\beta(G)$. С другой стороны, очевидно, что  $\beta(G)\geq \pi(G)$, значит, они равны.\QEDA\\

\hdr{Частично упорядоченные множества}

\definition{Частично упорядоченное множество} пара $(M,\cmp)$ (где $\cmp$ --- операция $M\times M\to \{0,1\}$) такая, что:
\begin{enumerate}
	\item $x\cmp x$ (рефлексивность);
	\item $x\cmp y,y\cmp z\implies x\cmp z$ (транзитивность);
	\item $x\cmp y,y\cmp x\implies x=y$ (антисимметричность).
\end{enumerate}

\shdr{Примеры}

\begin{itemize}
	\item $(\mathbb R,\leq)$.
	\item Множества по операции вложения.
	\item $(\mathbb N,a\cmp b\iff a\mid b)$.
\end{itemize}

\definition{Наибольший элемент} такое $x\in M$, что $\forall y\in M:y\cmp x$.

\definition{Максимальный элемент} такое $x\in M$, что $\forall y\in M:x\cmp y\implies x=y$.

\definition{Цепь} набор попарно сравнимых элементов.

\definition{Антицепь} набор попарно несравнимых элементов.

\theoremn{Мирский} Пусть максимальный размер цепи в $M$ равен $k$. Тогда $M$ можно разбить на $k$ антицепей.\label{mirsk}

\proof Доказываем по индукции. База для $k=1$ очевидна. Для шага уберём все максимальные элементы, тогда все цепи уменьшились на 1. \QEDA

\textbf{Следствие.} Рассмотрим граф, у которого вершины --- элементы, а рёбра --- сравнимые элементы. Этот граф совершенный. Действительно, его хроматическое число $k$, а кликовое тоже $k$.\\

\theoremn{Шпернер} Пусть $|A|=m$. Тогда максимальное количество подмножеств, которые можно выбрать из $A$, чтобы никакое не лежало в никаком другом, равно $\binom m{\floor{\frac{m}{2}}}$.

\proof Пусть $B_i\subset A,|B_i|=k_i$. Рассмотрим все цепи в $A$ длиной $m+1$. Таких цепей $m!$. В каждой из цепей может лежать максимум одно из $B_i$, и каждое $B_i$ лежит в $(m-k_i)!k_i!$ цепей. Это выражение минимально, когда $k_i=m-k_i$ или отличается на 1. В этом случае получается искомая оценка. Пример на такое число --- все подмножества размера $\floor{\frac m2}$.\QEDA\\

\newpage
\hdr{Совершенство}

\lemma Следующие утверждения эквивалентны:
\begin{enumerate}
	\item $G$ совершенный.\label{ideal}
	\item В любом индуцированном $H\subset G$ существует антиклика, которая пересекает все клики максимального размера.\label{anticlick}
	\item То же, что и \ref{anticlick}, но эта антиклика содержит фиксированную вершину $v$.\label{pclick}
\end{enumerate}

\proof Очевидно, что из \ref{pclick} следует \ref{anticlick}. Докажем, что из \ref{ideal} следует \ref{pclick}. Заметим, что нужно доказывать только для $H=G$, т.к. индуцированные подграфы $G$ совершенны. Покрасим $G$ в $k=\omega(G)$ цветов. Каждая клика размера $k$ теперь покрашена в $k$ цветов, и можно взять антиклику --- цвет с вершиной $v$.

Теперь выведем \ref{ideal} из \ref{anticlick}. Пусть $\omega(G)=k$. Возьмём антиклику и покрасим её в 1-й цвет, затем уберём её. Получим подграф, для которого условие также выполняется. Индукция по $k$.\QEDA\\

Рассмотрим операцию <<дублирования>> вершины $v$: создадим вершину $v'$, а потом соединим её с $v$ и всеми её соседями.

\lemma После дублирования совершенный граф останется совершенным.

\proof Пусть мы дублировали $v$. Если $v$ входила в максимальную клику в $G$, то $\omega(G)$ увеличилось на 1 и можно покрасить $v'$ в новый цвет. Пусть это не так. Проверим условие \ref{anticlick}. Если в $H$ нет $v$ и $v'$ одновременно, то это условие верно. Пусть в $H$ есть $v$ и $v'$. Размер максимальной клики не увеличился. Значит, сейчас максимальные клики либо размера $k$, либо размера $k-2$ и содержат обе вершины $v$ и $v'$. В обоих случаях всё получается.\QEDA\\
% TODO: записать нормальное решение

\prooft{maintheorem} Пусть $G$ --- индуцированный подграф совершенного графа, $\alpha(G)=x$. Возьмём каждую вершину $v$ и продублируем её $\alpha_v-1$ раз, где $\alpha_v$ --- количество антиклик с $v$ размера $x$ (если $\alpha_v=0$, то стираем $v$). Получим граф $H$. Разрежем $H$ на антиклики таким образом. У нас $\alpha_v$ антиклик с вершиной $v$ и столько же копий $v$, и можно произвольно провести биекцию между ними. Теперь у нас $H$ разбился на несколько антиклик размером $k$. Заметим, что это его оптимальная раскраска, т.к. $\alpha(H)=k$. Пусть в ней $t$ цветов, тогда $\chi(H)=t$. Т.к. $H$ совершенный (наши операции этого не ломали), то $\omega(H)=t$. Посмотрим на $X$ --- прообраз клики размера $t$. Мы получим клику, но возможно, меньше $t$. Посмотрим на любую антиклику в $G$ размера $k$. Она пересекается с $X$, а это условие на совершенство $\overline{G}$.\QEDA\\

\theoremn{Дилуорс} Пусть максимальный размер антицепи в $M$ равен $k$. Тогда $M$ можно разбить на $k$ цепей.

\proof Воспользуемся \ref{mirsk}. Получим, что граф частично упорядоченного множества совершенный. Тогда по \ref{maintheorem} и его дополнение совершенно. Отсюда и следует теорема.\QEDA\\

\hdr{Хордальные графы}

\definition{Хордальный граф} такой граф, что в любом цикле длины хотя бы 4 есть хорда.

\theorem Хордальные графы совершенные.

\proof Пусть $G$ хордальный, $\omega(G)=k$.  \QEDA\\

\end{document}
