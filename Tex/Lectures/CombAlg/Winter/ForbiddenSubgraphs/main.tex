\documentclass[12pt,a4paper]{article}
\usepackage{tpl}
\newcommand{\q}[1]{{\underline{#1}}}
\dbegin[22 февраля 2020 г.]{Комбинаторика и алгоритмы, зима 2020}{Графы с запрещёнными подграфами}{Трушков Владимир Викторович}

Обозначим за $ex(n,G)$ максимальное число рёбер на $n$ вершинах в графе, в котором отсутствует подграф $G$. 

\theorem $ex(n,K_3)=\floor{\frac{n^2}{4}}$.\label{turanth}
\prooftms{turanth}

\prooftm{turanth} Пример очевиден (полный двудольный граф). Для оценки рассмотрим вершину самой большой степени. Пусть её степень $k$. Заметим, что в нижней компоненте нет рёбер, а в верхней $n-k$ вершин. Тогда если максимальная степень $k$, то максимум рёбер $k(n-k)$ и оно максимальное, когда $k=n-k$ или отличается на 1.\QEDA\\
\inkscapefigure{turan}{Треугольники}

\prooftm{turanth} Доказываем оценку для $2n$ по индукции (для $2n+1$ аналогично). Уберём любые две соединённые вершины, тогда из них в старые $2n$ выходит максимум $2n$ рёбер, откуда всё следует.\QEDA\\

\theorem $ex(n,K_m)=\frac{m-2}{2m-2}(n^2-r^2)+C^2_r$, где $r=n\%(m-1)$.

\proof Аналогично доказательству 2, пример --- полный $m$-дольный граф.\QEDA\\

\theoremn{Эрдёш, Шимонович} $ex(n,G)=\frac{\chi(G)-2}{2\chi(G)-2}n^2+o(n^2)$.

\theoremn{Гипотеза Алона} $ex(n,K_{s,t})=\gamma_tn^{2-\frac{1}{s}}+o(n^{2-\frac{1}{s}})$.\label{hypothesis}

\theorem В графе 20 вершин, степень каждой вершины хотя бы 10. Тогда в нём есть $K_{3,3}$.\label{vser}

\proof Запишем для каждой вершины все тройки её соседей. Тогда записано хотя бы $20\binom{10}3$ троек. Заметим, что это больше, чем $2\binom{20}3$, т.е. удвоенное количество троек. Значит, какая-то тройка встретилась хотя бы трижды. Это $K_{3,3}$.\QEDA\\

Мы хотим найти $ex(n,K_{2,2})$. Пусть $d_1,\ldots ,d_n$ --- степени вершин. Сделаем так же, как в \ref{vser}. Тогда мы запишем $\sum_i\binom{d_i}2$ пар. Если это больше, чем $\binom n 2$, то в графе есть $K_{2,2}$. Тогда:
\begin{align*}
	\sum_i \frac{d_i^2-d_i}2>\frac{n(n-1)}2\\
	\sum_i d_i^2 - \sum_i d_i>n(n-1)\\
	\textcolor{blue}{\sqrt{\frac{\sum d_i^2}n}\geq \frac{\sum d_i}n}\\
	\textcolor{blue}{\sum d_i^2\geq \frac1n\left(\sum d_i^2\right)}\\
\end{align*}

Тогда из этих неравенств (и из того, что $e=\frac{1}{2}\sum d_i$) мы получаем, что $\frac{2e^2}{n}-e-\frac{n(n-1)}{2}>0$. Значит, $e>\frac{1+\sqrt{4n-3}}{2}n\sim \frac{1}{2}n^{\frac{3}{2}}$, при условии, что в графе нет $K_{2,2}$.

Заметим, что если $n=p^2+p+1$, то $4n-3=(2p+1)^2$, т.е. корень в этой формуле извлекается. А ещё $p^2+p+1$ --- это количество точек в $\mathbb C_p$.\\

\hdr{Конечная геометрия}

Будем рассматривать тройки $x:y:z$ остатков по модулю $p$, которые не все одновременно равны 0, и разобьём на классы эквивалентности $kx:ky:kz(k\in \mathbb Z_p)$. Заметим, что их $p^2+p+1$. Также можно рассмотреть прямые вида $ax+by+cz=0$.

Рассмотрим граф, у которого вершины --- точки конечной геометрии, и точки $a:b:c$ и $x:y:z$ соединены ребром, если $ax+by+cz=0$. Заметим, что в этом графе нет  $K_{2,2}$. Действительно, пусть в одной компоненте графа  $a_1:b_1:c_1$ и $a_2:b_2:c_2$, а в другой $x_1:y_1:z_1$ и $x_2:y_2:z_2$. Тогда точки $x_i:y_i:z_i$ обе лежат на прямой через точки $a_i:b_i:c_i$.

Теперь посчитаем количество точек на каждой прямой. Заметим, что наша структура --- квадрат $p\times p$, прямая с $p$ точками на бесконечности и бесконечная точка. Значит, на каждой прямой $p+1$ точка. Значит, у каждой вершины степень $p+1$ и мы получаем точный пример\ldots  Но это не работает, в графе теперь есть петли. Их столько же, сколько решений у уравнения $a^2+b^2+c^2=0$ в остатках. Легко понять, что их не больше, чем $2p+1$. Значит, у нас пример на $(p^2+p+1)\frac{p+1}{2}-(2p+1)$ для всех простых $p$ (когда вершин $p^2+p+1$). Также по \textbf{улучшенному постулату Бертрана} этот пример почти идеален для чисел другого вида.\QEDA\\

\definition{Обобщённое число Рамсея $r(G_1,\ldots ,G_n)$} такое минимальное $x$, что в любом графе на $x$ вершинах, покрашенном в $n$ цветов, есть $G_i$, покрашенный в $i$-й цвет, для какого-то $i$.

\theorem $r(C_4,C_4,C_4,C_4)\leq 21$.

\proof В графе $K_{21}$ 210 рёбер. Тогда есть 53 ребра какого-то одного цвета. С другой стороны, из формулы выше следует, что если в графе на 21 вершинах нет $C_4$, то в нём максимум 52 ребра --- противоречие.\QEDA\\

\hdr{Большие запреты}

Будем запрещать подграф $K_{2,m+1}$. Пусть степени вершин --- $d_1,\ldots ,d_n$. Этот подграф точно есть, если $\sum_i\binom{d_i}2>m\binom n 2$. Тогда $\frac{(\sum d_i)^2}{2n}-\frac{\sum d_i}{2}\geq m\binom n 2$.

\z Какое максимальное $|E|$, если это неравенство не выполняется?\\

Ответом тут будет что-то порядка $\frac{\sqrt m}{2}n^{\frac{3}{2}}$. Для получения примера посмотрим на поле размером $q=p^{\varphi(m)}$. Пусть решения уравнения $x^m=1$ в нём --- числа $\varepsilon_1,\ldots ,\varepsilon_m$.

Посмотрим на это поле. Заметим, что в нём $(q-1)^2$ пар $(x,y)$, таких, что $xy\neq 0$. Будем считать, что пары $(x,y)$ и $(\varepsilon_kx,\varepsilon_ky)$ одинаковые, тогда таких классов эквивалентности  $\frac{(q-1)^2}{m}$. Будем обозначать их $<x,y>$.

Обозначим $T_{<a,b>}$ --- множество таких пар $<x,y>$, что $(ax+by)^m=1$. Это нормальное определение: если $x$ и $y$ умножить на $\varepsilon_i$, то при возведении в степень $m$ получится $ax+by$.

\lemma В $T_{<a,b>}$ элементов ровно  $q-2$.

\proof Пусть $(ax+by)^m=1$. Тогда $ax+by=\varepsilon_k$, откуда $y_k=\frac{\varepsilon_k-ax}{b}$. Получается, что каждому $y$ соответствует $m$ иксов. Аналогично каждому $x$ соответствует $m$ игреков. При этом у нас 1 запрет на икс, 1 запрет на игрек и они разные. Кроме того, мы потом делим на $m$.\QEDA\\

\lemma $|T_{<a,b>}\cap T_{<c,d>}|\leq m$.\label{intersection}

\proof Пусть $(ax+by)^m=(cx+dy)^m=1$. Тогда $ax+by=\varepsilon_k(cx+dy)$ и $x(a-c\varepsilon_k)=y(d\varepsilon_k-b)$. Значит, $(x(a+b\ell_k))^m=1$, у этого уравнения не более $m^2$ решений, но они разбиваются на классы эквивалентности.\QEDA\\

\textbf{Построение (Furedi, 1996).} Пусть вершины графа --- $<x,y>$ и две вершины $<a,b>$ и $<c,d>$ соединены ребром, если $(ac+bd)^m=1$. Тогда в графе нет $K_{2,m+1}$ по \ref{intersection}. Однако у этого графа есть петли, но их максимум $2(q-1)$. Таким образом, \[
	E\geq \frac{1}{2}\left(\frac{(q-2)(q-1)^2}{m}-2(q-1)\right)>\frac{\sqrt m}{2}n^{\frac{3}{2}}
.\]

\hdr{Нерешённая часть}

Когда мы оцениваем количество рёбер в графе без $K_{2,m}$, у нас получается квадратное неравенство и в нём всё хорошо. Но при больших $s,t$ проблемы, потому что больше 1 интервала, больше корней, нет формул и т.п. Для решения такой задачи обозначим $d=\frac{2e}{v}$ и $x^\q{n}=x(x-1)\ldots (x-n+1)$.

\lemma Пусть в графе на $v$ вершинах нет $K_{m,n}$. Тогда $d^\q{n}\leq (m-1)(v-1)^\q{n-1}$.

\proof Если $d\leq n-1$, то утверждение очевидно. Значит, $d>n-1$. Если есть вершина $v_j$ с меньше чем $d-1$ вершинами, то есть и $v_i$ больше чем с $d-1$. Заметим, что \[
	d_i^\q{n}+d_j^\q{n}=d_i^\q{n}\geq (d_i+d_j-n+1)^\q n=(d_i+d_j-n+1)^\q n+(n-1)^\q n
.\] Значит, иметь маленькие точки нам невыгодно. Теперь заметим, что при $x>n-1$ функция $y=x^\q n$ выпукла вниз, потому что коэффициент при $x^{n-2}$ у неё положительный, тогда по неравенству Йенсена \[
	d^\q n\leq \frac{1}{n}\cdot \left( \sum_i d_i^\q n\right)\leq (m-1)(v-1)^\q {n-1}
,\] где последнее неравенство следует из подсчёта групп по $m$.\QEDA\\

\theoremn{Kovari, S\'os,Tur\'an, 1955} $ex(v,K_{m,n})\leq \frac{1}{2}\left((m-1)^{\frac{1}{n}}v^{2-\frac{1}{n}}+nv\right)$.

\proof Обозначим $d_0=(m-1)^{\frac{1}{n}}v^{1-\frac{1}{n}}+n$. Тогда $d_o^\q n>\left((m-1)^{\frac{1}{n}}v^{1-{\frac{1}{n}}}\right)^n>(m-1)v^{n-1}>(m-1)(v-1)v^\q{n-1}>d$. Значит, средняя степень не больше $d_0$, откуда и следует теорема.\QEDA\\

\theoremn{Алон} Если $n>(m-1)!$, то \ref{hypothesis} верно для  $\gamma=\frac{1}{2}(m-1)^{\frac{1}{n}}$.

\theoremn{Эрдёш, Шимонович} $ex(v,cube)=\alpha n^{\frac{8}{5}}+o(\ldots )$.\\

\hdr{$K_{3,3}$}

Рассмотрим $\mathbb Z_p^3$, где $p=4k+3$, тройки точек $(x,y,z)$ и сферы $(x-x_0)^+(y-y_0)^2+(z-z_0)^2=r^2$.

\lemma Никакие три точки сферы не лежат на одной прямой.

\proof Пусть прямая проходит через начало координат. Для неё $x=t a_1,y=t a_2,z=t a_3$. Тогда у нас квадратное уравнение относительно $t$. Пусть его коэффициенты занулились. У прямой какие-то коэффициенты не нули, пусть $a_1\neq 0$. Тогда \[
	a_1^2r^2=a_1^2(x_0^2+y_0^2+z_0^2)=(a_2y_0+a_3z_0)^2+a_1^2y_0^2+z_0^2=-(a_3y_0+a_2z_0)^2
.\] Но если $p=4k+3$, такого не бывает, потому что $a_1\neq 0$, а $-1$ невычет.\QEDA\\

Зафиксируем $r$. Рассмотрим такой граф. Его вершины --- точки, и две точки соединены ребром, если расстояние между ними $r^2$ (т.е. одна лежит на сфере радиуса $r$ с центром в другой точке).

\lemma В этом графе нет $K_{3,3}$.

\proof Пусть $a,a',a''$ --- точки в одной доле подграфа, а в другой есть $(x_0,y_0,z_0)$. Это значит, что $(x_0,y_0,z_0)$ лежит на трёх сферах. Получаем систему относительно $x_0,y_0,z_0$. Когда мы вычтем первое уравнение из остальных, мы получим либо два уравнения плоскостей через $(0,0,0)$ (тогда прямая пересечения пересекается со сферой в трёх точках  $(x_0,y_0,z_0)$), либо что $a,a',a''$ пропорциональны.\QEDA\\

\lemma Степень каждой вершины $p^2-p$.

Тогда из леммы выводится, что если $v=p^3$, то у нас пример на $e=\frac{p^5-p^4}{2}$ (\textbf{Браун, 1966}).

\end{document}
