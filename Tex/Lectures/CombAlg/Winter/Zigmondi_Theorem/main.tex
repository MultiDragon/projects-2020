\documentclass[12pt,a4paper]{article}
\usepackage{tpl}
\dbegin[26 февраля 2020 г.]{Комбинаторика и алгоритмы, зима 2020}{Теорема Зигмонди}{Гусев Антон Сергеевич}

Пусть нам даны числа $a>b$. Обозначим $d_n=a^n-b^n,s_n=a^n+b^n$.

\theoremn{Зигмонди $-$}\label{minust} Для любого $n$ у $d_n$ есть простой делитель, которого нет у $d_k$ при $k<n$, кроме случаев
\begin{enumerate}
	\item $n=2,a+b=2^l$. Тогда $2\mid a-b$ и других делителей у $a+b$ нет.
	\item $n=6,a=2,b=1$. Тогда у $63$ простые делители $3$ (встречается при $n=2$) и $7$ ($n=3$).
\end{enumerate}

\theoremn{Зигмонди $+$}\label{plust} То же, что и \ref{minust}, для $s_n$. Исключение --- $n=3,a=2,b=1$.
\vskip10pt\hrule\vskip20pt

Обозначим $ord_p(x)=ord_{p_j}(p_1^{\alpha_1}\cdot \ldots \cdot p_n^{\alpha_n})=\alpha_j$.

\lemman{об уточнении показателя}\label{utoch} Пусть $a,b,n\in\mathbb N,p\in\mathbb P,p\mid a-b,p\nmid a,p\nmid b$. Тогда:
\begin{enumerate}
	\item Если $p>2$, то $ord_p(a^n-b^n)=ord_p(a-b)+ord_p(n)$.
	\item Если $p=2$, то это верно при условии, что $4\mid a-b$.
\end{enumerate}

\proof Рассмотрим несколько случаев:
\begin{enumerate}
	\item Пусть $p\nmid n$. Тогда $a^n-b^n=(a-b)(a^{n-1}+a^{n-2}b+\ldots +b^{n-1})$. Докажем, что вторая скобка не делится на $p$. Так как $a\equiv b\mod p$, то эта скобка сравнима с $a^{n-1}n$, но $p\nmid a,p\nmid n$.
	\item Пусть $p=n$. Тогда ясно, что вторая скобка делится на $p$ (там $p$ слагаемых, каждое сравнимо с 1). Пусть $a-b=pk$. Если $p\mid k$, то это верно (аналогично предыдущему пункту). Значит, $p\nmid k$. Тогда пусть $k=px-m$. Значит, \[
		a^{p-1}+a^{p-2}b+\ldots +b^{p-1}\equiv a^{p-1}+a^{p-2}(a+mp)+\ldots +(a+mp)^{p-1}\mod p^2
	.\] Посмотрим на $a^{p-1-t}(a+pm)^t\mod p^2$. В разложении этого по биному почти все слагаемые делятся на  $p^2$, остаётся $a^{p-1}+a^{p-2}pmt$. Когда мы сложим все такие слагаемые, мы получим $pa^{p-1}+a^{p-2}pm\frac{p(p-1)}{2}\mod p^2$ и первое слагаемое не делится на $p^2$ при $p>2$, а второе делится. Если же $p=2$, то это тоже работает, потому что $m=0$.
\end{enumerate}
\textbf{Завершение доказательства.} Пусть $n=p^rs,p\nmid s$ и $r>0$. Тогда \[
	ord_p\left(a^n-b^n\right)=ord_p\left(\left(a^{p^{r-1}}s\right)^p-\left(b^{p^{r-1}}s\right)^p\right)=1+ord_p\left(a^{\frac{n}{p}}-b^{\frac{n}{p}}\right)
.\] Индукция по $r$. \QEDA\\

\hdr{Многочлены деления круга}

Рассмотрим $P(x)=x^n-1$. Заметим, что корни этого многочлена --- точки, которые делят единичную комплексную окружность на $n$ равных частей. Рассмотрим такое $\alpha$, что $\lrp{\cos\alpha+i\sin\alpha}^n=1$. Тогда по формуле Муавра получается $\alpha=\frac{2\pi k}{n}$. Тогда если рассмотреть $x_1$ --- корень с самым маленьким положительным аргументом, то получится $x_j=x_1^j$ для всех остальных корней. Можно записать $\xi_n=x_1$. Тогда обозначим \textbf{многочлен деления круга} --- \[
	\Phi_n(x)=\prod\limits_{k=1;\ (k,n)=1}^n (x-\xi_n^k)
.\]

\shdr{Примеры}

\begin{itemize}
	\item $\Phi_1(x)=x-1$.
	\item $\Phi_2(x)=x-(-1)=x+1$.
	\item $\Phi_3(x)=\frac{x^3-1}{x-1}=x^2+x+1$.
	\item $\Phi_4(x)=(x-i)(x+i)=x^2+1$.
\end{itemize}

\lemma $\varphi(n)=\sum_{d\mid n}\varphi(d)$.

\proof Рассмотрим дроби $\frac{1}{n}, \frac{2}{n}, \ldots , \frac{n}{n}$ и сократим их. Тогда для каждого $d\mid n$ дробей со знаменателем $d$ будет $\varphi(d)$.\QEDA\\

\lemma $x^n-1=\prod_{d\mid n}\Phi_d(x)$.\label{fprod}

\proof Если мы докажем, что у них совпадают степень, старший коэффициент и корни, то мы докажем это. У обоих многочленов старший коэффициент 1 и степень $n$ (потому что у $\Phi_m$ степень $\varphi(m)$). Набор корней $x^n-1$ --- это $\xi_n^k$ для разных $k$. Заметим, что $\xi_n^k=\xi_{n/t}^{k/t}$. Тогда если взять $t=(n,k)$, то это число будет корнем $\Phi_{n/t}(x)$.\QEDA\\

\lemma У $\Phi_n$ целые коэффициенты.

\proof Индукция по $n$.

\textbf{База.} $n=1,2$.

\textbf{Переход $n\leq k\to n=k+1$.} По \ref{fprod} мы знаем, что $x^{k+1}-1=\prod_{d\mid k+1}\Phi_d(x)$, то есть $\Phi_{k+1}=\frac{x{k+1}-1}{\prod\Phi_{\ldots} }$ и у знаменателя целые коэффициенты по предположению индукции. Заметим, что когда мы будем делить многочлен $\mathbb Q[x]$ на другой многочлен $\mathbb Q[x]$, получим $\mathbb Q[x]$ (если поделится нацело). Значит, у $\Phi_{k+1}$ рациональные коэффициенты. Теперь у нас $x^{k+1}-1=\Phi_{k+1}(x)g(x),g(x)\in\mathbb Z[x]$. Вынесем из произведения $t$ --- общий знаменатель всех коэффициентов $\Phi_{k+1}$. Пусть $p\mid t$ и в обоих многочленах есть коэффициент, который не делится на $p$. Тогда рассмотрим минимальный такой коэффициент $a_ix^i$ и минимальный аналогичный коэффициент $b_jx^j$. Тогда коэффициент при $x^{i+j}$ должен делиться на $p$, но он не делится --- противоречие. В $g(x)$ есть такой коэффициент (потому что старший член 1), значит, все коэффициенты $\Phi_{k+1}$ делятся на $p$. Значит, на самом деле $\Phi_{k+1}\in \mathbb Z[x]$.
\QEDA\\

\lemman{Критерий Эйзенштейна} Пусть у $P(x)$ все коэффициенты, кроме старшего, кратны $p$, и свободный член не делится на $p^2$. Тогда $P(x)$ неприводим над $\mathbb Q$.\label{eizens}

\proof Пусть $P(x)=\sum a_ix^i=\lrp{\sum b_jx^j}\lrp{\sum c_kx^k}$. Ровно один из коэффициентов $b_0$ и $c_0$ делится на $p$. Пусть $p\mid b_0$. Заметим, что $p\nmid b_m$, значит, можно рассмотреть минимальное такое $i$, что $b_i$ не кратно $p$. Тогда коэффициент при $x^i$ у $P(x)$ не может делиться на $p$ --- противоречие.\QEDA\\

\lemma $\Phi_p(x)$ неприводим над $\mathbb Q$.

\proof $\Phi_p(x)=\frac{x^p-1}{x-1}=x^{p-1}+\ldots +x+1$. Подставим $x=y+1$. Тогда у получившегося многочлена свободный член будет равен $p$, а все остальные, кроме старшего, кратны $p$, что противоречит \ref{eizens}.\QEDA\\

\lemma Пусть $n\in\mathbb N,p\in\mathbb P$. Тогда
\begin{align*}
	\Phi_{np}(x)=
	\begin{cases}
		\Phi_n(x^p),p\mid n\\
		\dfrac{\Phi_n(x^p)}{\Phi_n(x)}, p\nmid n
	\end{cases}
\end{align*}

\proof Пусть $p\mid n$. Тогда у этих многочленов одинаковый старший член 1 и одинаковая степень $\varphi(np)$. Посмотрим на их корни. У первого многочлена это точки, делящие окружность на $np$ частей, но не на меньшее количество, а у второго те же самые точки.

Теперь пусть $p\nmid n$. Старший коэффициент у них снова 1. Степень $\Phi_{np}(x)$ равна $\varphi(n)(p-1)$, степень  $\Phi_n(x^p)$ равна $p\varphi(n)$, степень $\Phi_n(x)$ равна $\varphi(n)$. Докажем про корни. Корни $\Phi_{np}(x)$ --- те точки, номера которых взаимно просты с $np$, а корни $\Phi_n$ --- те, которые взаимно просты с $n$ и делятся на $p$. Тогда у произведения корни --- все точки, номера которых взаимно просты с $n$.\QEDA\\

\textbf{Пример использования леммы.} Лемма позволяет посчитать любое $\Phi_n$ через $\Phi_p$: \[
	\Phi_{12}(x)=\Phi_6(x^2)=\frac{\Phi_3(x^4)}{\Phi_3(x^2)}=\frac{x^8+x^4+1}{x^4+x^2+1}
.\]

\newpage
\lemma Пусть $k\mid n$. Тогда $\Phi_n(a)(a^k-1)\mid a^n-1$.\label{div}

\proof \[
	a^n-1=\prod\limits_{d\mid n}\Phi_d(a)=\Phi_n(a)\cdot \prod\limits_{d\mid n,d<n}\Phi_d(a)
,\] причём в правом произведении есть множитель $a^k-1$.\QEDA\\

\theoremn{Упрощение Зигмонди}\label{easier} Пусть $b=1$. Тогда для любого $n$ у $d_n$ есть простой делитель, которого нет у $d_k$ при $k<n$, кроме тех же двух исключений.

\lemma Если \ref{easier} неверна, то $p\mid \Phi_n(a)\implies p\mid n$.

\proof $p\mid\Phi_n(a)\implies p\mid a^n-1$. Тогда если теорема \ref{easier} неверна, то $p\mid a^k-1$ для какого-то $k<n$. Тогда по алгоритму Евклида можно получить, что $p\mid a^m-1$ для какого-то $m\mid n$. Применим \ref{div}. Тогда $p(a^k-1)\mid a^n-1$. Тогда если $p>2$ или $4\mid a^k-1$, то по \ref{utoch} мы получим \[
	ord_p(a^n-1)=ord_p(a^k-1)+ord_p\lrp{\frac{n}{k}}\implies p\mid \frac{n}{k}\implies p\mid n
,\] что и требовалось. Пусть $p=2$ и $a^k-1$ не делится на 4. Кроме того, $2\mid a^n-1\implies k=1$. Тогда если \ref{utoch} не работает, то  $4\nmid a-1$. Значит, $a=4k+3$. Пусть $2\nmid n$. Тогда $ord_2(a^n-1)=ord_2(a^k-1)$, противоречие с \ref{div}.\QEDA\\

\lemma $\Phi_n(a)$ свободно от квадратов.\label{free}

\proof Докажем, что $p^2\nmid\Phi_n(a)$. Пусть $n=p^\alpha s$. Тогда $\Phi_n(a)\cdot \lrp{a^{p^{\alpha-1}s}-1}\mid a^{p^\alpha s}-1$. Заметим, что $t\mid s$, где $t$ --- показатель $a\mod p$. Тогда знаменатель кратен $p$. Применим \ref{utoch}. Получим, что степень вхождения $p$ в $(a^{n/p}-1)/(a^n -1)$ равен 1, кроме случая $p=2$.\QEDA\\

\lemma Пусть $p\mid \Phi_n(a)$ и $n=p^\alpha s$. Тогда $s$ --- показатель $a$ по $p$.\label{order}

\proof Пусть $T$ --- показатель. Тогда \[
	p\mid \Phi_n(a)\mid \frac{a^n-1}{a^{p^\alpha T}-1} \text{ и } 1\leq ord\lrp{\frac{a^{p^\alpha s}-1}{a^{p^\alpha T}-1}}=ord\lrp{\frac{a^s-1}{a^T-1}}=0
.\QEDA\]

\prooft{easier} Заметим, что по \ref{order} если $p\mid F_n(a)$, то $p$ --- самый большой простой делитель в $n$. Значит, простых делителей максимум 1. Тогда по \ref{free} $F_n(a)=p$. С другой стороны, $\Phi_n(a)=\Phi_{p^\alpha s}(a)$. Рассмотрим несколько случаев:

\begin{enumerate}
	\item Пусть $\alpha>1$. Тогда $\Phi_n(a)=\Phi_{n/p}(a^p)\geq a^p-1>p$. Такого не бывает.
	\item Пусть $\alpha=1,a\neq2$. Тогда $\Phi_{ps}\geq (a-1)^{\varphi(ps)}\geq 2^{p-1}$.
	\item Пусть $\alpha=1,a=2$. Тогда $\Phi_{ps}(2)=\frac{\Phi_s(2^p)}{\Phi_s(2)}\geq \frac{(2^p-1)^{\varphi(s)}}{3^{\varphi(s)}}$. Если $p>3$, то такого не бывает. Если $p=3$, то либо $n=3$, либо $n=6$. Если $p=2$, то $n=2$.
\end{enumerate}

Итак, мы доказали, что у $\Phi_n(a)$ есть <<уникальный>> простой делитель $p$. Тогда $p\mid a^n-1$ и если $p\mid a^k-1$, то $p\mid\Phi_y(a)$ для какого-то $y$ --- противоречие.\QEDA\\

\prooftf{minust}{(идея)} Обозначим $\Phi_n(a,b)=\lrp{\frac{b}{a}}^{\varphi(n)}\Phi_n(a)$. Дальше надо те же самые рассуждения провести для $\Phi_n(a,b)$.\QEDA\\

\prooftf{plust}{через \ref{minust}} Очевидно следует из того, что $a^k+b^k\mid a^{2k}-b^{2k}$.\QEDA\\

\end{document}
