\documentclass[12pt,a4paper]{article}
\usepackage{tpl}
\newcommand{\del}{Делон\'e}
\dbegin[23 февраля 2020 г.]{Комбинаторика и алгоритмы, зима 2020}{Триангуляции \del}{Соколов Артемий Алексеевич}

Будем рассматривать триангуляции конечного множества точек (т.е. разбиения выпуклой оболочки множества на треугольники с вершинами в этих точках).

\inkscapefigure{triangulations}{Триангуляция}

\lemma В каждой триангуляции одинаковое количество рёбер.

\proof Для триангуляции $V-E+F=2$, причём $V=n$ --- количество точек в множестве и $2E=3(F-1)+k$, где $k$ --- количество точек в выпуклой оболочке. Это линейное уравнение относительно $E$. \QEDA\\

\shdr{Алгоритмы построения триангуляции}

\begin{itemize}
	\item По индукции: когда мы добавляем новую точку, она лежит в какой-то грани, тогда мы проводим триангуляцию этой грани. Этот алгоритм примерно квадратичный, потому что надо искать треугольник, в котором лежит точка.
	\inkscapefigure{inductive-algorithm}{Индуктивный алгоритм}
	\item Ещё можно отсортировать все точки по координате и идти справа налево, тогда каждый раз новая точка вне выпуклой оболочки.
\end{itemize}

\definition{Триангуляция \del} такая триангуляция, что описанная окружность каждого треугольника триангуляции не содержит других точек множества.\label{global}

\definition{Локальное свойство \del} свойство триангуляции, что на каждом ребре описанные окружности треугольников на этом ребре не содержат противоположных точек.\label{local}

\theorem Если для каждого ребра выполнено \ref{local}, то выполнено и \ref{global}.

\proof Пусть описанная окружность $\triangle ABC$ содержит точку $D$. Докажем, что описанная окружность $\triangle BCE$ тоже её содержит. Действительно, $\angle BAC+\angle BEC>180^\circ>\angle BAC+\angle BDC$. Тогда мы стали <<ближе>> к точке, а локальное свойство снова не выполнено. В какой-то момент мы дойдём до неё, значит, \ref{local} не выполнено.\QEDA\\

\begin{tikzpicture}
	\qsetscale{0.5}
	\qcoord A58
	\qcoord B35
	\qcoord C85
	\qcspoint A\qcspoint[left] B\qcspoint[right] C
	\qcccircle ABC
	\qcoord D6{3.5}
	\qcoord E02
	\qcspoint[above,green] D\qcspoint[left] E
	\qctriangle ABC
	\qctriangle BEC
	\qcsegment[green] BD
	\qcsegment[red] CD
	\qcsegment[red] DE
\end{tikzpicture}

\textbf{Алгоритм получения триангуляции \del.} Возьмём любую триангуляцию. Пусть она не \del. Тогда для какого-то ребра не выполнено \ref{local}. Заменим это ребро в 4-угольнике на другое.

\theorem Этот процесс заканчивается.\label{main}
\prooftms{main}

\prooftm{main} Мы каждый 4-угольник меняем не более 1 раза, значит, процесс конечен.\QEDA\\

\prooftm{main} Пусть наши точки расположены на плоскости $Oxy$. Построим параболоид $z=x^2+y^2$ и для каждой точки триангуляции поднимем её до параболоида. Заметим, что уравнение любой окружности имеет вид $x^2+y^2-kx-ly-c=0$, но $x^2+y^2=z$, значит, эта окружность перешла в пересечение плоскости с параболоидом. Пусть есть 2 треугольника триангуляции. Они перешли в тетраэдр на параболоиде. Тогда когда мы делаем флип, мы заменяем две <<верхние>> плоскости тетраэдра на <<верхнюю>> и <<нижнюю>>. Также \ref{global} означает, что плоскости разбиения параболоида образуют выпуклую фигуру, а при флипе объём увеличивается.\QEDA

В частности, это доказывает, что триангуляция \del\ существует (если точки общего положения).\\

\definition{Область Вороного $V_P$ точки $P$} множество точек, которые ближе к $P$, чем к другим точкам множества. Это пересечение каких-то полуплоскостей, значит, это выпуклый многоугольник.\\

Рассмотрим такое разбиение на фигуры. Соединим $P$ и $Q$ отрезком, если $|V_P\cap V_Q|\geq 2$. Заметим, что если никакие 4 точки не лежат на 1 окружности, то это триангуляция.

\theorem Это триангуляция \del.

\textit{<<Вы когда-нибудь видели спину жирафа? Она из многоугольников состоит. Так вот, это области Воронова, потому что пигмент разрастается.>>}

\theorem Рассмотрим самый маленький угол в треугольниках триангуляции. Тогда в триангуляции \del\ он наибольший.

\proof Из любой триангуляции можно перейти в \del\ флипами. Из рисунка ниже очевидно, что при флипе минимальный угол не уменьшается.
\inkscapefigure[4]{min-angle}{min angle}
\QEDA\\

\textbf{Применение.} Пусть мы хотим приблизить функцию $f(x,y)$ на множестве точек $S$. Тогда хороший план --- триангулировать $S$ и на каждом треугольнике построить плоскость. Оказывается, что оптимум этого алгоритма --- когда триангуляция \del.

\end{document}
