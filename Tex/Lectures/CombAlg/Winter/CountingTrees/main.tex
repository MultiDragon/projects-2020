\documentclass[12pt,a4paper]{article}
\usepackage{tpl}
\dbegin[24 февраля 2020 г.]{Комбинаторика и алгоритмы, зима 2020}{Перечисление деревьев и электрические цепи}{Бычков Борис Сергеевич}

Нас интересует количество остовных деревьев данного графа $G$.

\theoremn{Кэли} Количество остовных деревьев $K_n$ равно $n^{n-2}$.\\

\definition{Определитель матрицы $\det(A)=|A|$} ориентированный объём $m$-мерного параллелепипеда, натянутого на вектора, координаты которых образуют строки матрицы $A(m\times m)$.

\definition{Определитель матрицы $\det(A)$} число, получаемое по следующей формуле: \[
	\det(A_{i,j})=\sum\limits_{\sigma\in S_m}(-1)^{inv(\sigma)}\prod\limits_{i=1}^m A_{i,\sigma(i)}
.\]

\definition{Алгебраическое дополнение к $A_{i,j}$} число, равное $(-1)^{i+j}\cdot \det B$, где $B$ --- матрица, получаемая из $A$ вычёркиванием $i$-й строки и $j$-го столбца.

\definition{Матрица Кирхгофа} матрица такая, что $A_{i,j}=-1$ при $i\neq j$ и $A_{i,i}=\deg v_i$.

\theorem Все алгебраические дополнения матрицы Кирхгофа равны между собой и равны количеству остовных деревьев графа.\label{kirg}\\

Пусть $G$ ориентирован. Обозначим $x_{i,j}$ --- проводимость ребра $(v_i,v_j)$.

\definition{$T$-матрица отклика} такая матрица что $A_{i,i}=\sum_{j\neq i}x_{i,j}; A_{i,j}=-x_{i,j}$.\\

Заметим, что в этом случае каждому остовному дереву $G$ соответствует ориентированное корневое остовное дерево (т.е. дерево, где все пути идут в корень).

\lemma При фиксированном корне количество ориентированных корневых остовных деревьев и количество остовных деревьев равны между собой.\\

\definition{Производящая функция остовных деревьев} такое выражение от переменных $x_{i,j}$:\[
	S_*(\overline{x})=\sum\limits_{\tau\in Tree}\prod\limits_{(i,j)\in\tau} x_{i,j}
.\]

\theoremn{Матричная теорема о деревьях} Все алгебраические дополнения матрицы отклика равны между собой и равны $S_*(\overline{x})$.

Заметим, что это обобщение \ref{kirg}, потому что можно подставить $x_{i,j}=1-\delta_{i,j}$.

\proof Пусть корень --- $v_1$. Рассмотрим алгебраическое дополнение $T_{1,1}$: \[
	T_{1,1}=\det\binom{t_{2,2}\ldots t_{2,n}}{t_{n,2}\ldots t_{n,n}}=\det\binom{\sum x_{2,i}\ldots -x_{2,n}}{-x_{n,2}\ldots \sum x_{i,2}}
.\] Дальше нужно доказывать, что всё с циклами сокращается, но в решении лектора была лажа.\\

\shdr{План доказательства}

\begin{enumerate}
	\item Показать, что все лишние диагональные элементы сокращаются.
	\item Каждому недиагональному слагаемому соответствует цикл (возможно, больше одного); этому слагаемому отвечает подмножество циклов какой-то $\sigma:[n]\to[n]$; каждому подмножеству циклов такой перестановки отвечает выбранное слагаемое, потому что остальную часть множителей можно набрать диагональными элементами.
	\item Пусть $\sigma=(1\ldots k)\cdot \ldots $. Тогда беспорядков в цикле длины $k$ ровно $k-1$. Значит, в каждом цикле произведение $-1$. Тогда коэффициент при слагаемом $\sigma$ будет $\sum(-1)^i\binom mi=0$.
\end{enumerate}

\newpage
\hdr{Физика}

Пусть наш граф --- это электрическая сеть. В каждой вершине есть своё напряжение, а на каждом ребре --- сопротивление.

\textbf{Закон Ома:} $U=IR$, где $U$ --- напряжение, $I$ --- ток, $R$ --- сопротивление.

\textbf{Закон Кирхгофа:} в каждую вершину втекает столько же, сколько и вытекает.

Пусть $x_{i,j}=\frac{1}{R_{i,j}}$ --- это проводимости рёбер. Пусть мы знаем $T$-матрицу и хотим определить разницу потенциалов между стоком и истоком. Заметим, что $\sum_{i\neq k}x_{k,i}(U_k-U_i)=I$ для истока, $-I$ для стока и $0$ для остальных вершин. Мы также знаем, что

\begin{align*}
	t_{i,j}=
	\begin{cases}
		-x_{i,j},i\neq j\\
		\sum x_{i,m}, i=j
	\end{cases}.
\end{align*}

Мы можем переписать систему уравнений как \[
	\sum\limits_{j=1}^n t_{i,j}v_j=I(\delta_{i,k}-\delta_{i,l})
\] для всех $i$. Это $n$ линейных уравнений с $n$ неизвестными, они решаются.\\

\theoremn{Правило Крамера} Пусть есть система из линейных уравнений:

\begin{align*}
	\begin{cases}
		a_{1,1}x_1+\ldots +a_{1,n}x_n=b_1\\
		\vdots
		a_{n,1}x_1+\ldots +a_{n,n}x_n=b_n
	\end{cases}
\end{align*}

Тогда решение этой системы --- $x_k=\frac{\det A}{\det C}$, где $C$ --- система, которая получается из $A$ заменой столбца $A_{i,k}$ на $B$.

\textbf{Пример.} Пусть система выглядит так:

\begin{align*}
	\begin{cases}
		2x+3y=1\\
		x+y=7
	\end{cases}
\end{align*}

Тогда $A=\binom{2\ 3}{1\ 1}$,  $C_x=\binom{1\ 7}{3\ 1}$, $\det C_x=-20,\det A=-1,x=20$.

\end{document}
