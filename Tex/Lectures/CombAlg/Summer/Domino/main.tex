\documentclass[a4paper,12pt]{article}
\usepackage[russian]{babel}
\usepackage[utf8]{inputenc}
\usepackage[a4paper, margin=2.5truecm, top=1.8truecm, bottom=1.3truecm]{geometry}
\usepackage{hyperref}
\usepackage{graphicx}
\usepackage{amsmath,amsfonts,amssymb}
\usepackage{tikz}

\relpenalty=10000 \binoppenalty=10000
\newcounter{theorems}
\newcommand*{\lemma}{\refstepcounter{theorems}{\bf Лемма \arabic{theorems}. }}
\newcommand*{\theorem}{\refstepcounter{theorems}{\bf Теорема \arabic{theorems}. }}
\newcommand*{\lemman}[1]{\refstepcounter{theorems}{\bf Лемма \arabic{theorems} (#1). }}
\newcommand*{\theoremn}[1]{\refstepcounter{theorems}{\bf Теорема \arabic{theorems} (#1). }}

\newcounter{defs}
\newcommand*{\definition}{\refstepcounter{defs}{\bf Определение \arabic{defs}. }}
\newcommand*{\definitionn}[1]{\refstepcounter{defs}{\bf Определение \arabic{defs} (#1). }}
\newcommand*{\QEDA}{\hfill\ensuremath{\blacksquare}}

\newcommand{\proof}{{\bf Доказательство. }}
\newcommand{\statement}{{\bf Утверждение. }}
\newcommand{\prooft}[1]{{\bf Доказательство теоремы~\ref{#1}. }}
\newcommand{\prooftms}[1]{\newcounter{tproof#1}}
\newcommand{\prooftm}[1]{\refstepcounter{tproof#1}{\bf Доказательство теоремы~\ref{#1}~\#\arabic{tproof#1}. }}
\newcommand{\proofl}[1]{{\bf Доказательство леммы~\ref{#1}. }}
\newcommand{\prooflms}[1]{\newcounter{lproof#1}}
\newcommand{\prooflm}[1]{\refstepcounter{lproof#1}{\bf Доказательство леммы~\ref{#1}~\#\arabic{lproof#1}. }}

\newcommand{\fillsq}[4]{\fill (#1,#2) -- (#1,#4) -- (#3,#4) -- (#3,#2) -- cycle;}
\newcommand{\drawsq}[4]{\path[draw] (#1,#2) -- (#1,#4) -- (#3,#4) -- (#3,#2) -- cycle;}

\newcounter{problems}
\newcommand{\z}{\refstepcounter{problems}{\bf Задача \arabic{problems}. }}
\begin{document}

\begin{center}
{\it Комбинаторика и алгоритмы}
\vskip9pt\hrule\vskip14pt
\vfill
{\Large\bf Замощения доминошками}
\vskip20pt
{\large\bf Смирнов Евгений Юрьевич}

\vfill
Берендеевы поляны, 18--19 августа 2019 г.
\end{center}

\newpage

\centerline{\large\scshape Замощения}
Будем обозначать за $F_n$ {\it числа Фибоначчи} --- последовательность чисел со следующими свойствами: $F_0=F_1=1,F_{n+2}=F_{n+1}+F_n$.

\theorem Число способов разбить доску $n\times 2$ на доминошки равно $F_n$.

\proof Доказываем по индукции. Для $n=0,1$ верно.

{\bf Шаг индукции.} Пусть мы разбиваем доску $(n+2)\times2$. Посмотрим на последнюю её доминошку. Если она лежит вертикально, оставшуюся часть можно разбить на доминошки $F_{n+1}$ способами, а если стоит горизонтально --- $F_n$ по предположению индукции. Тогда всю доску можно разбить $F_{n+2}$ способами.\QEDA\\

\definitionn{$G$-числа Фибоначчи} Будем каждому разбиению доски писать в соответствие $x^m$, где $m$ --- количество вертикальных доминошек в разбиении. Просуммируем все такие выражения по всем разбиениям. Получим $F(n,x)$ --- $n$-е $G$-число Фибоначчи.

\definition Ацтекский Бриллиант --- структура, состоящая из горизонтальных отрезков длиной $2,4,\ldots,2n,2n,\ldots,4,2$, написанных друг под другом с общим центром.

\theorem Кол-во способов разрезать Бриллиант на доминошки равно $2^{\binom{n+1}2}$.\label{aztec}\\

\lemman{конденсация Доджсона}\label{condense} Пусть $F_S$ --- количество разбиений фигуры без точек множества $S$ на доминошки, $F:=F_\emptyset$. Точки $a,b,c,d$ в фигуре таковы, что любые пути по клеточкам между $a,c$ и между $b,d$ пересекаются. Тогда верно: \[
	F\cdot F_{abcd}=F_{ab}F_{cd}+F_{bc}F_{ad}%+F_{ac}F_{bd}
.\]

\proof Рассмотрим любую пару из разбиения фигуры и разбиения фигуры без точек $a,b,c,d$. Совместим эти разбиения (сделав их предварительно разных цветов). Тогда фигура разобьётся на <<змей>>. Змея, выходящая из $a$, приходит в одну из трёх вершин. Пусть она приходит в $b$. Возьмём разбиение исходной фигуры и сделаем из него два --- в одном перекрасим змею $ab$, в другом $cd$. Если сделать такую операцию для каждой пары разбиений, получится биекция между парами разбиений из левой части и парами разбиений из правой части.\QEDA\\

\prooft{aztec} Доказываем утверждение по индукции. Для $n=1,2,3$ верно.

{\bf Шаг индукции.} Докажем для $n+1$. Применим~\ref{condense} для точек на рисунке. Получим $A(n+1)A(n-1)=2A^2(n)$, откуда и следует утверждение задачи.\QEDA\\

\begin{tikzpicture}
	\draw[step=1,thin,black!20] (-3,-3) grid (3,3);
	\path[draw] (-3,0) -- (-3,1) -- (-2,1) -- (-2,2) -- (-1,2) -- (-1,3) -- (0,3) -- (1,3) -- (1,2) -- (2,2) -- (2,1) -- (3,1) -- (3,0);
	\path[draw] (-3,0) -- (-3,-1) -- (-2,-1) -- (-2,-2) -- (-1,-2) -- (-1,-3) -- (0,-3) -- (1,-3) -- (1,-2) -- (2,-2) -- (2,-1) -- (3,-1) -- (3,0);

	\drawsq{-1}{-1}{0}{-2}\node at (-0.5,-1.5) {$a$};
	\drawsq0211\node at (0.5,1.5) {$c$};
	\drawsq102{-1}\node at (1.5,-0.5) {$b$};
	\drawsq{-2}1{-1}0\node at (-1.5,0.5) {$d$};
\end{tikzpicture}


\newpage
\centerline{\large\scshape Треугольная сетка}
Рассмотрим центрально-симметричный шестиугольник со сторонами $p,q,r$ на треугольной сетке. Будем искать $P(p,q,r)$ --- количество способов его замостить. (например, из-за рисунка ниже $P(1,1,1)=2$)

Заметим, что существует биекция между трёхмерными диаграммами Юнга размера $(p,q,r)$ и разбиениями шестиугольников.

\lemma $P(0,q,r)=1$. Действительно, разбиение параллелограмма с углами $60^\circ,120^\circ$ на доминошки однозначно: все параллелограммы <<параллельны>> исходному.

\lemma $P(1,1,r)=r+1$. Действительно, это двумерные диаграммы Юнга, вписанные в прямоугольник $r\times 1$. Аналогично, $P(1,q,r)=\binom{q+r}q$.

\statement Утверждение \ref{condense} верно для треугольной решётки.\\

{\bf Поиск $P(p,q,r)$.} Применим \ref{condense} для точек на рисунке. Получим \[
	P(p,q,r)\cdot P(p,q-1,r-1)=P(p,q-1,r)\cdot P(p,q,r-1)+P(p-1,q,r)\cdot P(p+1,q-1,r-1)
.\] Сделаем индукцию по $p$ (будем находить функцию для всех $q,r$ одновременно). Получим: \[
	P(p+1,q,r)=\dfrac{P(p,q+1,r+1)P(p,q,r)-P(p,q,r+1)P(p,q+1,r)}{P(p-1,q+1,r+1)}
.\] Посчитаем $P(2,q,r)$: \begin{align*}
	P(2,q,r)=\binom{q+r+2}{q+1}\binom{q+r}q-\binom{q+r+1}q\binom{q+r+1}{q+1}=\\
	=\dfrac{(q+r+2)!(q+r)!}{(q+1)!(r+1)!q!r!}-\dfrac{(q+r+1)!^2}{q!(q+1)!r!(r+1)!}=\\
	=\dfrac{(q+r)!(q+r+1)!}{q!r!(q+1)!(r+1)!}=\dfrac{\binom{q+r}q\binom{q+r+2}{q+1}}{q+r+2}
\end{align*}

\theoremn{Макмагон} \[
	P(p,q,r)=\prod\limits_{i=1}^q\prod\limits_{j=1}^r\dfrac{p+i+j-1}{i+j-1}=\prod\limits_{i=1}^p\prod\limits_{j=1}^q\prod\limits_{k=1}^r\dfrac{i+j+k-1}{i+j+k-2}
.\]

{\bf Как запомнить формулу и что она означает.} Будем обозначать $h(\eta)$ (ф-ция от кубика в трёхмерной диаграмме Юнга) --- манхэттеновское расстояние до угла (от углового кубика $h(\eta)=1$). Тогда $P(p,q,r)=\prod\limits_\eta \frac{h(\eta)+1}{h(\eta)}$.

\newpage
\centerline{\large\scshape Фризы}
Будем нумеровать вершины треугольной сетки следующим образом. Занумеруем любую строку всеми единицами, следующую за ней --- любой периодической последовательностью натуральных чисел без двух единиц подряд (можно считать, что строка с единицами находится сверху другой занумерованной). Каждую следующую строку нумеруем так: пусть число сверху от вершины, которую мы нумеруем сейчас, $N$, справа сверху и слева сверху --- $W$ и $E$. Тогда напишем в текущую ячейку число $S$ так, что $WE-NS=1$. Если в какой-то момент получим ещё одну строку из всех единиц, заканчиваем процесс. Полученная структура называется {\it фризом}. Например, если взята последовательность с периодом $2,2,1,3$:

\begin{verbatim}
	1 1 1 1 1 1 1 1 1 1 1 1
	 2 2 1 3 2 2 1 3 2 2 1
	5 3 1 2 5 3 1 2 5 3 1 2
	 7 1 1 3 7 1 1 3 7 1 1
	4 2 0 1 4 2 0 1 4 2 0 1
\end{verbatim}
и т.п.\\

\definition Континуанты --- семейство многочленов $V_n(a_1,\ldots,a_n)$ таких, что $V_0=1,V_1=a_1,V_n=a_nV_{n-1}-V_{n-2}$.

\theoremn{правило Эйлера-Морзе} $V_n$ получается так. Возьмём граф с рёбрами $(a_1,a_2),(a_2,a_3),\ldots,(a_{n-1},a_n)$. Рассмотрим все его вполне несвязные подмножества (в каждой компоненте связности не больше 2 вершин) и просуммируем по ним $(-1)^\tau\cdot\prod_{a_i\not\in A} a_i$, где $A$ --- множество вершин с рёбрами и $|A|=2\tau$.\label{morze}

\proof Аналогично доказательству любой теоремы про Фибоначчи.\QEDA\\

\theorem Следующий ромб унимодулярен (т.е. $WE-NS=1$): \[
	N=V_{n-2}(a_2,\ldots,a_{n-1}),W=V_{n-1}(a_1,\ldots,a_{n-1}),E=V_{n-1}(a_2,\ldots,a_n),S=V_n(a_1,\ldots,a_n)
.\]\label{unimod}

\proof По \ref{morze} $W$ считает конфигурации отрезков без последней точки, а $E$ --- без первой. Возьмём какую-нибудь конфигурацию. Обозначим отрезки из $W$ и $S$ чёрным, а из $E$ и $N$ --- красным, и перекрасим крайнюю правую <<змею>>. Единственная конфигурация при нечётном $n$, которую нельзя перекрасить --- знакопеременная (одна длинная змея) с весом 1, а при чётном --- тоже знакопеременная, но теперь она не перекрашивается в обратную сторону и у неё вес -1.\QEDA\\

\z Доказать~\ref{unimod} по индукции.\\

\theorem Все числа фриза целые.

\proof Числа фриза удовлетворяют рекурренте континуант и их граничным условиям. Значит, в верхней вершине правильного треугольника, на верхней стороне которого написаны $a_1,\ldots,a_n$, написано $V_n(a_1,\ldots,a_n)$ --- целое число.\QEDA\\

\theorem Пусть фриз зациклился (т.е. все числа на уровне $n-1$ единицы для $n>1$; число $n$ называется его порядком). Тогда он обладает скользящей симметрией.

\proof Очевидно из унимодулярности.\QEDA\\

\theorem Пусть фриз зациклился. Тогда в его изначальном периоде есть единица.

\proof Пусть там нет единицы. Тогда для любого поворота верно: \[
	V_k=a_kV_{k-1}-V_{k-2}\geq 2V_{k-1}-V_{k-2}\implies V_k-V_{k-1}\geq V_{k-1}-V_{k-2}\geq\ldots\geq V_1-V_0>0
,\] т.е. числа фриза на диагонали не убывают --- противоречие.\QEDA\\

Фризы малых порядков: 1 --- порядка 3; 12 --- порядка 4; 13122 --- порядка 5.

\theoremn{Кокстер, Конвей, 1973} Фризов порядка $n$ ровно $C_n$.\label{cat}

\lemma У фриза с второй строкой $(a_1,\ldots,a_n)$ порядок $n$ тогда и только тогда, когда у фриза с второй строкой $(a_1,\ldots,a_{n-1}+1,1,a_n+1)$ порядок $n+1$.\label{splice}

\proof Пусть $w_i$ --- коэффициенты на диагонали нового фриза, а $v_i$ --- старого. Заметим, что $v_i=w_i\forall\ i<n-1,w_{n-1}=v_{n-1}-v_{n-2},w_n=v_{n-1},w_{n+1}=v_n$.\QEDA\\

\prooft{cat} Будем нумеровать триангуляцию многоугольника так. Берём все свободные вершины, пишем в них 1 и удаляем. Затем пишем в свободные вершины 2, удаляем и т.п. Затем берём последовательность вершин в любом порядке. Заметим, что операция из \ref{splice} делает из триангуляции $n$-угольника триангуляцию $(n+1)$-угольника, причём взаимно однозначно. Т.к. у этих последовательностей совпадают начала и рекуррента, они совпадают.\QEDA\\ 

\end{document}
