
\documentclass[a4paper,12pt]{article}
\usepackage[russian]{babel}
\usepackage[utf8]{inputenc}
\usepackage[a4paper, margin=2.5truecm, top=1.8truecm, bottom=1.3truecm]{geometry}
\usepackage{hyperref}
\usepackage{graphicx}
\usepackage{amsmath,amsfonts,amssymb}


\relpenalty=10000 \binoppenalty=10000
\newcounter{theorems}
\newcommand*{\lemma}{\refstepcounter{theorems}{\bf Лемма \arabic{theorems}. }}
\newcommand*{\theorem}{\refstepcounter{theorems}{\bf Теорема \arabic{theorems}. }}
\newcommand*{\lemman}[1]{\refstepcounter{theorems}{\bf Лемма \arabic{theorems} (#1). }}
\newcommand*{\theoremn}[1]{\refstepcounter{theorems}{\bf Теорема \arabic{theorems} (#1). }}

\newcounter{defs}
\newcommand*{\definition}{\refstepcounter{defs}{\bf Определение \arabic{defs}. }}

\newcommand*{\QEDA}{\hfill\ensuremath{\blacksquare}}

\newcommand{\proof}{{\bf Доказательство. }}
\newcommand{\prooft}[1]{{\bf Доказательство теоремы~\ref{#1}. }}
\newcommand{\prooftms}[1]{\newcounter{tproof#1}}
\newcommand{\prooftm}[1]{\refstepcounter{tproof#1}{\bf Доказательство теоремы~\ref{#1}~\#\arabic{tproof#1}. }}
\newcommand{\proofl}[1]{{\bf Доказательство леммы~\ref{#1}. }}
\newcommand{\prooflms}[1]{\newcounter{lproof#1}}
\newcommand{\prooflm}[1]{\refstepcounter{lproof#1}{\bf Доказательство леммы~\ref{#1}~\#\arabic{lproof#1}. }}

\newcounter{problems}
\newcommand{\zs}[1]{\refstepcounter{problems}{\bf Упражнение \arabic{problems} (#1 балла).}}
\begin{document}

\begin{center}
{\it Комбинаторика и алгоритмы}
\vskip9pt\hrule\vskip14pt
\vfill
{\Large\bf Первообразные корни и построение правильных многоугольников с помощью циркуля и линейки}
\vskip20pt
{\large\bf Савватеев Алексей Владимирович}

\vfill
Берендеевы поляны, 18--20 августа 2019 г.
\end{center}

\newpage
\centerline{\large\scshape Комплексные числа}
Напишем многочлен с корнями $1$,$2$,$-3$ (нам понадобится многочлен с другими корнями). Он равен \[
	P(x)=(x-1)(x-2)(x+3)=x^3-7x+6
.\]
Напишем для $P(x)$ формулу Кардано:
\begin{eqnarray*}
	x=\alpha+\beta\\
	(\alpha+\beta)^3-7(\alpha+\beta)+6=0\\
	\alpha^3+\beta^3+(\alpha+\beta)(3\alpha\beta-7)+6=0\\
	\begin{cases}
		3\alpha\beta=7\\
		\alpha^3+\beta^3=-6
	\end{cases}\\
	(y-\alpha^3)(y-\beta^3)=0\\
	y=-3\pm\sqrt{9-\frac{343}{27}}\\
	x=\sqrt[3]{-3-\frac{\sqrt{-100}}{27}}+\sqrt[3]{-3+\frac{\sqrt{-100}}{27}}\\
	x=\sqrt[3]{-3-\frac{10i}{3\sqrt3}}+\sqrt[3]{-3+\frac{10i}{3\sqrt3}}\\
\end{eqnarray*}
и при правильном извлечении кубических корней получим те же самые три корня (для каждого $\alpha^3$ будет однозначно определяться $\beta^3$).\\

\definition Комплексные числа $\mathbb C$ --- расширение $\mathbb R$, получаемое из него добавлением формального символа $i$ со свойством $i^2=-1$.

{\bf Свойства.} \begin{itemize}
	\item $\mathbb C$ --- поле ($(x+iy)+(a+ib)=(x+a)+i(y+b)$ и т.п.)
	\item Если нарисовать $\mathbb C$ на плоскости, ставя в соответствие точке $(x,y)$ число $x+iy$, сумма комплексных чисел будет суммой векторов.
	\item Поставим каждому числу $z\in\mathbb C$ в соответствие $\arg z$ --- угол между $Ox$ и лучом из $(0,0)$ в $z$ и $|z|$ --- длину вектора от $(0,0)$ в $z$. Тогда при умножении комплексных чисел аргументы чисел будут складываться, а модули умножаться. Т.е. при умножении всех точек плоскости на одно конкретное (ненулевое) число $z$ происходит поворотная гомотетия с коэффициентом $|z|$ и углом $\arg z$.
	\item Аналогично, при инверсии ($z\mapsto\frac1z$) будет так: $\arg\frac1z=-\arg z,|\frac1z|=\frac1{|z|}$.
	\item Поставим в соответствие числу $z=x+iy$ число $\overline z=x-iy$ --- сопряжённое число (геометрически это отражение относительно $Ox$). Тогда при арифметических операциях сопряжение сохраняется ($\overline z+\overline t=\overline{z+t}$ и т.п.) и $|z|^2=z\overline z$.
	\item Корень $n$-й степени извлекается из любого ненулевого комплексного числа ровно $n$ способами --- это вершины правильного $n$-угольника.
\end{itemize}

%\zs{1} Проверить ассоциативность умножения комплексных чисел ($x(yz)=(xy)z$).

%\zs{3} Разобраться, почему у последнего уравнения 3 корня.
\newpage
\centerline{\large\scshape Построение фигур}
\centerline{\scshape Треугольник $72^\circ,72^\circ,36^\circ$}
Заметим, что если у этого треугольника провести биссектрису большого угла, он разобьётся на 2 равнобедренных, поэтому если его малая сторона $1$, а большая $x$, то $x-1=\dfrac1x$, т.е. $x=\dfrac{1+\sqrt5}2$. Чтобы построить этот отрезок, построим прямоугольный треугольник со сторонами 1 и 2, тогда половина его гипотенузы будет $\dfrac{\sqrt5}2$.\QEDA\\

\centerline{\scshape Правильный пятиугольник}
Построим треугольник с углами $72^\circ,72^\circ,36^\circ$ и опишем вокруг него окружность, затем построим серединные перпендикуляры к его большим сторонам и пересечём с окружностью. Получатся 2 точки, которые вместе с треугольником образуют правильный пятиугольник.\QEDA\\

\centerline{\scshape Правильный пятиугольник II}
Будем строить решения уравнения $z^5=1$ в комплексных числах. Пусть его решения --- $1,\xi,\xi^2,\xi^{-2},\xi^{-1}$. Заметим, что по теореме Виета (или из-за поворотов) сумма решений равна 0, т.е. $(\xi+\xi^{-1})(\xi^2+\xi^{-2})=-1$. Пусть левая скобка $\alpha$, а правая --- $\beta$, тогда по теореме Виета $\alpha$ и $\beta$ --- корни уравнения $x^2+x-1=0$. Найдём $\alpha$ (положительный корень, равный $\frac{\sqrt5-1}2$), построим его на вещественной оси (пусть это точка $A$), тогда пересечение серединного перпендикуляра к $OA$ с единичной окружностью даст 2-ю и 3-ю вершины пятиугольника.\QEDA\\

\centerline{\scshape Правильный 17-угольник}
Пусть $\xi=\cos\frac{2\pi}{17} + i\sin\frac{2\pi}{17}$ --- <<первый>> корень 17-й степени из единицы. Будем строить 
$\xi$, или, что аналогично, $\xi+\xi^{-1}$. Аналогично предыдущему построению заметим, что $(\xi+\xi^{-1})(\sum\limits_{j=1}^8 \xi^j)=-1$. Разобьём степени $\xi$ на две группы: в $\alpha$ степени $\pm1,\pm2,\pm4,\pm8$, в $\beta$ все остальные.\\

%\zs2 Раскрыть скобки и доказать, что $\alpha\beta=c\cdot(\xi+\ldots+\xi^{-1})$, где $c$ целое.\\

Пусть мы это доказали. Тогда $\alpha$ и $\beta$ --- корни уравнения $x^2+x-4=0$, т.е. $(\alpha,\beta)=\dfrac{\pm\sqrt{17}-1}2$. Тогда очевидно, что \[
	\alpha=\dfrac{\sqrt{17}-1}2,\ \beta=\dfrac{-\sqrt{17}-1}2
.\]

%\zs2 Построить $\sqrt m$, если есть отрезки длиной $m$, $n$ и $1$. При тех же условиях построить $m+n$, $m-n$, $mn$, $\frac mn$.\label{postr}\\

Теперь разобьём каждую из $\alpha$ и $\beta$ на 2 группы: пусть в $\gamma_1$ степени $\pm1,\pm4$, в $\gamma_2$ --- $\pm2,\pm8$, в $\gamma_3$ --- $\pm3,\pm5$, в $\gamma_4$ --- $\pm6,\pm7$. Тогда окажется, что $\gamma_1\gamma_2=\gamma_3\gamma_4=-1$ и по теореме Виета (и \ref{postr}) все $\gamma_i$ построимы (большие корни --- соответственно, $\gamma_1$ и $\gamma_3$. Наконец, разобьём $\gamma_1$ на две группы --- в $\delta_1$ степени $\pm1$, а в $\delta_2$ --- $\pm4$. Тогда $\delta_1+\delta_2=\gamma_1,\delta_1\delta_2=\gamma_3$. По теореме Виета строим $\delta_1$ (больший корень уравнения $x^2-\gamma_1x+\gamma_3=0$), строим точку $A$ с этой координатой на вещественной оси, тогда серединный перпендикуляр к отрезку $OA$ даст две вершины 17-угольника.\QEDA\\

\newpage
В дальнейших леммах имеются в виду правильные многоугольники.\\

\lemma Если $m$-угольник построим, то и $2m$-угольник тоже.

\lemma Если $mn$-угольник построим, то $m$- и $n$-угольники тоже.

Обе леммы очевидны.\\

\lemma Если $(m,n)=1$, то существуют $k,l\in\mathbb Z$ такие, что $km+ln=1$.\label{euclid}

\proof Пусть $d>0$ --- минимальное представимое в виде $km+ln=d$. Заметим, что $m\vdots d$. Действительно, следующее после $d$ число, которое можно получить --- это $2d$, затем $3d$ и т.п., т.к. иначе можно было бы получить число, меньшее $d$. Аналогично $n\vdots d$. Тогда $(m,n)\geq d$, значит, $d=1$.\QEDA\\

\lemma Если $m$- и $n$-угольники построимы и $(m,n)=1$, то $mn$-угольник тоже.

\proof По \ref{euclid} существуют такие $k$ и $l$, что $km-ln=1$. Значит, если построить эти многоугольники на одной окружности с общей вершиной, какие-то две вершины будут на расстоянии $\frac1{mn}$ от длины окружности.\QEDA\\

\newpage
\centerline{\large\scshape Школьная теория полей}
\definition Абелева Группа --- множество $(G,\oplus,\ominus)$ с определёнными на нём операциями $\oplus$ и $\ominus$ со следующими свойствами:
\begin{itemize}
	\item $\forall a,b\in G\ a\oplus b,\ominus a\in G$;
	\item $\forall a,b\in G\ a\oplus b=b\oplus a$ (коммутативность);
	\item $\forall a,b,c\in G\ a\oplus(b\oplus c)=(a\oplus b)\oplus c$ (ассоциативность);
	\item $\exists e\in G\forall a\in G\ a\oplus e=a$;
	\item $\forall a\in G\ a\oplus(\ominus a)=e$ (обратимость).
\end{itemize}

\definition Поле --- множество $\mathbb F$ с определёнными на нём операциями $+,-,\cdot,/$ со следующими свойствами:
\begin{itemize}
	\item $(\mathbb F,+,x\mapsto -x)$ --- абелева группа;
	\item $(\mathbb F/0,\cdot,x\mapsto 1/x)$ --- абелева группа;
	\item Выполняется дистрибутивность ($\forall a,b,c\in \mathbb F\ a\cdot(b+c)=a\cdot b+a\cdot c$).
\end{itemize}

%\zs2 Никакое множество из более чем одного элемента не может быть дистрибутивной Абелевой группой по сложению и умножению одновременно.

%\zs2 Остатки по модулю $p\in\mathbb P$ можно делить.\label{fld}

\newpage
\centerline{\large\scshape Первообразные}
\definition Первообразный корень по модулю $p$ --- такой остаток $q$, что все ненулевые остатки по модулю $p$ являются степенями $q$ в каком-то порядке.

\theorem Первообразный корень существует для всех простых $p$.\label{trivial-root}\\

Обозначим $aS=\{ab|b\in S\}$.

\lemma Если $S\subset\mathbb F_p$, то $|aS|=|S|$. В частности, $a\mathbb F^*_p=\mathbb F^*_p$.

\proof Вытекает из \ref{fld}.\QEDA\\

\theoremn{малая теорема Ферма} Если $n\not\vdots p$, то $n^{p-1}\equiv1\mod p$.\label{ferma}

\proof Заметим, что $(p-1)!\equiv n\cdot(2n)\cdot\ldots\cdot(p-1)n=(p-1)!n^{p-1}\mod p$, откуда следует утверждение задачи.\QEDA\\

\definition Порядок $a$ по модулю $p$ (обозначается $ord_p(a)$) --- минимальное такое натуральное $k$, что $a^k\equiv 1\mod p$.

\lemma $ord_p(a)|p-1$.

\proof Заметим, что все числа разбиваются на множества вида $a_1^n,a_2^n,\ldots$. Пусть $ord_p(a_1)=l$. Тогда до тех пор, пока есть не рассмотренные числа, будем брать одно из них ($b$) и добавлять к рассмотренным числа вида $b\cdot a^n$. Тогда на каждом шаге все рассматриваемые числа различные и добавляется $l$ чисел за шаг, откуда всё следует.\QEDA\\

\theoremn{Безу} Пусть $P(x)$ --- многочлен с коэффициентами из кольца. Тогда если $\alpha$ --- корень $P$, то $P\vdots(x-\alpha)$. (упражнение на 1 балл)

\lemma Пусть $P(x)$ --- многочлен с коэффициентами из поля. Тогда если $\alpha_1,\ldots,\alpha_n$ --- корни $P$, то $P\vdots(x-\alpha_1)\ldots(x-\alpha_n)$. (упражнение на 1 балл)

\lemma У многочлена степени $n$ с коэффициентами из поля не более $n$ корней. (упражнение на 1 балл)\label{bezu2}\\

\lemman{Гаусс} Пусть $S$ --- множество остатков от 1 до $\frac{p-1}2$. Тогда $|\{n,-n\}\cap aS|=1$ для всех $n$ и $a\not\equiv 0$ и, кроме того, $a^{\frac{p-1}2}=(-1)^m$, где $m$ --- количество таких $n\leq\frac{p-1}2$, что в $aS$ есть $-n$.\label{gauss}

\proof Первая часть очевидна из принципа Дирихле. Вторая часть следует из того, что $(\frac{p-1}2)!\equiv (-1)^m\cdot a(2a)\ldots(\frac{p-1}2a)$.\QEDA\\

% TODO: вывести закон квадратичной взаимности Гаусса из леммы Гаусса

\definition Функция Эйлера $\varphi(d)$ --- количество натуральных чисел, меньших $s$ и взаимно простых с ним

\lemma $d=\sum\limits_{l|d}\varphi(l)$ для всех $d$.\label{eulerf}

\proof Напишем дроби $\frac1d,\frac2d,\ldots,\frac nd$ и сократим их. Заметим, что дробей с знаменателем $l$ ровно $\varphi(l)$, откуда всё следует.\QEDA\\

\prooft{trivial-root} Пусть $d|n$. Тогда $x^n-1=(x^d-1)\Psi(x)$, где у $\Psi$ степень $n-d$. Значит, по~\ref{bezu2} есть ровно $d$ решений уравнения $x^d-1=0$, т.е. существует ровно $d$ чисел с порядком, делящим $d$. Обозначим $\psi(d)=|\{a|ord_p(a)=d\}|$. Тогда по \ref{eulerf} $d=\sum\limits_{l|d}\psi(l)\ \forall d|n$. Значит, $\psi\equiv\varphi$, в частности, $\psi(p-1)=\varphi(p-1)>0$.\QEDA\\

%\zs2 Выведите из формул, что $\psi\equiv\varphi$.

\newpage
\centerline{\large\scshape Финал}

\lemma Если $2^k+1\in\mathbb P$, то $k=2^l$ для натурального $n$. (такие числа называются числами Ферма)

\lemma Если для $p$ такого вида $a$ --- не первообразный, то $a^{\frac{p-1}2}\equiv 1\mod p$.

\proof По \ref{ferma} это либо 1, либо -1, и -1 не подходит.\QEDA\\

{\bf Утверждение.} Число построимо тогда и только тогда, когда существует список квадратных уравнений, последнее из которых имеет это число решением.

\theorem Если $p>3\in\mathbb P$ таково, что правильный $p$-угольник построим, то $3$ --- первообразный корень по модулю $p$.\label{finale}

\lemma Простые числа Ферма, большие 3, сравнимы с 5 по модулю 12.

\prooftms{finale}\prooftm{finale} Докажем по индукции, что для чисел, сравнимых с 5 по модулю 12,~\ref{gauss} даёт нечётное количество минусов. База для 17 очевидна, шаг очевиден.\QEDA\\

\prooftm{finale} Заметим, что такие $p$ имеют вид $2^{2^n}+1$. Применим закон квадратичной взаимности Гаусса: $\left(\frac 3p\right)\left(\frac p3\right)=(-1)^{\frac{p-1}2}=1$. Заметим, что $\left(\frac p3\right)=\left(\frac{-1}p\right)=-1$, значит, и $\left(\frac 3p\right)=-1$. С другой стороны, $\text{ord }n\mod p=2^k$, а значит, и $\frac{p-1}{\text{ord }n\mod p}=2^k$, но мы доказали, что это нечётное число, значит, это 1.\QEDA\\

{\bf Идея построения.} Вначале разбиваем сумму всех $\xi$ на слагаемые вида $\xi^{3^{2^k(m+1)}}$ и на $\xi^{3^{2^{k+1}}}$, где $k$ --- номер шага (например, вначале разбиваем на чётные и нечётные степени тройки). Тогда после каждого шага каждая сумма строится.

%\zs5 Доказать, что после каждого шага алгоритма выше получается целочисленный множитель одной из предыдущих сумм.

%\zs2 Доказать, что $n^{\frac{p-1}2}\equiv1\mod p$ тогда и только тогда, когда $n$ --- квадратичный вычет (т.е. $\exists b\in\mathbb F_p:b^2\equiv a\mod p$).\\

%\vfill

%\centerline{\large\bf Зачёт: 10 баллов}

\end{document}
