
\documentclass[a4paper,12pt]{article}
\usepackage[russian]{babel}
\usepackage[utf8]{inputenc}
\usepackage[a4paper, margin=2.5truecm, top=1.8truecm, bottom=1.3truecm]{geometry}
\usepackage{hyperref}
\usepackage{graphicx}
\usepackage{amsmath,amsfonts,amssymb}


\relpenalty=10000 \binoppenalty=10000
\newcounter{theorems}
\newcommand*{\lemma}{\refstepcounter{theorems}{\bf Лемма \arabic{theorems}. }}
\newcommand*{\theorem}{\refstepcounter{theorems}{\bf Теорема \arabic{theorems}. }}
\newcommand*{\lemman}[1]{\refstepcounter{theorems}{\bf Лемма \arabic{theorems} (#1). }}
\newcommand*{\theoremn}[1]{\refstepcounter{theorems}{\bf Теорема \arabic{theorems} (#1). }}

\newcounter{defs}
\newcommand*{\definition}{\refstepcounter{defs}{\bf Определение \arabic{defs}. }}

\newcommand*{\QEDA}{\hfill\ensuremath{\blacksquare}}

\newcommand{\proof}{{\bf Доказательство. }}
\begin{document}

\begin{center}
{\it Комбинаторика и алгоритмы}
\vskip9pt\hrule\vskip14pt
\vfill
{\Large\bf Теорема Брукса}
\vskip20pt
{\large\bf Гусев Антон Сергеевич}

\vfill
Берендеевы поляны, 15--17 августа 2019 г.
\end{center}

\newpage

\centerline{\Large Раскраски графов}
Будем красить вершины так, что любые две соседние вершины имеют разные цвета (такие раскраски называются {\it правильными\/}).

\definition Хроматическое число графа $G$ --- минимальное такое $n=\chi(G)\in\mathbb N$, что существует правильная раскраска $G$ в $n$ цветов.

\definition Число независимости графа $G\ (\alpha(G))$ --- размер максимального независимого множества вершин (внутри которого нет рёбер).\\

\theorem Пусть $G=(V,E)$. Тогда $\chi(G)\geq\dfrac{|V|}{\alpha(G)}$.\\\label{indep}

\proof Каждый цвет в раскраске --- независимое множество, поэтому кол-во вершин каждого цвета не больше $\alpha(G)$, откуда и следует утверждение.\QEDA\\

\theoremn{Брукс, 1941} Пусть $G=(V,E)$ связен и $\forall v\in V\ \deg(v)\leq n$, кроме того, $G$ --- не нечётный цикл и не полный граф. Тогда $\chi(G)\leq n$.

\proof Разберём несколько случаев:
\begin{enumerate}
	\item Пусть в графе есть вершина степени меньше $n$. Тогда можно её удалить и в оставшемся графе есть вершина степени меньше $n$. Покрасим граф по индукции, затем покрасим удалённую вершину в один из цветов, не использованных в её соседях.\label{simple}
	\item Пусть в графе есть вершина-мост (при удалении которой граф теряет связность). Удалим её, покрасим компоненты вместе с мостом (по~\ref{simple} --- в компонентах у вершины-моста степень меньше $n$) так, что у удалённой вершины в раскрасках один и тот же цвет. Тогда получится правильная раскраска.\label{singlebridge}
	\item Пусть в графе есть мост из двух вершин.\label{doublebridge}
	\begin{enumerate}
		\item Вершины моста соединены ребром. Тогда сделаем то же самое, что и в~\ref{singlebridge}.
		\item Вершины моста не соединены ребром. Сделаем то же самое, что и в~\ref{singlebridge}. Тогда в раскрасках они либо одного цвета, либо разных. Если они одновременно одного цвета или одновременно разных цветов, то задача решена. Иначе рассмотрим ту компоненту, где они одного цвета. Если в ней есть вершина с двумя или более рёбрами в другую компоненту, уберём её, покрасим оставшуюся часть компоненты, затем покрасим удалённую вершину в цвет, не совпадающий с цветом другой вершины моста. Иначе обе вершины соединены с одной вершиной извне, и можно доказать, что одна из этих <<пограничных>> вершин соединена с одной вершиной извне. Тогда мы нашли вершину степени 2, следовательно, у всех вершин степень 2, значит, это нечётный цикл --- противоречие.
	\end{enumerate}
	В обоих случаях мы красим граф в $n$ цветов или попадаем в противоречие.\QEDA
	\item Есть три вершины $v,u,w$, такие, что $v\sim u,u\sim w,v\not\sim w$, и при удалении пары $(v,w)$ граф не теряет связность. Тогда подвесим граф на вершину $u$, затем поставим $v$ и $w$ на уровень $-1$ и покрасим их в первый цвет. Идём снизу вверх по уровням графа. Заметим, что любую вершину на текущем нижнем уровне можно покрасить. Действительно, она соединена с $n$ вершинами, но (хотя бы) одна из этих вершин находится на уровень выше и не даёт запретов. Красим так все уровни от последнего до первого. На нулевом уровне есть только вершина $u$, у неё $n$ соседей, но хотя бы у двух --- $v$ и $w$ --- цвет совпадает, поэтому есть не запрещённый цвет.\label{galka}\QEDA
\end{enumerate}
Доказательство заканчивается вот так. Рассмотрим какие-то две вершины, не соединённые ребром, и рассмотрим между ними кратчайший путь. Тогда первые три вершины имеют вид $u,v,w$ из~\ref{galka}. Тогда если при удалении двух из них граф не теряет связность --- случай~\ref{galka}, иначе~\ref{doublebridge}.\QEDA\\

\lemma У каждого члена парламента не больше семи врагов.\footnote{Лемма обобщается до $mn-1$ врагов. Тогда, перемещая члена с $n$ врагами в палату с минимальным количеством врагов, можно разбить парламент на $m$ палат так, что в каждой палате у каждого её члена меньше $n$ врагов.}  Тогда их можно разбить на две палаты, что в каждой палате у каждого её члена не более трёх врагов.\label{parlam}

\proof Разобьём парламент на две палаты как угодно, затем за 1 шаг будем перемещать члена с 4 или более врагами в другую палату. Тогда после шага общее количество пар врагов в палатах уменьшается, и когда оно уменьшится до минимума, условие будет выполнено.\QEDA\\

\lemma Пусть в графе нет $K_4$ и $\deg(v)\leq7$. Тогда $\chi(G)\leq6$.

\proof По~\ref{parlam} разобьём вершины на 2 компоненты. В каждой из них нет $K_4$, поэтому работает теорема Брукса для $n=3$ и каждая из них красится не более в три цвета. Поэтому весь граф красится не более чем в шесть цветов.\QEDA\\

\theoremn{Брукс-Super} Пусть в $G$ нет $K_3$. Тогда $\chi(G)\leq\dfrac{9n}{\log_2n}$ для $n>n_0$. \label{super}

\lemma Пусть в $G$ нет $K_3$ и выполняется $n=2^k-1$. Тогда $\chi(G)<\frac34n+1$.

\proof Применим~\ref{parlam} $k-2$ раз. Это разобьёт граф на $2^{k-2}$ компонент так, что в каждой компоненте выполняется $\deg(v)\leq3$. Тогда для этой компоненты можно применить теорему Брукса для $n=3$ --- каждая из компонент красится не более чем в три цвета, значит, весь граф --- не более чем в $3\cdot2^{k-2}<\frac34\cdot n+1$.\QEDA\\

\theorem В графе без $K_3$ хроматическое число может быть сколь угодно большим.\label{erdesht}

\proof Доказываем по индукции, что для любого $n$ существует граф $G_n$ без $K_3$ такой, что $\chi(G_n)=n$. База для $n=1$ очевидна. Пусть у нас есть $G_n=\{\{v_1,v_2,\ldots,v_m\},E\}$. Будем строить $G_{n+1}$ так. Добавим к $G$ вершины $u_1,u_2,\ldots,u_m,w$. Для каждой пары $(v_i,v_j)\in E$ добавим рёбра $(u_i,v_j)$ и $(v_i,u_j)$. Наконец, добавим все рёбра вида $(w,u_i)$. Тогда:
\begin{enumerate}
	\item Пусть вершины $(x,y,z)$ образуют треугольник. Тогда если $w\in(x,y,z)$, то остальные 2 вершины в $u_i$, что невозможно. Иначе одна из вершин в $u_i$, а остальные две --- в $v_i$, что противоречит предположению индукции. {\bf Следовательно, в $G_{n+1}$ треугольников нет}.
	\item $G_{n+1}$ красится в $n+1$ цвет. Действительно, можно покрасить $w$ в цвет $n+1$, а все $u_i$ --- в цвета соответствующих $v_i$.
	\item $\chi(G_{n+1})=n+1$. Действительно, пусть он красится в $n$ цветов. Можно считать, что $w$ покрашена в первый цвет. Тогда $u_i$ не покрашены в первый цвет. Пусть какая-то $v_i$ (для конкретного $i$) покрашена в первый цвет. Тогда перекрасим её в цвет соответствующей $u_i$. Докажем, что раскраска всё ещё правильная. Действительно, теперь возможные плохие рёбра имеют вид $(v_i,v_j)$. Тогда $(u_i,v_j)$ тоже плохое --- противоречие. Следовательно, можно избавиться в $G_n$ от первого цвета, т.е. этот граф красится правильным образом в $n-1$ цвет, что противоречит предположению индукции. \QEDA
\end{enumerate}

\newpage

\lemma Пусть в графе нет $K_3$ и $\deg(v)\leq7$. Тогда $\chi(G)\leq4$.\label{halving}

\theoremn{Зачёт} В условиях~\ref{super} асимптотически верно $\chi(G)\leq\frac n2+k$.

\proof Применим~\ref{parlam} для $m=\lceil\frac{n+1}8\rceil,n=8$. Тогда по~\ref{halving} граф разобьётся на $\lceil\frac{n+1}8\rceil$ компонент с хроматическим числом не более 4. Тогда $\chi(G)\leq4\lceil\frac{n+1}8\rceil$, что асимптотически равно $\frac n2$.\QEDA\\

\theoremn{Гид, 1999} Если в графе $G$ нет $K_n$ и $\deg(v)\leq n$, то $\chi(G)\leq n-1$. {\it Доказательство не рассказал, т.к. нужен вероятностный метод.}

\definition Охват графа $o(G)$ --- длина минимального цикла графа. Охват леса по определению равен $+\infty$. 

\theoremn{Эрдёш, 1963} $\forall k,d\ \exists G: o(G)\geq d,\chi(G)\geq k$. \footnote{\ref{erdesht} --- частный случай теоремы для $d=4$.}

\proof Зафиксируем $n$ --- число вершин графа, $\delta n$ --- число рёбер графа. Будем считать матожидание величины $X_l$ --- количества циклов длины $l$. Заметим, что количество способов выбрать такой цикл в графе равно $\binom nl\cdot\frac{(l-1)!}2$, следовательно, суммарное количество циклов в этих графах --- $\binom nl\cdot\frac{(l-1)!}2\cdot\binom{m-l}{\delta n-l}$, где $m=\binom n2$. Тогда $\mathbb EX_l=\dfrac{\binom nl\cdot\frac{(l-1)!}2\cdot\binom{m-l}{\delta n-l}}{\binom m{\delta n}}$.

\[
	\binom nl\cdot\frac{(l-1)!}2\leq\frac{n^l}{2l}\text{ и }\dfrac{\binom{m-l}{\delta n-l}}{\binom m{\delta n}}=\dfrac{(\delta n)!(m-l)!}{(\delta n-l)!m!}=\dfrac{\delta n(\delta n-1)\ldots(\delta n-l+1)}{m(m-1)\ldots(m-l+1)}<\left(\dfrac{\delta n}m\right)^l
.\] Тогда $\mathbb EX_l<\left(\dfrac{\delta n}m\right)^l\cdot\dfrac{n^l}{2l}=\dfrac{\delta^l}{2l}\cdot\left(\dfrac{n^2}m\right)^l = \left(1+\dfrac1{n-1}\right)^l\cdot\dfrac{2^l\cdot\delta^l}{2l}\leq \dfrac{2^l\delta^l}3\leq \dfrac{2^l\cdot\delta^l}3$, где предпоследнее неравенство выполняется при достаточно больших $n$ (т.к. $l<g$). Далее:

\[
	\mathbb EX:=\sum\limits_{i<g}\mathbb EX_i<\dfrac13\cdot \sum\limits_{l=3}^{g-1}(2\delta)^l\leq\dfrac13\cdot\dfrac{(2\delta)^g-1}{2\delta-1}\leq \dfrac{(2\delta)^g}{6\delta-3}
.\] Тогда можно взять очень большое $\delta$ так, что $\mathbb EX<\frac n6$. Тогда больше чем в половине графов число <<маленьких>> циклов не больше, чем $\frac n3$.\\

Теперь посчитаем матожидание $Y_p$ --- количество независимых множеств (антиклик) размера $p$. Их суммарное количество в графах --- $\binom np\cdot\binom{m-t}{\delta n}$, где $t=\binom p2$. Тогда $\mathbb EX_p=\dfrac{\binom np\cdot\binom{m-t}{\delta n}}{\binom m{\delta n}}<2^n\cdot\dfrac{\binom{\delta n}{m-t}}{\binom m{\delta n}}< 2^n\cdot\left(\dfrac{m-t}m\right)^{\delta n}=\left(2\left(\dfrac{m-t}m\right)^\delta\right)^n$. Далее, т.к. $p=\frac nc$:
\[
	\dfrac tm=\dfrac{p(p-1)}{n(n-1)}<\dfrac1{c^2}
.\] Тогда (в предположении, что $c$ фиксировано и $n$ очень большое) \[
	\mathbb EX_p=\left(2\left(1-\dfrac tm\right)^\delta\right)^n<\left(2\left(1-\dfrac1{c^2}\right)^\delta\right)^n<\dfrac12
.\] Таким образом, больше чем в половине графов нет антиклик размера $\frac nc$. Значит, существует граф, в котором оба этих свойства выполнены --- <<маленьких>> циклов не больше $\frac n3$ и число независимости не больше $\frac nc$. Удалим из каждого цикла по вершине. Тогда останется не меньше $\frac{2n}3$ вершин, значит, по~\ref{indep} верно $\chi(G)\geq\frac{2c}3$. Подставим $c=\frac{3k}2$ и получим решение задачи.\QEDA

\end{document}
