
\documentclass[a4paper,12pt]{article}
\usepackage[russian]{babel}
\usepackage[utf8]{inputenc}
\usepackage[a4paper, margin=2.5truecm, top=1.8truecm, bottom=1.3truecm]{geometry}
\usepackage{hyperref}
\usepackage{graphicx}
\usepackage{amsmath,amsfonts,amssymb}


\relpenalty=10000 \binoppenalty=10000
\newcounter{theorems}
\newcommand*{\lemma}{\refstepcounter{theorems}{\bf Лемма \arabic{theorems}. }}
\newcommand*{\theorem}{\refstepcounter{theorems}{\bf Теорема \arabic{theorems}. }}
\newcommand*{\lemman}[1]{\refstepcounter{theorems}{\bf Лемма \arabic{theorems} (#1). }}
\newcommand*{\theoremn}[1]{\refstepcounter{theorems}{\bf Теорема \arabic{theorems} (#1). }}

\newcounter{defs}
\newcommand*{\definition}{\refstepcounter{defs}{\bf Определение \arabic{defs}. }}

\newcommand*{\QEDA}{\hfill\ensuremath{\blacksquare}}

\newcommand{\proof}{{\bf Доказательство. }}
\newcommand{\prooft}[1]{{\bf Доказательство теоремы~\ref{#1}. }}
\newcommand{\prooftms}[1]{\newcounter{tproof#1}}
\newcommand{\prooftm}[1]{\refstepcounter{tproof#1}{\bf Доказательство теоремы~\ref{#1}~\#\arabic{tproof#1}. }}
\newcommand{\proofl}[1]{{\bf Доказательство леммы~\ref{#1}. }}
\newcommand{\prooflms}[1]{\newcounter{lproof#1}}
\newcommand{\prooflm}[1]{\refstepcounter{lproof#1}{\bf Доказательство леммы~\ref{#1}~\#\arabic{lproof#1}. }}
\newcommand{\prooftf}[2]{{\bf Доказательство теоремы~\ref{#1} #2.}}
\begin{document}

\begin{center}
{\it Комбинаторика и алгоритмы}
\vskip9pt\hrule\vskip14pt
\vfill
{\Large\bf Теорема о бутерброде и её применения в комбинаторике}
\vskip20pt
{\large\bf Григорьев Михаил Александрович}

\vfill
Берендеевы поляны, 22--25 августа 2019 г.
\end{center}

\newpage

\definition Множество точек называется выпуклым, если для любых точек $X,Y$, лежащих в множестве, в нём также лежат все остальные точки отрезка $XY$.

\theoremn{Хелли} Пусть дан набор множеств $A_1,\ldots,A_n\subset\mathbb R^d$, любые $d+1$ из которых пересекаются. Тогда и все множества пересекаются.\label{helly}

\theorem Утверждение \ref{helly} верно для бесконечного набора множеств, если они ограничены и замкнуты.\\

Будем рассматривать {\it хорошие меры} --- счётно-аддитивные функции $\mu$ из $\sigma\subset2^{\mathbb R^d}$ в $\mathbb R$, которые дают конечное положительное число на всём пространстве, непрерывны и обнуляются на гиперплоскостях. Например, можно рассматривать $f_T(X)=S(T\cap X)$ или $f_p(X)=\int\limits_X p(x,y)dxdy$.

\theoremn{Минковский, Радон} Для любого множества и любой хорошей меры существует точка $p$ такая, что любая гиперплоскость, проходящая через $p$, делит множество в отношении не более чем $d:1$ (имеется в виду отношение мер частей).\label{limp}

\proof Пусть $\mu(\mathbb R^d)=1$. Рассмотрим каждую <<направленную>> гиперплоскость. Для неё есть положение, отсекающее <<слева>> от неё множество меры $\frac d{d+1}$. Рассмотрим все полупространства такого вида. Заметим, что любые $d+1$ из них имеют общую точку. Действительно, пусть нет. Тогда их дополнения (вернее, их части с положительной мерой) не пересекаются. Посмотрим на полупространства меры $\frac1{d+1}+\varepsilon$ и устремим $\varepsilon$ к нулю сверху, получим противоречие. Таким образом, любые $d+1$ множества пересекаются, значит, по \ref{helly} они все пересекаются.\QEDA\\

\theoremn{о бутерброде} В $d$-мерном пространстве даны $d$ хороших мер. Тогда существует гиперплоскость такая, что по обе стороны от неё все меры совпадают.\label{maint}

\lemma Пусть есть $n$ точек общего положения в $d$-мерном пространстве. Тогда можно взять такой радиус $R$, что если взять шары радиуса $R$ вокруг каждой точки, что никакая гиперплоскость не пересекает $d+1$ шар.\label{amount}

\proof Допустим, для любого радиуса можно пересечь не менее $d+1$ шара. Обозначим за $l_\varepsilon$ --- гиперплоскость такую, что если у всех шаров радиус $\varepsilon$, то шары $B_{i,\varepsilon}$ для $i=1,\ldots,m$ пересекаются этой гиперплоскостью. Заметим, что всего количество наборов по $m$ шаров конечно, значит, в последовательности $B_{i,\varepsilon\cdot2^{-k}}|k\in\mathbb N$ какое-то множество шаров встречается бесконечное количество раз. Тогда можно сделать предельный переход по этой подпоследовательности и получить, что центры этих шаров лежат в одной гиперплоскости --- противоречие.\QEDA\\

\theoremn{о дискретном бутерброде} В $d$-мерном пространстве дано множество точек общего положения, покрашенное в $d$ цветов. Тогда существует гиперплоскость такая, что по обе стороны от неё количества точек каждого цвета совпадают, а на плоскости лежит не более 1 точки каждого цвета.\label{discrete}

\proof Пусть все количества точек нечётны (если чётны, можно убрать одну из точек, сделать то же самое, затем вернуть точку и немного подвигать искомую гиперплоскость). Сделаем такую меру для каждого цвета: возьмём шар достаточно малого радиуса вокруг каждой точки, и пусть мера от шара равна 1, если точка и мера одного цвета. Применим~\ref{maint}. Шары каждого цвета разделятся на 2 группы. Заметим, что:
\begin{itemize}
	\item Для каждого цвета есть шар этого цвета, который пересекает гиперплоскость из Теоремы.
	\item Всего по~\ref{amount} пересечено не более $d$ шаров.
\end{itemize}
Тогда для каждого цвета пересечён ровно один шар этого цвета, а т.к. все шары пересекаются гиперплоскостью в центре, утверждение задачи выполнено.\QEDA\\

\theorem Есть 100 коробок, в каждой из которых лежат яблоки, апельсины и бананы. Тогда можно выбрать 51 из них, что в них не меньше половины яблок, не меньше половины апельсинов и не меньше половины бананов.

\proof Расставим любые 100 точек общего положения в пространстве. Возьмём достаточно маленький шар вокруг каждого из них (с радиусом из~\ref{amount}). Поставим для каждого из них три меры --- $\mu_A(i)$ --- вес яблок в $i$-й коробке, $\mu_B$ --- вес бананов, $\mu_O$ --- вес апельсинов. Применим теорему о бутерброде. Тогда шары разобьются на 3 группы. В одной из них не более 3 шаров. Пусть там $n$. Тогда взяв из <<меньшей>> (по количеству шаров) группы все шары + эти $n$, можно добиться такого ответа: $n+\left[\frac{100-n}2\right]$, и при $n\leq 3$ это число не больше 51.\QEDA\\

\theoremn{Алон, Акияма} В $d$-мерном пространстве есть $dn$ точек общего положения, покрашенных в $d$ цветов, причём каждого цвета поровну. Тогда их можно разбить на группы так, что в каждой группе ровно одна точка каждого цвета и выпуклые оболочки групп не пересекаются.

\proof Доказываем по индукции. Для $n=1$ это очевидно.

{\bf Шаг индукции.} Применим~\ref{discrete}. Тогда на гиперплоскости пересечения либо нет точек, либо по одной каждого цвета. Возьмём все точки на гиперплоскости пересечения и объединим их в одну группу, затем применим предположение индукции для двух других групп.\QEDA\\

\definition Триангуляцией называется разбиение многоугольника на треугольники такое, что любая пара треугольников либо не пересекается, либо пересекается по стороне, либо пересекается по вершине.

\lemman{Шпернер} Триангуляция $n$-мерного симплекса покрашена в $n+1$ цвет --- $a_1,\ldots,a_{n+1}$ --- так, что на грани $a_i$ нет вершин цвета $a_i$. Тогда существует $n+1$-цветный симплекс триангуляции.\label{sperner}

\proof Доказываем по индукции. Для $n=1$ это очевидно.

{\bf Шаг индукции.} Назовём грань {\it дверью}, если его концы покрашены в цвета $a_i$ для всех $i<n+1$. Заметим, что в симплексе нечётное число дверей тогда и только тогда, когда либо симплекс граничит с исходным, либо он разноцветный. Посчитаем количество пар (дверь, $\triangle$). С одной стороны, их нечётно, т.к. на границе исходного симплекса нечётное число дверей. С другой стороны, чётность их количества совпадает с чётностью количества разноцветных симплексов. Следовательно, разноцветных симплексов нечётно.\QEDA\\

\theoremn{Брауэр} Пусть отображение $f$ из компакта в $\mathbb R^2$ в себя непрерывно. Тогда у него есть неподвижная точка. (определение компакта ниже)

\proof Доказываем для треугольника. Пусть это правильный треугольник с вершинами $(1,0,0),(0,1,0),(0,0,1)$. Будем триангулировать треугольник с целью применить предельный переход в~\ref{sperner}. Пусть вершина триангуляции $(x_1,x_2,x_3)$ перешла в $(f_1,f_2,f_3)$. Заметим, что $\sum f_i=\sum x_i$, значит, для какого-то $i$ верно $x_i\geq f_i$. Покрасим вершину в такой цвет (если есть несколько вариантов, выберем почти любой --- при перекрашивании вершины, например, $(1,0,0)$ выберем цвет 1, а при перекрашивании $(0,y,z)$ выберем любой другой цвет). Будем триангулировать так: делим каждую сторону на $m$ частей и проводим прямые, параллельные сторонам, и устремляем $m$ к бесконечности. После предельного перехода через \ref{compact} получим $x_i\geq f_i$ для трёхцветной <<точки>>, т.е. $x=f(x)$.{\it Замечание: то же доказательство работает для симплекса в  $n$-мерном пространстве.}\QEDA

\newpage

\definition Компакт --- такое множество, что для любой последовательности с элементами из него есть сходящаяся подпоследовательность.

\theorem Куб $[0,1]^n$ является компактом.\label{comcube}\prooftms{comcube}

\prooftm{comcube} Будем за шаг делить куб пополам в каком-то направлении (перпендикулярном $Ox_i$), в котором мы его не делили последние $n-1$ шагов, и после каждого шага выбирать половину с бесконечностью точек и брать из неё любую точку. Тогда взятая последовательность сходится.\QEDA\\

\prooftm{comcube} Доказываем по индукции. Для $n=1$ верно.

{\bf Шаг индукции.} Возьмём подпоследовательность, сходящуюся по первым $n$ координатам. Из неё по теореме для $n=1$ можно выбрать подпоследовательность, сходящуюся по $n+1$-й координате. Она подходит.\QEDA\\

\theorem Любое замкнутое ограниченное множество $S$ --- компакт.\label{compact}

\proof Опишем вокруг множества достаточно большой куб. Этот куб является компактом, следовательно, для любой $a_i\subset S$ есть сходящаяся $b_i\subset a_i$. Тогда по замкнутости $S$ предел $b_i$ тоже лежит в $S$.\QEDA\\

\lemman{Таккер} Триангуляция правильного $2n$-угольника покрашена в цвета $\pm1,\pm2$ так, что противоположные вершины покрашены в противоположные цвета. Тогда существует пара соединённых вершин, покрашенных в противоположные цвета.

\theoremn{Борсук, Улам} Пусть $f:\mathbb S^n\to\mathbb R^n$ из $n$-мерной сферы в $n$-мерную плоскость непрерывна и нечётна (т.е. $f(-x)=-f(x)$). Тогда есть $x:f(x)=0$.\label{borul}\\

\prooftf{maint}{при $n=1$} Пусть $f(x)=\mu((x,+\infty))-\mu((-\infty,x))$. Тогда $f(+\infty)>0,f(-\infty)<0$, значит, по непрерывности есть точка такая что $f(0)=0$.\QEDA

\prooftf{maint}{при $n=2$} Для каждого направления прямой по непрерывности есть такое положение, что $\mu_1(H_l^+)=\mu_1(H_l^-)$. Пусть $f(\varphi)=\mu_2(H_{l_\varphi}^+)-\mu_2(H_{l_\varphi}^-)$. Заметим, что $f(\varphi)=-f(\varphi+\pi)$. Тогда по непрерывности для какого-то $\varphi$ выполняется $f(\varphi)=0$.\QEDA

\prooftf{maint}{через~\ref{borul}} Для каждого направления гиперплоскости есть такое положение, что $\mu_1(H_l^+)=\mu_1(H_l^-)$. Пусть \[
	f(v)=x_1(\mu_2(H_l^+)-\mu_2(H_l^-)) + \ldots + x_{d-1}(\mu_d(H_l^+)-\mu_d(H_l^-))
\] --- функция из сферы в $d$-мерное пространство. Заметим, что она нечётна и непрерывна, тогда по~\ref{borul} есть точка, где она обнуляется. Эта точка искомая.\QEDA\\

\newpage
\centerline{\large\scshape Кнезеровские графы}
\definition Кнезеровский граф $KG(n,k)$ --- граф, вершины которого --- $k$-элементные подмножества множества из $n$ элементов, и две вершины соединены ребром, если соответствующие подмножества не пересекаются.

Кликовое число такого графа равно $\omega(KG(n,k))=\lfloor\frac nk\rfloor$, а на число независимости можно написать такую оценку --- $\alpha(KG(n,k))\leq\binom{n-1}{k-1}$.

\theoremn{Эрдёш, Туран} $\alpha(KG(n,k))=\binom{n-1}{k-1}$.

\theorem $\chi(KG(n,k))\leq n-2k+2$.

\proof Пусть исходное множество было множеством $\{1,2,\ldots,n\}$. Если у подмножества минимальный элемент $i$, красим его в $i$-й цвет. Делаем так до тех пор, пока не используем $n-2k+1$ цвет. Когда мы это сделаем, у нас в конце останутся пересекающиеся подмножества, которые можно раскрасить в $n-2k+2$-й цвет.\QEDA\\

\theoremn{Люстерник, Шнирельман} Пусть $d$-мерная сфера покрыта $d+1$ открытыми (или замкнутыми) множествами, либо $d$ множествами одного типа и одним множеством другого типа. Тогда найдётся множество с противоположными точками. {\it Задача эквивалентна~\ref{borul}.}\label{lust}\\

\theorem Если $KG(n,k)$ нетривиален, то $\chi(KG(n,k))=n-2k+2$.\label{knezerchroma}

\proof Рассмотрим $n$ точек на $\mathbb S^d$ так, что они и центр сферы общего положения, где $d$ --- число цветов в какой-то раскраске. Тогда на каждом большом экваторе не более $d$ точек. Рассмотрим множества $A_1,\ldots,A_d$, взятые следующим образом. Возьмём любую точку $x$, представим, что она --- северный полюс, и рассмотрим северное открытое полушарие. Если в нём есть $k$ точек, покрашенных в цвет $r$, то точка $x$ лежит в множестве $A_r$. Тогда все множества $A_i$ открытые. Пусть теперь $A_{d+1}=\mathbb S^d/A_1/\ldots/A_d$. Тогда по \ref{lust} есть две точки, лежащие в одном множестве. Теперь рассмотрим два случая:
\begin{itemize}
	\item Пусть эти две точки в $A_r$ при $r\leq d$. Тогда есть две вершины, соответствующие этим $k$-элементным подмножествам. Заметим, что они должны быть соединены ребром, но они $r$-го цвета --- противоречие.
	\item Эти две точки в $A_{d+1}$. Заметим, что если $x\in A_{d+1}$, то в его полушарии нет $k$ точек никакого цвета, значит, там меньше $k$ точек. Тогда всего точек не больше $2k-2+d$. Значит, $2k-2+d\geq n$, что заканчивает доказательство.\QEDA\\
\end{itemize}

\theoremn{Радон} Пусть $X_1,\ldots,X_{d+2}\in\mathbb R^d$. Тогда их можно разбить на 2 множества так, что их выпуклые оболочки пересекаются.\label{radon}\\

\prooft{helly} Доказываем по индукции.

{\bf База индукции.} $n=d+2$. Пусть $X_i$ --- точка пересечения всех множеств, кроме $A_i$. Применим \ref{radon}. Тогда $X_i$ разобьются на 2 множества $A$ и $B$. Пусть их выпуклые оболочки содержат одну и ту же точку $C$. Тогда все $A_i$ содержат точку $C$.

{\bf Шаг индукции.} Пусть мы умеем доказывать для $n$. Добавим множество $A_{n+1}$. Заметим, что $A_1\cap A_{n+1}$ выпукло и для $A_2,\ldots,A_{n-1},A_1\cap A_{n+1}$ выполняется теорема.\QEDA\\

\theorem Пусть конечные $F_1,\ldots,F_d\subset2^{\mathbb R^d}$ таковы, что $\forall i\ \forall A,B\in F_i$ пересекаются. Тогда существует гиперплоскость, пересекающая все.\label{divis}

\proof Будем считать, что все множества в семействах ограничены. Спроецируем их все на какую-то прямую. Мы получим набор попарно пересекающихся отрезков для каждого семейства. По~\ref{helly} для $d=1$ они пересекаются по отрезку (или точке). Заменим каждый отрезок пересечения на его середину. Пусть $f(l)$ --- вектор из расстояний от этих середин до одной фиксированной из них. Заметим, что $f:\mathbb S^{d-1}\to \mathbb R^{d-1}$ нечётна. Тогда по~\ref{borul} в какой-то точке она обнуляется. Проведём такую прямую, на ней будет точка, являющаяся серединой всех пересечений проекций, т.е. лежащая на проекции каждого множества. Тогда если провести перпендикулярную гиперплоскость, она пересечёт все множества.\QEDA\\

\theoremn{Дольников} Пусть $\mu_1,\mu_2,\ldots,\mu_k$ --- хорошие меры над $\mathbb R^d$, нормированные так, что $\mu(\mathbb R^d)=1$. Тогда существует такое $k-1$-мерное подпространство, что для любого полупространства $H^+$, её содержащего, выполняется $\mu_i(H^+)\geq\frac1{d+2-k}$.\footnote{Эта теорема --- обобщение как~\ref{limp} при $k=1$, так и~\ref{maint} при $k=d$. Также существует множественный аналог, обобщающий и~\ref{divis}, и~\ref{helly}.}\\

\theorem Для любого конечного $A\subset\mathbb R^d$ существует такая точка $p$, что любое полупространство, её содержащее, содержит хотя бы $\frac1{d+1}$ точек из $A$.

\proof Аналогично~\ref{limp}.\QEDA\\

\lemma \ref{helly} неверна для бесконечного количества множеств.

\proof Можно рассмотреть такие примеры: <<замкнутый>> $X_n=\{(x,y)|x\geq n\}$ и <<ограниченный>> $X_n=(0,\frac1n)$. Они, очевидно, не подойдут.\QEDA\\

\theorem Есть 100 слив с косточками весом от 1 до 2 кг каждая. Общая масса косточек втрое меньше общей массы слив. Тогда их можно разбить на 2 ящика по 50 слив так, что в каждой доля косточек меньше 35\%.

\proof Возьмём 100 точек общего положения в $\mathbb R^3$, опишем вокруг них маленькие шары и поставим на каждом шаре три меры --- $\mu_N(S)=1,\mu_S,\mu_K$. Применим~\ref{maint}. Получившаяся гиперплоскость пересекает от 0 до 3 шаров. Дальше нужно писать оценки, но у лектора это не удалось.\\

\theorem Два разбойника нашли ожерелье с $d$ видами камней, каждого вида чётно. Ожерелье имеет вид незамкнутой проволоки. Тогда можно разделить ожерелье $d$ разрезами и раздать части так, что каждый получит поровну камней каждого вида.

\proof Определим кривую моментов в $d$-мерном пространстве как множество $\{(t,t^2,\ldots,t^d)|t\in\mathbb R\}$.

\lemma Кривая моментов пересекается с любой гиперплоскостью не более чем в $d$ точках.

\proof Пусть гиперплоскость имеет вид $a_1x_1+\ldots+a_dx_d=c$. Тогда пересечения её с кривой моментов являются решением многочлена $a_1t+a_2t^2+\ldots+a_dt^d-c=0$, у которого не более $d$ корней по теореме Безу.\QEDA\\

{\bf Завершение доказательства.} Поставим ожерелье на кривую моментов и применим \ref{discrete}. Мы получим гиперплоскость, не пересекающуюся с точками ожерелья, с каждой стороны от которой поровну камней каждого вида. Тогда разрезание ожерелья по этой гиперплоскости даст решение задачи.\QEDA\\

\theorem Дана хорошая мера $\mu:\mathbb R^2\to\mathbb R$. Тогда можно провести две прямые, делящие плоскость на 4 части, в каждой из которых мера одинакова.

\proof Разделим на 2 равные части как угодно, затем применим~\ref{maint} для двух мер --- <<слева>> и <<справа>>.\QEDA\\

\theorem Существует такое $d$ и хорошая мера $\mu:\mathbb R^d\to\mathbb R$, что нельзя провести $d$ гиперплоскостей, делящих $\mathbb R^d$ на $2^d$ частей, в каждой из которых мера одинакова.

\proof Пусть вся мера сосредоточена на кривой моментов.\\

Будем красить семейство множеств так --- красим каждый элемент в один из $m$ цветов, чтобы в семействе не было множества одноцветных элементов. В частности, если множества 2-элементные (<<рёбра>>), это классическая раскраска графов.

\definition Цветной дефект семейства множеств $cd_m(\mathcal F)$ --- минимальное такое $r$, что можно удалить $r$ множеств так, что результат будет краситься в $m$ цветов. 

\definition Обозначим за $KG(\mathcal F)$ граф, вершины которого --- элементы $\mathcal F$, и две вершины соединены ребром, если соответствующие множества не пересекаются. В частности, $KG(n,k)=KG(\mathcal F=C^k_A)$.

\theorem $\chi(KG(\mathcal F))\geq cd_2(\mathcal F)$.

\proof Аналогично~\ref{knezerchroma}. Финальная оценка строится так: выкинем все точки на экваторе, все остальные точки по аналогичным причинам покрасятся в 2 цвета.\QEDA\\

\end{document}
