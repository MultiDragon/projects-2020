
\documentclass[a4paper,12pt]{article}
\usepackage[russian]{babel}
\usepackage[utf8]{inputenc}
\usepackage[a4paper, margin=2.5truecm, top=1.8truecm, bottom=1.3truecm]{geometry}
\usepackage{hyperref}
\usepackage{graphicx}
\usepackage{amsmath,amsfonts,amssymb}
\usepackage{tikz}

\relpenalty=10000 \binoppenalty=10000
\newcounter{theorems}
\newcommand*{\lemma}{\refstepcounter{theorems}{\bf Лемма \arabic{theorems}. }}
\newcommand*{\theorem}{\refstepcounter{theorems}{\bf Теорема \arabic{theorems}. }}
\newcommand*{\lemman}[1]{\refstepcounter{theorems}{\bf Лемма \arabic{theorems} (#1). }}
\newcommand*{\theoremn}[1]{\refstepcounter{theorems}{\bf Теорема \arabic{theorems} (#1). }}

\newcommand*{\matrixtt}[4]{\binom{#1\ #2}{#3\ #4}}
\newcommand*{\matrixot}[2]{\binom{#1}{#2}}
\newcommand*{\matrixto}[2]{(#1\ #2)}
\newcommand*{\chain}{\dfrac1}
%\newcommand*{\det}[1]{\left|#1\right|}

\newcounter{defs}
\newcommand*{\definition}{\refstepcounter{defs}{\bf Определение \arabic{defs}. }}

\newcommand*{\QEDA}{\hfill\ensuremath{\blacksquare}}

\newcommand{\proof}{{\bf Доказательство. }}
\begin{document}

\begin{center}
{\it Комбинаторика и алгоритмы}
\vskip9pt\hrule\vskip14pt
\vfill
{\Large\bf Диаграммы Фарея, цепные дроби и топокарты}
\vskip20pt
{\large\bf Соколов Артемий Алексеевич}

\vfill
Берендеевы поляны, 19--23 августа 2019 г.
\end{center}

\newpage

\centerline{\large\scshape Диаграммы Фарея}
\definition Медианта отрезка с концами $\frac ab$ и $\frac cd$ ($(a,b)=(c,d)=1,\frac ab<\frac cd$) --- точка с координатами $\frac{a+c}{b+d}$.\\

Заметим, что 
\[
\frac ab<\frac{a+c}{b+d}\iff a(b+d)<b(a+c)\iff ad<bc \iff d(a+c)<c(b+d) \iff\frac{a+c}{b+d}<\frac cd
.\]

Будем отмечать на отрезке $(0,1)$ точки по очереди таким образом. Изначально отмечены концы отрезка. За один шаг берутся все текущие отрезки и для каждого из них отмечается их медианта. Тогда получатся {\it диаграммы Фарея}.

Заметим, что после каждого шага для соседних дробей $\frac ab$ и $\frac cd$ выполняется $ad-bc=-1$. Действительно, $a(b+d)-b(a+c)=ad-bc=(a+c)d-(b+d)c$ и изначально эта разность равна $-1$.

\newpage

\centerline{\large\scshape Цепные дроби}
\begin{equation}
	\dfrac pq=a_0+\chain{a_1+\chain{a_2+\dfrac{\vphantom{a_n1}\dots}{a_{n-1}+\chain{a_n}}}}
\end{equation}

Другие формы записи:
\[\dfrac pq=[a_0;a_1,a_2,\dots,a_n]\]
\[\dfrac pq=a_0+1/(a_1+1/\dots)\]

Например:
\begin{equation}
	\dfrac7{11}=0+\chain{1+\dfrac47}=0+\chain{1+\chain{1+\dfrac34}}=0+\chain{1+\chain{1+\chain{1+\chain3}}}=[0;1,1,1,3]
\end{equation}

\centerline{\scshape Определение по Гурвицу}
Идём по Диаграммам Фарея. Будем считать, что $0<p<q$. Вначале мы находимся справа от точки $1$ и движение направлено влево. На каждом шаге делаем один шаг диаграммы Фарея и идём в нужную сторону. Когда мы меняем направление (перед шагом по новому направлению), добавляем текущее число в (изначально пустой) массив в начало, затем обнуляем текущее число.\\

\centerline{\scshape Способ подсчёта}
Определяем $x=\dfrac pq$ рекурсивно по $[a_0;a_1\dots]$. Изначально $p_0=a_0,q_0=1$. Затем подставляем $a_0=0$ и пишем рекурренту: \[
	p_{n+1}=p_na_{n+1}+a_{n-1},\ 
	q_{n+1}=q_na_{n+1}+q_{n-1}
\]
Тогда $x=\dfrac{p_n}{q_n}$.
\newpage

\centerline{\large\scshape Матрицы}
Будем сопоставлять дроби $\frac pq$ матрицу $\matrixot qp$.\\

\centerline{\scshape Операции над матрицами}
\begin{itemize}
	\item Матрицы можно складывать: $\matrixtt abcd+\matrixtt pqrs=\matrixtt{a+p}{b+q}{c+r}{d+s}$.
	\item Матрицы можно умножать на число: $\matrixtt abcd\cdot n=\matrixtt{an}{bn}{cn}{dn}$.
	\item От матриц можно брать определитель: $\det\matrixtt abcd=ad-bc$. В частности, $\det\matrixtt a{ka}b{kb}=0$.
	\item Матрицу $2\times1$ можно умножить на матрицу $1\times 2$, при этом получается число: $\matrixto ac\cdot\matrixot xy=ax+cy$. Умножение больших матриц делается по дистрибутивности, например: $\matrixtt acbd\cdot\matrixtt xzyt=\matrixtt{ax+cy}{az+ct}{bx+dy}{bz+dt}$.
	\item Матрицу $2\times2$ можно обращать: $\matrixtt abcd^{-1}=\det^{-1}\matrixtt abcd\cdot\matrixtt d{-b}{-c}a$. Обращение работает так: $\matrixtt abcd\cdot\matrixtt abcd^{-1}=\matrixtt 1001$.
	\item Матрицы $1\times2$ можно перемножать, при этом получается дробь: $\frac ab\otimes\frac cd=\matrixot ba\cdot\matrixot dc=\matrixot{b+d}{a+c}=\frac{a+c}{b+d}$.
\end{itemize}

\newpage
\centerline{\large\scshape Квадратичные формы}
\definition Квадратичная форма --- однородный многочлен от двух переменных степени 2, т.е. $ax^2+bxy+cy^2$. Если форма целочисленная, то $a,b,c,x,y\in\mathbb Z$ (можно считать, что это функция $Q:\mathbb Z^2\rightarrow\mathbb Z$).

\definition Базис --- множество векторов в пространстве $\mathbb F^n$, через линейную комбинацию которых с коэффициентами из $\mathbb F$ представляется каждый вектор из пространства. При представлении $\vec v=k_1\vec{e_1}+\dots+k_n\vec{e_n}$ числа $k_1\dots k_n$ называются координатами $\vec v$ в базисе $\vec{e_1}\dots\vec{e_n}$. Нестрогим базисом назовём базис, перед каждым вектором которого стоит $\pm$.

Если $\vec{e_1}=\matrixot ab,\vec{e_2}=\matrixot cd$, то $k\vec{e_1}+l\vec{e_2}=\matrixot{ka+lc}{kb+ld}=\matrixtt acbd\matrixot kl$. Поэтому если таким образом представляется любой вектор, то $\det\matrixtt abcd=\pm 1$.

\definition Супербазис --- тройка векторов $(\vec{e_1},\vec{e_2},-\vec{e_1}-\vec{e_2})$, где $(\vec{e_1},\vec{e_2})$ образуют базис. Аналогично определяется нестрогий супербазис.\\

% TODO: картинка базисного фрактала (topograph) --- супербазисы как вершины, базисы как рёбра, вектора как грани

\lemman{об арифметической прогрессии}
\begin{align*}
	Q(v)&=Ax^2+Bxy+Cy^2\\
	Q(v_1\pm v_2)&=A(x_1\pm x_2)^2+B(x_1\pm x_2)(y_1\pm y_2)+C(y_1\pm y_2)^2\\
	Q(v_1+v_2)+Q(v_1-v_2)&=2(Q(v_1)+Q(v_2))
\end{align*}

В частности, если $Q(v)=x^2+y^2$, то это тождество параллелограмма.\\

Будем писать на гранях фрактала $Q(k\vec{e_1}+l\vec{e_2})$, а на рёбрах --- разность числа на грани справа и суммы чисел на соседних гранях (или, что эквивалентно, разность суммы и числа на граммы слева). Затем проведём на грани стрелку в сторону увеличения.

\lemman{о возрастании} Пусть $Q(x)>0,Q(y)>0,h>0$. Тогда справа от первого базиса все стрелки направлены вправо.\label{incr}

\lemman{о колодцах} Пусть $Q(x)>0,Q(y)>0$. Тогда есть такая точка, что все стрелки на рёбрах направлены от неё.\\

Будем красить ребро в красный, если с одной стороны от него положительное число, а с другой отрицательное. Тогда по~\ref{incr} получится ломаная (<<река>>). Она конечна (с одной стороны) тогда и только тогда, когда на топокарте есть грань, в которой записан $0$ (<<озеро>>), и конечна с обоих сторон тогда и только тогда, когда на топокарте два <<озера>>.\\

\centerline{\scshape Решение квадратичных форм}
Чтобы проверить, есть ли решение уравнения $ax^2+bxy+cy^2=n$ в целых числах, можно идти по реке, находить на ней все <<отростки>> нужного знака и идти по ним. На каждом отростке модуль увеличивается, значит, по каждому отростку мы будем идти конечное количество времени. Однако река обычно бесконечная. Докажем, что на самом деле она периодичная.

Пусть форма имеет вид $ax^2+bxy+cy^2$. Тогда в грани с $x$ написано $a$, с $y$ --- $c$, с $x+y$ --- $a+b+c$, с $x+2y$ --- $a+2b+4c$ и т.п. Заметим, что форма представляется в виде перемножения матриц --- $Q(x,y)=\matrixto xy\matrixtt a{b/2}{b/2}c\matrixot xy$. Обозначим $D_Q=ac-\frac{b^2}4=\frac{-D}4$.\\

\lemma Для каждого ребра топокарты $D_Q$ постоянен.

\proof Доказываем по индукции по длине топокарты. Для первого ребра это верно по определению. Для следующего ребра это верно, так как: \[
	a(a\pm b+c)-\left(a\pm\frac b2\right)^2=a^2+ab+ac-a^2-ab-\frac{b^2}4=ac-\frac{b^2}4
.\QEDA\]

\theorem Вершины реки топокарты периодичны.

\proof $ac-\frac{b^2}4=-d_1$ --- константа. Тогда $d_1=-ac+\frac{b^2}4$, следовательно, $\frac{b^2}4<d_1$, т.к. $ac<0$ по определении реки. Следовательно, существует всего конечное количество чисел $b$, т.е. конечное количество чисел $ac$, а т.к. $a,c\in\mathbb Z$, то количество троек $(a,b,c)$ на реке конечно. С другой стороне, по тройке топокарта восстанавливается однозначно, значит, это период.\QEDA\\

Следовательно, существует {\bf конечный} алгоритм проверки, принимает ли квадратичная форма заданное целое значение.

\vskip20pt\hrule\vskip8pt
\centerline{\large\scshape Связь с диаграммами Фарея}

Нарисуем граф Диаграммы Фарея. Заметим, что двойственный к нему --- топокарта (как бесконечный граф со степенью 3). Каждая грань топокарты содержит единственную точку на границе Диаграммы. Тогда вектор, написанный в грани топокарты, равен дроби, записанной в этой точке диаграммы.

\begin{tikzpicture}
	\path[draw] (-2,0) -- (2,0);
	\path[draw] (-3,3) -- (-2,0) -- (-3,-3);
	\path[draw] (3,3) -- (2,0) -- (3,-3);
	\path[draw] (-5,4) -- (-3,3) -- (-2,5);
	\path[draw] (5,4) -- (3,3) -- (2,5);
	\path[draw] (-5,-4) -- (-3,-3) -- (-2,-5);
	\path[draw] (5,-4) -- (3,-3) -- (2,-5);

	\node at(0,2) {$Q(0,1)$};
	\node at(0,-2) {$Q(1,0)$};
	\node at(4,0) {$Q(1,1)$};
	\node at(-4,0) {$Q(1,-1)$};
	\node at(3.5,-5) {$Q(2,1)$};
	\node at(-3.5,-5) {$Q(2,-1)$};
	\node at(3.5,5) {$Q(1,2)$};
	\node at(-3.5,5) {$Q(1,-2)$};
\end{tikzpicture}

\newpage
\centerline{\large\scshape Цепные дроби-II}
Рассмотрим разложение $\sqrt2$ в цепную дробь:
\begin{equation}
	\sqrt2=1+\chain{\frac1{\sqrt2-1}}=1+\chain{\sqrt2+1}=1+\chain{2+\sqrt2-1}=[1;2,2,2\dots]=[1;\overline2]
\end{equation}
%\begin{equation}
%	\sqrt3=1+\chain{\frac1{\sqrt3-1}}=1+\chain{\frac{\sqrt3+1}2}=[1;1,2,1,2\dots]=[1;\overline{1,2}]
%\end{equation}

\theoremn{Лагранж} Цепная дробь $\alpha$ периодична тогда и только тогда, когда $\exists a,b,c\in\mathbb Z:a\alpha^2+b\alpha+c=0$, т.е. $\exists a,b\in\mathbb Q,n\in\mathbb N:\alpha=a+b\sqrt n,n\neq m^2$.

{\bf Идея доказательства.}
\begin{equation}
	\alpha(x)=a_0+\chain{a_1+\chain{a_2+\dfrac{\vphantom{a_n1}\dots}{a_{n-1}+\chain{a_n+x}}}}=\dfrac{m+nx}{l+kx}
\end{equation} Тогда мы решаем уравнение вида $\dfrac{m+nx}{l+kx}=\dfrac{m'+n'x}{l'+k'x}$, а оно квадратное.\QEDA\\

\theorem $\sqrt d=[a_0;\overline{a_1,a_2,\dots,a_n}]$, причём $a_n=2a_0$, а остальные $a_k$ образуют палиндром.\\

Нарисуем топокарту для $Q=x^2-2y^2$, затем выпрямим реку. Тогда, если соединить соседние смежные с рекой грани отрезками, получится разложение $\sqrt2$ в цепную дробь по Гурвицу. Кроме того, вектора, находящиеся в гранях, в которых написано $\pm 1$, дают приближения $\sqrt2$ и решения уравнения Пелля. Наконец, если делать то же самое для формы $x^2-dy^2$, получится то же самое для $\sqrt d$.

\begin{tikzpicture}
	\path[draw,red] (-2,0) -- (2,0);
	\path[draw] (-3,3) -- (-2,0);
	\path[draw,red] (-2,0) -- (-3,-3);
	\path[draw] (3,3) -- (2,0);
	\path[draw,red] (2,0) -- (3,-3);
	\path[draw] (-5,4) -- (-3,3) -- (-2,5);
	\path[draw] (5,4) -- (3,3) -- (2,5);
	\path[draw,red] (-5,-4) -- (-3,-3);
	\path[draw] (-3,-3) -- (-2,-5);
	\path[draw,red] (5,-4) -- (3,-3);
	\path[draw] (3,-3) -- (2,-5);

	\node at(0,2) (y) {$-2$};
	\node at(0,-2) (x) {$1$};
	\node at(4,0) (xy) {$-1$};
	\node at(-4,0) (yx) {$-1$};
	\node at(3.5,-5) (xxy) {$2$};
	\node at(-3.5,-5) (yxx) {$2$};
	\node at(3.5,5) (xyy) {$-7$};
	\node at(-3.5,5) (yyx) {$-7$};

	\path[draw,blue] (yxx) -- (yx) -- (x) -- (y);
	\path[draw,blue] (x) -- (xy) -- (xxy);
\end{tikzpicture}

\centerline{\scshape Решение уравнений Пелля}
Посмотрим на матрицу, умножение на которую переводит тривиальное решение уравнений $x^2-dy^2=\pm1$ в первое нетривиальное. Умножим тривиальное решение на эту матрицу в какой-то степени и получим решение. Например, для $\sqrt2$ это $\matrixtt 3423$, и умножение на эту матрицу даёт следующее решение, например, $\matrixtt 3423\cdot\matrixot 32=\matrixot{17}{12}$.

\end{document}
