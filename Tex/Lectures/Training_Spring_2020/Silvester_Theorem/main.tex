\documentclass[12pt,a4paper]{article}
\usepackage{tpl}
\dbegin[15 мая 2020 г.]{Весенние сборы 2020}{Задача Сильвестра}{Тиморин Владлен Анатольевич}

Пусть на плоскости есть конечное множество точек $A$, не все из которых коллинеарны.

\definition{Соединительная прямая} прямая, проходящая через хотя бы 2 точки.

\definition{Прямая Сильвестра} прямая, проходящая через ровно 2 точки.

\theorem{Галлаи, Сильвестр} для любого множества $A$ существует прямая Сильвестра.\label{silv}

\prooftms{silv}

\prooftmn{silv}{Келли} Рассмотрим все пары $(l\ni x,y\in A,a\in A\setminus l)$. Пусть $(l^*,A^*)$ пара с минимальным расстоянием между $A$ и $l$. Пусть $l^*$ не прямая Сильвестра. Значит, на ней хотя бы 3 точки, значит, хотя бы 2 из них находятся по одну сторону от основания перпендикуляра из $A^*$ на $l^*$. Тогда $d(B,AC)<d(l^*,A^*)$ (см.рис.), противоречие.\QEDA

\textbf{Проблема этого доказательства:} оно ломается при проективном преобразовании. Если сделать проективное преобразование, мы часто получим новую прямую Сильвестра.\\

\shdr{Проективные свойства}
\begin{itemize}
	\item Прямая содержит точку.
	\item Точка принадлежит прямой.
	\item Две пары точек разделяют или не разделяют друг друга.
\end{itemize}

\prooftmn{silv}{Галлаи} Рассмотрим проекцию, при которой ровно одна точка переходит на бесконечность. Будем считать, что точка на бесконечности находится по вертикальному направлению. Предположим, что прямой Сильвестра не существует (т.е. на любой вертикальной прямой хотя бы 2 точки, а на не-вертикальной хотя бы 3). Рассмотрим соединительную прямую, составляющую самый острый угол с вертикалью. На вертикальной прямой через $K$ есть ещё точка $X$. Если она <<выше>> $K$, то угол между вертикалью и $XA$ уменьшился, а если <<ниже>>, то уменьшился угол между вертикалью и $XB$ (см.рис.)\QEDA

\begin{figure}[!htb]
	\begin{minipage}{0.48\textwidth}\centering
		\inkscapeinclude{silvestr}{}
		\caption{Доказательство Келли}
	\end{minipage}
	\begin{minipage}{0.48\textwidth}\centering
		\inkscapeinclude{gallaiprf}{}
		\caption{Доказательство Галлаи}
	\end{minipage}
\end{figure}

\textbf{Почему углы лучше расстояния.} Из-за <<свойства разделения>> большие углы после проекция соответствуют большим углам до проекции, значит, доказательство проективно.

\theoremn{Келли, Мозер, 1958} Прямых Сильвестра не меньше $\frac{3n}{7}$.\\

\theorem В $\mathbb C^2$ \ref{silv} не выполняется.

\proof Рассмотрим $A_i=(0,\zeta^i),B_i=(\zeta^i,0)$, где $\zeta=\cos 120^\circ+i\sin 120^\circ$. Рассмотрим такие тройки прямых: $A_iB_{i+j}$ (нумерация циклическая, $j$ фиксировано для тройки). Каждая из этих троек --- три параллельные прямые, и на них можно взять точку на бесконечности $C_j$. Тогда точки $A_i,B_i,C_i$ противоречат \ref{silv}.\QEDA\\

Кстати, эта конфигурация ещё есть в $\mathbb Z^2_3$ (см. рисунок ниже), игра Сет основана на её четырёхмерной версии, кроме того, точки перегиба у кривой $x^3+y^3=1$ также соответствуют этой конфигурации (проверяется счётом в координатах).

\newpage

\inkscapefigure{gesseconf}{Конфигурация Гессе в $\mathbb Z^2_3$. Параллельные прямые обозначаются одним цветом.}

\hdr{Теорема на сфере}

Спроектируем сферу в плоскость из центра сферы. У нас диаметрально противоположные точки склеятся, будем считать их одинаковыми. Тогда прямым на плоскости будут соответствовать диаметральные окружности на сфере. Кроме того, точки на горизонтальном диаметре перейдут в бесконечно удалённые точки.

\shdr{Сферическая двойственность}

Будем считать, что экваториальная окружность $\alpha$ двойственна паре из северного и южного полюсов $A$, и наоборот. Обозначение: $\alpha\sim A$.

\shdr{Факты про двойственность}
\begin{itemize}
	\item Если $\alpha\ni B,\alpha\sim A,\beta\sim B$, то $A\in\beta$.
	\item $d(A,B)=\angle(\alpha,\beta)$ (под углом между окружностями имеется в виду угол между их плоскостями, а расстояние между точками угловое).
\end{itemize}

\theoremn{Двойственная сферная т. Сильвестра} Отмечено несколько диаметральных окружностей на сфере, не все из которых имеют общей точки. Тогда существует точка, через которую проходят ровно две окружности.\label{reverse}\\

\definition{Карта на сфере} конфигурация точек и кривых на сфере, такая, что кривые пересекаются только в отмеченных точках. Точки называются вершинами, кривые рёбрами, а области между кривыми --- гранями.

Заметим, что большие окружности на сфере образуют карту с вершинами в их точках пересечения. При этом рёбрами будут части окружностей.

\theoremn{Мельхиор, Стинрод} Не существует такой карты на сфере, что каждая грань ограничена хотя бы тремя рёбрами, а из каждой вершины выходит хотя бы 6 рёбер.\label{planar}

\prooft{planar} Это свойство планарных графов (каждая карта на сфере --- это планарный граф).\QEDA\\

\prooft{reverse} Докажем, что выполняются условия \ref{planar}. Во-первых, если у нас есть двуугольник (т.е. грань с двумя рёбрами), то все окружности проходят через его концы. Ещё мы знаем, что через каждую вершину проходят хотя бы 3 окружности, и каждая из них даёт два ребра (если теорема неверна). Задача доказана.\QEDA\\

\newpage
\hdr{Теорема в пространстве}

Правда ли, что если есть конечное количество точек в $\mathbb R^3$, не все из которых в одной плоскости, то можно найти плоскость, на которой ровно 3 отмеченные точки? Нет. Можно рассмотреть 2 скрещивающиеся прямые, на каждой из которых по 4 точки. Если есть плоскость через 3 точки, то хотя бы 2 из них находятся на одной из двух прямых, а значит, и все 4 точки этого цвета присутствуют (см.рис.) Однако в пространстве верен её аналог.

\inkscapefigure{crossline}{Прямой аналог теоремы Сильвестра неверен в $\mathbb R^3$}

\theoremn{Моцкин} Пусть есть несколько точек в $\mathbb R^3$, не все из которых компланарны. Тогда можно выбрать плоскость так, что на ней хотя бы 3 точки, не все точки на этой плоскости лежат на одной прямой, и можно удалить одну из точек, чтобы все оставшиеся лежали на одной прямой.\\

\theoremn{Сильвестр на сфере} Пусть на сфере есть конечное число точек, не все из которых лежат на одной окружности. Тогда есть окружность, на которой лежит ровно 3 точки.\label{sphere}

\proof Рассмотрим плоскость Моцкина. На ней есть точка и прямая. На прямой лежат максимум 2 точки, значит, в плоскости ровно 3 точки. Они образуют плоскость, которая пересекается с сферой по окружности.\QEDA\\

\theorem Пусть на плоскости конечное число точек, не все из которых на одной прямой или окружности. Тогда существуют три точки, лежащих на одной прямой или окружности.\footnote{Лектору неизвестно, есть ли аналогичное утверждение без прямых, скорее всего должно быть верно.}

\proof Сделаем стереографическую проекцию и применим \ref{sphere}.\QEDA\\

\theorem Пусть на плоскости конечное число точек, не все из которых лежат на одной прямой или окружности. Зафиксируем $A$ --- одну из этих точек. Тогда есть прямая или окружность, которая содержит точку $A$ и ещё ровно две отмеченные точки.

\proof Будем доказывать это утверждение для сферы. Сделаем стереографическую проекцию в точке $A$. Мы получим на одну точку меньше на плоскости. Применим \ref{silv}. Получим, что на нижней плоскости будет прямая, на которой две точки. Тогда у соответствующей окружности будут образы этих двух точек и точка $A$.\QEDA\\

\definition{Кв-ный трёхчлен} график ур-я $ax^2+bx+c=y$ (возможно, $a=0$).

\theoremn{Борвейн} На плоскости отмечено конечное число точек с различными абсциссами, не все из которых лежат на графике одного и того же квадратного трёхчлена. Тогда существует КТ, на котором лежат ровно 3 отмеченные точки.\label{analog}

\proof Рассмотрим пары $(A,g)$, где $A$ --- отмеченные точки, а $g$ --- квадратные трёхчлены, не проходящие через $A$. Рассмотрим пару такую, что $|y_A-g(x_A)|$ минимально среди всех трёхчленов. Пусть $g$ проходит через точки с абсциссами $x_1<x_2<x_3<x_4$. Пусть $x_2<a<x_3$ (остальные случаи аналогичные). Пусть $f$ --- КТ, проходящий через точки $A,x_1,x_4$ (см.рис.) Обозначим $h=f-g$. Докажем, что $h(x_2)$ или $h(x_3)$ меньше $|y_A-g(x_A)|$. Действительно, если $A$ правее вершины $h$, то $h(x_3)$ мало, иначе $h(x_2)$. Противоречие с выбором пары $(A,g)$.\QEDA\newpage

\imagefigure{parabols}{Теорема Сильвестра для квадратных трёхчленов}

\theorem Аналоги \ref{analog} работают для:\\
\p многочленов степени $d$ (эта теорема для многочленов степени $2$);\\
\p функций вида $y=a\sin x+b\cos x+c$.\\

\definition{Коника} кривая, задающаяся уравнением $a_0+a_1x+a_2y+a_3x^2+a_4y^2+a_5xy=0$ (если на ней лежит бесконечное количество точек).

\theoremn{Вайзман, Вильсон} Пусть отмечено конечное количество точек на плоскости, причём не все лежат на одной конике. Тогда существует коника, проходящая ровно через 5 отмеченных точек, причём она определяется этими точками (т.е. если какие-то 4 лежат на одной прямой, то и все 5 лежат).\label{cubic}

\textbf{Вопрос.} Правда ли, что если существует конечное количество точек, не все из которых лежат на одной кубике, то есть кубика, проходящая через ровно 9 отмеченных точек и определяющаяся ими?

Эта задача не решена. Известно, что если убрать последнее условие, то ответ <<существует>>.\\

\definition{Барицентрические координаты точки $P$ относительно $\triangle ABC$} такие числа $(a:b:c)$, что $a^2+b^2+c^2\neq 0$ и $a\cdot \overline{PA}+b\cdot \overline{PB}+c\cdot \overline{PC}=\overline{0}$.

\textbf{Факт.} Если $(a:b:c)$ --- координаты точки $P$, то и $(\lambda a:\lambda b:\lambda c)$ тоже для любого $\lambda\neq 0$.

\textbf{Факт.} Уравнение коники в БК --- это однородное квадратное уравнение относительно $a,b,c$.

\textbf{Факт.} Если коника проходит через $A,B,C$, то её уравнение имеет вид $kab+lbc+mca=0$.\\

Рассмотрим какую-то конику, описанную около $\triangle ABC$. Сделаем замену $u=\frac{1}{x},v=\frac{1}{y},w=\frac{1}{z}$. Получим $ku+lv+mw=0$, уравнение прямой в БК. Назовём эту замену \textit{стандартным квадратичным преобразованием}.

\prooftf{cubic}{(идейно)} Зафиксируем три точки $A,B,C$ и сделаем СКП относительно этого треугольника. Если мы нашли <<хорошие>> три точки, то после преобразования у нас останутся $n-3$ точки, можно найти среди них прямую Сильвестра $D'E'$ и получить, что коника $ABCDE$ подходит.\QEDA\\

\newpage

\lemma Пусть $P\in AB,Q\in BC,R\in CA$. Тогда минимальный (по площади) из треугольников $\triangle PQR,\triangle QRC,\triangle PQB,\triangle PRA$ --- это не треугольник $\triangle PQR$ (он не может быть строго минимальным, и если он нестрого минимальный, то $P,Q,R$ --- середины сторон).\label{lemma}

\prooftmn{silv}{Элвис, Зайденберг} Доказываем двойственную теорему. Предположим, что никакие 2 прямые не параллельны (иначе будем смотреть на проективную плоскость). Рассмотрим $\triangle PQR$ наименьшей площади, образованный отмеченными прямыми. Пусть никакая из точек $P,Q,R$ не подходит. Тогда через каждую из них проходит по прямой. Рассмотрим два случая:

\begin{itemize}
	\item Есть прямая, которая создаёт ненулевой отрезок внутри треугольника. Пусть она проходит через $Q$. Тогда $S(\triangle PQR)>S(\triangle PQS)$, где $S$ --- это пересечение этой прямой с $PR$. Противоречие с выбором $\triangle PQR$.
	\item Такой прямой нет. Тогда это противоречит \ref{lemma}, потому что у нас образуется $\triangle ABC$ (на самом деле, возможно, что точки $P,Q,R$ лежат на продолжении его сторон, но этот случай решается обобщением леммы).\QEDA\\
\end{itemize}

%\inkscapefigure[2]{elvisproof}{Доказательство Элвиса}

\end{document}
