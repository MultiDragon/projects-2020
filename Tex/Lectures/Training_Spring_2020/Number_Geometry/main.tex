\documentclass[12pt,a4paper]{article}
\usepackage{tpl}
\dbegin[13 мая 2020 г.]{Весенние сборы 2020}{Комбинаторика в геометрии чисел}{Райгородский Андрей Михайлович}

Будем говорить про решётки в пространстве.

\definition{Решётка $\Lambda$ в пространстве $\mathbb R^n$} множество точек $X\in \mathbb R^n$, которые задаются в виде целочисленной комбинации векторов $\overline{b_1},\overline{b_2},\ldots ,\overline{b_n}$. То есть все точки $X=\sum a_i\overline{b_i},a_i\in \mathbb Z$. (эти вектора фиксированы для $\Lambda$)

\definition{Решётка в $\mathbb R^n$} дискретная абелева (коммутативная) подгруппа по сложению векторов, заполняющая всё пространство.\\

Рассмотрим какой-то вектор с рациональными координатами \[
	\overline{a}=\{\frac{a_1}{q},\frac{a_2}{q},\ldots ,\frac{a_n}{q}\},a_i\in \mathbb Z,q\in \mathbb N,(a_1,\ldots ,a_n,q)=1
\]
и рассмотрим \textit{центрировку $a$} \[
	\Lambda_a=\langle \mathbb Z^n,\overline{a}\rangle_{\mathbb Z}=\{\overline{a}l+\overline{b}:l\in \mathbb Z,b\in \mathbb Z^n\}
.\]

\textbf{Пример.} Пусть $n=2,\overline{a}=\{\frac{1}{2},\frac{1}{2}\}$. Тогда получим решётку, порождаемую $\overline{a}$ и $\{0,1\}$.\\

\textbf{Пример.} Пусть $n=2,\overline{a}=\{\frac{1}{2},\frac{1}{3}\}=\{\frac{3}{6},\frac{2}{6}\}$. Тогда $2\overline{a}=\{1,\frac{2}{3}\}$, значит, в решётке есть вектор $\{0,\frac{1}{3}\}$. Аналогично можно получить $\{\frac{1}{2},0\}$.\\

Назовём $\mathcal E$ --- \textit{каноническим базисом $\mathbb R^n$} $n$ векторов вида $\{0,0,0\ldots 0,1,0,0,\ldots ,0\}$, где в каждом векторе $n-1$ координаты нулевые,

\definition{Дефект решётки $d(\mathcal E,\Lambda)$} минимальное количество векторов, которые надо заменить в $\mathcal E$, чтобы получить базис в $\Lambda$ (это минимальное множество может быть не единственным). \textbf{Пример}: $d(\mathcal E_2,\Lambda_{\{\frac{1}{2},\frac{1}{2}\}})=1$, $d(\mathcal E_2,\Lambda_{\{\frac{1}{2},\frac{1}{3}\}})=2$.

\lemma В $\mathbb R^n$ существует $\overline{a}$ такой, что $d(\mathcal E_n,\Lambda_a)=k$ при любом $0\leq k\leq n$.

\proof Рассмотрим вектор $\overline{a}=\{\frac{1}{2},\frac{1}{3},\ldots ,\frac{1}{p_k},0,\ldots ,0\}$. Докажем, что дефект этого вектора равен $k$. Действительно, для любого $j\leq k$ вектор $\overline{a}$ можно умножить на $\prod_{i\neq j}p_i$ и получить вектор, у которого координата $j$ не целочисленная. Значит, в решётке есть какой-то вектор $\{0,0,\ldots ,0,t,0,\ldots ,0\}$, где $t$ находится на месте $j$ и не целое. Значит, нам придётся удалить $\overline{e_j}$. С другой стороны, вектора с номерами $j+1$ и больше можно не удалять.\QEDA\\

Пусть $q$ из канонической записи $\overline{a}$ равно $p_1^{\alpha_1}\cdot \ldots \cdot p_s^{\alpha_s}$. Рассмотрим такие $M_1,\ldots ,M_s\subset\{1,\ldots ,n\}$: $M_i=\{\nu:(a_\nu,p_i)=1\}$ (неформально, это номера тех дробей, у которых числитель не кратен $p_i$). Теперь рассмотрим все такие $S\subset\{1,\ldots ,n\}$, что $\forall i:S\cap M_i\neq\emptyset$ (\textit{систему общих представителей $M_i$})

\theorem $d(\mathcal E,\Lambda_a)=\min\{|S|\}$.\label{cop}

\textbf{Пример.} Если $a=\{\frac{1}{2},\frac{1}{2}\}$, то $q=2,s=1,M_1=\{1,2\}$, тогда минимальная система общих представителей имеет мощность 1.

\textbf{Пример.} Если $a=\{\frac{1}{2},\frac{1}{3}\}$, то $q=6,s=2,M_1=\{1\},M_2=\{2\}$, тогда минимальная система общих представителей имеет мощность 2.

\proof Вначале докажем, что $d\leq|S|$. Заметим, что $d\leq x$ равносильно тому, что $\exists \mathbb R^{n-x}\subset \mathbb R^n$, в котором все точки решётки $\Lambda_a$ целые. Рассмотрим $S=\{\sigma_1,\ldots ,\sigma_\tau\}$ --- систему общих представителей для $M_i$. Пусть $I=\mathcal E\setminus \{\overline{e_i}|i\in S\}$. Рассмотрим такое подпространство $N$ --- $\mathbb R^{n-\tau}$, натянутое на векторы из $I$. Допустим, что точка $l\overline{a}+b$ оказалась в $N$. Мы знаем, что $\forall i\exists j:\sigma_j\in M_i\iff (a_{\sigma_j},p_i)=1$. Мы должны сделать $\tau$ целых чисел, чтобы потом их сделать равными нулю. То есть мы должны домножить на все простые множители знаменателей этих дробей, но там есть все простые множители, значит, мы умножим на общий знаменатель. Значит, целыми станут и все остальные компоненты вектора, тогда и весь вектор целый. Первая часть доказана. Аналогично можно доказать, что если мы нашли подпространство размерности $n-y$, то его антиподпространство размера $y$ порождено векторами из какой-то системы общих представителей.\QEDA\\

\newpage
\theoremn{Минковский} Пусть $\omega\in \mathbb R^n$ выпукло, симметрично относительно начала координат и $Vol\omega>2^n$. Тогда $\omega\cap \mathbb Z^n\setminus O\neq\emptyset$.\label{mink}\\

Обозначим $N_p=|\frac{1}{p}\mathbb Z^n\cap\omega|$. Тогда $\lim_{p\to\infty}\frac{N_p}{p^n}=Vol\omega\implies \exists P\forall p>P:N_p>(2p)^n$ (первое утверждение верно, потому что так работает объём).

\prooft{mink} Назовём точки $\overline{\frac{a}{p}}$ и $\overline{\frac{b}{p}}$ в $\frac{1}{p}\mathbb Z^n$ одинаковыми, если $2\mid\overline{a}-\overline{b}$. По принципу Дирихле существуют две <<одинаковые>> точки. Значит, $\frac{a-b}{2}$ по выпуклости лежит в $\omega$, а это ненулевая целая точка.\QEDA\\

Пусть $\Lambda$ --- решётка в $\mathbb R^n$. Назовём $\det\Lambda$ --- определитель матрицы, составленной из базисных векторов $\Lambda$ (легко показать, что это определение корректное, т.к. не зависит от базиса). Это объём базисного параллелепипеда.

\theorem Пусть $\omega\subset \mathbb R^n,\Lambda\subset \mathbb R^n$, $\omega$ выпукло и симметрично относительно начала координат, кроме того, $Vol\omega>2^n\det\Lambda$. Тогда $w\cap \Lambda\setminus O\neq\emptyset$.

\proof Аналогично \ref{mink}.\QEDA\\

\lemma $\det\Lambda_a=\frac{1}{q}$.\\

Обозначим за $O^n$ (октаэдр в $n$-мерном пространстве) тело $|x_1|+|x_2|+\ldots +|x_n|\leq1$.

\lemma $Vol\ O^n=\frac{2^n}{n!}$.\\

Дальше мы будем рассматривать только те решётки $\Lambda_{\overline{a*}}$, которые пересекаются с $O^n$ только в целых точках. Обозначим $\mathfrak f_c(n)=\frac{cn}{\ln n}\cdot (\ln\ln n)^2$.

\theorem $\max_{\overline{a*}}d(\mathcal E,\Lambda_{\overline{a*}})\leq \mathfrak f_c(n)$ для какого-то $c>0$.\label{target}

\theorem $\max_{\overline{a*}}d(\mathcal E,\Lambda_{\overline{a*}})\geq\mathfrak f_d(n)$ для какого-то $d>0$. \textit{Задача слишком сложная.}\\

\lemma $s\leq n$ ($s$ --- количество простых в каноническом разложении общего знаменателя координат $a$).

\proof Мы знаем, что $\Lambda_a\cap O^n=\{0,\pm \epsilon_i\}$. Значит, $\frac{2^n}{n}=Vol\ O^n\leq 2^n\det\Lambda_a=\frac{2^n}{q}$. Тогда $q\leq n!$.С другой стороны, $q=p_1^{\alpha_1}p_2^{\alpha_2}\ldots p_s^{\alpha_s}\geq s!$ (очевидно). Лемма доказана. \QEDA\\

Пусть $k_i=|M_i|$ --- размеры множеств из \ref{cop}.

\lemma $k_i!\geq p_i$.

\proof Рассмотрим координатные направления с номерами из $M_i$. В подпространстве $\pi$, которое ими образовано, есть $O^{k_i}\subset O^n$. Часть решётки $\Lambda_a\cap\pi$ содержит точки вида $\frac{1}{q'}\cdot \{a_1,\ldots ,a_n\}$, где $q'\geq p_i$ (вектор, сокращённый до нашего подпространства). Тогда $\det (\Lambda_a\cap\pi)\leq \frac{1}{p_i}$. Тогда по теореме Минковского у нас всё получается.\QEDA\\

\lemma $k!\geq p\implies k\geq \frac{\ln p}{2\ln\ln p}$.

\proof Пусть $k<\frac{\ln p}{2\ln\ln p}$. Тогда \[
	k!<k^k=e^{k\ln k}<e^{\frac{\ln p}{2\ln\ln p}\ln k}<e^{\frac{\ln p}{2\ln \ln p}\ln\ln p}=\sqrt p<p
.\QEDA\]

\prooft{target} Мы знаем следующее: $\{M_1,\ldots, M_s\};s\leq n;|M_i|\geq \frac{\ln p_i}{2\ln\ln p_i}$. Ищем $d(\mathcal E,\Lambda_a)=\tau(\{M_1,\ldots ,M_s\})$.

\lemma $\tau(\{M_1,\ldots ,M_s\})\leq \max\{\frac{n}{k},\frac{n}{k}\ln(\frac{sk}{n})\}+\frac{n}{k}+1$, где $k$ таково, что $|M_i|\geq k\forall i$.\label{coptheorem}

\proof См. следующую страницу.

\textbf{Завершение доказательства (через \ref{coptheorem}).} Посмотрим на такие $i$, что $p_i>n$. Все множества, у которых соответствующие $p$ большие, имеют множества мощности больше, чем $k=\frac{\ln n}{\ln\ln n}$. Воспользуемся \ref{coptheorem} для этих множеств. Получится $\tau\leq \frac{2n(\ln\ln n)^2}{\ln n}+\frac{2n\ln\ln n}{\ln n}+1<\frac{3n(\ln\ln n)^2}{\ln n}$. Из других множеств возьмём по одному представителю, их будет максимум $\frac{cn}{\ln n}$, а это ещё меньше.\QEDA\\

\newpage
\proofl{coptheorem} Можно удалить любой набор элементов из слишком больших множеств, чтобы получилось, что $|M_i|=k$. Рассмотрим несколько случаев:

\begin{enumerate}
	\item $\frac{sk}{n}<1\iff s<\frac{n}{k}\implies $ следует из того, что $\tau\leq s$.
	\item $\frac{n}{k}\ln\frac{sk}{n}\geq n\implies $ следует из того, что $\tau\leq n$.
	\item Ни то, ни другое не выполняется. Построим СОП жадным алгоритмом. Будем на $i$-м шаге брать элемент, который лежит в максимальном количестве множеств, которые не содержат первых $i-1$ элементов. Тогда первый элемент лежит хотя бы в $\rho_1=\frac{sk}{n}$ множествах (принцип Дирихле); второй элемент лежит хотя бы в $\rho_2=\frac{(s-\rho_1)k}{n}$ новых множествах; \ldots\\
	\[
		\text{Сделаем }N=\floor{\frac{n}{k}\ln \frac{sk}{n}}+1\text{ шагов}
	.\] Мы сделали хотя бы 1 шаг и не больше, чем $n$. Тогда $s_N$ (количество оставшихся множеств) \[
		s_N=s_{N-1}-\rho_N<s_{N-1}\left(1-\frac{k}{n}\right)<\ldots <s_0\left(1-\frac{k}{n}\right)^n=se^{N\ln(1-\frac{k}{n})}\leq s^{-\frac{k}{n}N}\leq se^{-\frac{k}{n}\frac{n}{k}\ln\frac{sk}{n}}=\frac{sn}{sk}=\frac{n}{k}
	.\] Возьмём из каждого оставшегося множества по элементу, получится утверждение.\QEDA\\
\end{enumerate}

\end{document}
