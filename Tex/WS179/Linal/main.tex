\documentclass[12pt,a4paper]{article}
\usepackage{tpl}
\dbegin[31 января 2020 г.]{Зимняя школа}{Линейная алгебра}{Дориченко Сергей Александрович}

\theorem Есть 101 корова. Если убрать любую, то оставшихся можно разделить на равные по количеству и массе группы. Тогда все коровы равны по массе.\label{cows}\\

\lemma То же, что и \ref{cows}, но веса коров --- целые числа.

\proof Заметим, что можно вычитать из каждого веса $a$ или делить каждый вес на $b$. Вычтем из каждой коровы вес самой маленькой коровы, уведём самую маленькую, получим, что суммарный вес целый. Значит, вес каждой коровы чётный (иначе можно увести нечётную), значит, можно поделить все веса на 2 --- бесконечный спуск.\QEDA\\

Аналогично можно доказать, что если веса рациональные или, например, $\sqrt2\cdot a_i$, то задача тоже решена.

\lemma То же, что и \ref{cows}, но веса коров --- $a_i+b_i\cdot \sqrt2$.\label{irrac}

\proof Пусть $a_1+\ldots +a_{50}=a_{51}+\ldots +a_{100}$. Тогда для рациональной и для иррациональной компоненты всё выполняется.\\

\definition{Линейно-независимое множество над $\mathbb Q$} множество $S=s_1,s_2, \ldots s_k$ такое, что если $s_1r_1+s_2r_2+\ldots +s_kr_k=0$ при $r_i\in\mathbb Q$, то $r_i=0\ \forall i$.

\prooftms{cows}
\prooftm{cows} Возьмём $S=a_1$ и вычеркнем всех коров, веса которых линейно зависимы с $S$. Добавим в $S$ минимальную корову, которая осталась, снова вычеркнем плохих коров и т.п. У нас получится базис на множестве коров над $Q$. Тогда задача решится аналогично \ref{irrac}.\QEDA\\

\prooftm{cows} У нас есть система: $a_1+\ldots +a_{50}-a_{51}-\ldots -a_{100}=0$ и т.п., всего 101 уравнение. Будем решать её подстановкой. Рассмотрим последний момент, когда получилось всё ещё не $0=0$. Заметим, что у нас есть хотя бы 1 свободная переменная, иначе у системы единственное решение. Если свободная переменная только одна, то все веса имеют вид  $a_k\cdot q_i$, где $k$ --- константа, $q_i\in\mathbb Q$. Если их хотя бы две, то проблемы: пусть последнее уравнение имеет вид $a_i=a_{i+1}q_1+a_{i+2}q_2+\ldots $. Возьмём все $a_{i+j}$ целыми, тогда все остальные $a_j$ будут рациональными --- противоречие.\QEDA\\

\lemma Дана система линейных уравнений с рациональными коэффициентами. Известно, что у неё есть решение. Тогда есть и рациональное решение.

\proof Аналогично предыдущему решению. \QEDA\\

\theorem Пусть прямоугольник $a\times b$ разрезан на квадраты. Тогда $\frac{a}{b}\in\mathbb Q$.\label{den}

\proof Обозначим стороны квадратов как $x_i$. Получим много линейных уравнений. У этой системы есть решение в положительных числах. Пусть у неё есть какое-то другое решение.

\lemma По решению можно восстановить картинку.

\proof Разрежем квадрат по каждой вертикальной и каждой горизонтальной линии. Тогда мы получим новую систему с длинами этих отрезков, с помощью них решим задачу. \QEDA

\textbf{Завершение доказательства \ref{den}.} У системы будет либо единственное решение, либо свободные переменные. Рассмотрим два случая:

\begin{itemize}
	\item Решение единственно. Тогда все отрезки равны $a\alpha_i+b\beta_i$, где  $\alpha_i,\beta_i\in \mathbb Q$. Подставим все $x_i$ в уравнения. Тогда либо какое-то уравнение превратится в $ka+lb=0$, тогда $\frac{a}{b}\in \mathbb Q$; либо все уравнения станут $0=0$. Немного изменим  $a,b$ так, что все $x_i$ останутся положительными. Тогда получим, что в каком-то интервале  $a^2\sum\alpha^2_i+2ab\cdot \sum(\alpha_i\beta_i)+b^2\sum\beta^2_i=ab$, т.е. все $\alpha_i$ и $\beta_i$ равны нулю --- противоречие.
	\item Есть свободные переменные. Пусть $t$ --- свободная переменная. Тогда можно менять $t$ на каком-то интервале и получить, что все коэффициенты при $t$ равны 0. Значит, от $t$ ничего не зависит.\QEDA\\
\end{itemize}\newpage

\theorem Пусть прямоугольник разрезан на прямоугольники. У каждого прямоугольника разрезания есть целая сторона. Тогда и у исходного есть целая сторона.\label{rect}

\lemma То же, что и \ref{rect}, но все стороны прямоугольников рациональные.

\proof Это следует из того, что если $a\times b$ разбили на $1\times n$, то $n\mid a$ или  $n\mid b$.\QEDA\\

\prooftms{rect}
\prooftm{rect} Будем считать, что все прямоугольники $1\times a$ (если $n\times a$, то можно разрезать на $n$ прямоугольников). Продлим каждую сторону до пересечения с стороной исходного прямоугольника. Сделаем систему уравнений, причём $a$ и $b$ сделаем неизвестными. Если у неё нет свободных переменных, то все $x_i$ рациональные, это мы умеем решать. Если две свободные переменные, сделаем $a$ и $b$ свободными, у нас $\sum x_i=ab$, противоречие. Значит, свободная переменная одна. Сделаем $a$ свободной и поставим в $a$ несколько близких рациональных чисел. Получим, что при этих $a$ все числа $b$ равны одному и тому же целому числу. Дальше надо додумать, а мы не смогли.\\

\prooftm{rect} Раскрасим плоскость в шахматную раскраску с шагом $\frac{1}{2}$. Тогда у прямоугольника чёрного и белого поровну тогда и только тогда, когда есть целая сторона.\QEDA\\

\prooftm{rect} Для каждого прямоугольника выберем одну из пар целых сторон и положим на них стеклянные трубочки. Тогда на каждом перекрёстке чётное количество трубочек. Значит, этот граф эйлеров. Пустим из угла мышку, она закончит путь в каком-то другом угле. Тогда соответствующая сторона целая.\QEDA\\

\prooftm{cows} Найдём по теореме Кронекера $N$ так, что $\forall\ i: Na_i$ отличается от целого числа не более чем на $\frac{1}{1000}$. Задача закончилась.\QEDA\\

\end{document}
