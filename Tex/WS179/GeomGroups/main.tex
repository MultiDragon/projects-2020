\documentclass[12pt,a4paper]{article}
\usepackage{tpl}
\dbegin[27 января 2020 г.]{Зимняя школа}{Геометрия по Клейну}{Рябичев}

Пусть $X$ --- множество и $\sigma(X)$ --- множество его биекций. Назовём множество $G\in\sigma$ группой, если $f,g\in G\implies f\circ g\in G$ и $f\in G\implies f^{-1}\in G$.

\theorem Любая группа представляется в таком виде.

\textbf{Пример.} Пусть $G=\mathbb Z / 5\mathbb Z$ с операцией сложения. Рассмотрим $X=\mathbb Z / 5\mathbb Z$ и число $k$ переходит в биекцию $r\mapsto r+k$. Очевидно, оно подходит.

\z Найти $|\sigma(\mathbb R^2)|$.\\

\shdr{Список интересных подгрупп $X=\mathbb R^2$}

\begin{itemize}
	\item Параллельные переносы; повороты вокруг нуля.
	\item Собственные движения (сохраняющие ориентацию).
	\item $\forall x,y\in \mathbb R\ |x,y|=|f(x),f(y)|$ --- движения.
	\item Сохраняются прямые и углы --- композиция движения и гомотетии.
	\item $\forall x,y,z\in \mathbb R\ xyz\text{ -- прямая}\implies f(x)f(y)f(z)\text{ -- прямая}$ --- аффинные преобразования.
	\item Непрерывные отображения.
	\item Все биекции --- $\sigma(X)$. 
\end{itemize}\vskip10pt

В списке выше каждая следующая группа --- строгое надмножество предыдущей.\\

\definition{Аффинное преобразование} преобразование, которое переводит фиксированную точку $O$ и два неколлинеарных вектора $\overline{u},\overline{v}$ в точку $f(O)$ и неколлинеарные вектора $f(\overline{u}),f(\overline{v})$, а затем любую точку $A=O+k\overline{u}+l\overline{v}$ переводит в точку $f(O)+k\overline{f(u)}+l\overline{f(v)}$.\\

\lemma При аффинном преобразовании образ прямой --- прямая.

\proof Представим прямую $l=O+\overline{u}+t\overline{w}$. Тогда её образ $f(l)=f(O)+\overline{f(u)}+t\overline{f(w)}$ --- тоже прямая. \QEDA\\

\theorem Любое преобразование, сохраняющее прямые, аффинно.\\

\theorem Любое преобразование, сохраняющее прямые и углы, является поворотной гомотетией (возможно, композицией с осевой симметрией).

\proof Это аффинное преобразование с такими свойствами: $\angle (\overline{f(u)},\overline{f(v)})=\angle (\overline{u},\overline{v})$ и $|f(u)|/|f(v)|=|u|/|v|$. Рассмотрим такую композицию: сначала параллельным переносом переведём $O$ в $f(O)$, затем поворотом $\overline{u}$ в $\overline{f(u)}$, затем, если надо, отразим относительно $\overline{f(u)}$, затем сделаем гомотетию с центром в $f(O)$.\QEDA\\

\theorem Любое преобразование, увеличивающее все расстояния в $c$ раз, является поворотной гомотетией (возможно, композицией с осевой симметрией).

\proof Сделаем гомотетию в $c^{-1}$ раз. Тогда получится преобразование, не изменяющее расстояние, т.е. движение. \QEDA\\

\theorem $\forall f\text{ -- движение }\exists!(t\in T,r\in R):r\circ t=f$. ($T$ --- параллельные переносы, $R$ --- повороты вокруг $O$)

\end{document}
