\documentclass[12pt,a4paper]{article}
\usepackage{tpl}
\dbegin[29 января 2020 г.]{Зимняя школа}{Тест Миллера-Рабина}{Ильинский}

\hdr{В чём смысл}

Можно сделать такой алгоритм шифрования. Пусть $p$ --- большое простое число, $g$ --- первообразный корень по его модулю. Алиса загадывает $a$ и передаёт $g^a$, а Боб $b$ и передаёт $g^b$. Тогда они оба знают число $g^{ab}$ и могут делать по его модулю разные вещи, а сторонний наблюдатель, зная $g^a$ и $g^b$, не может нормально узнать $g^{ab}$ (по крайней мере, не сейчас).

На самом деле $p$ не обязано быть простым --- просто надо, чтобы был какой-то $g$, что его степени дают много разных остатков, т.е. что у него большой порядок. Например, если $m=pq$, оно скорее всего подойдёт.\\

\definition{Система вычетов $Z_m$} множество всех остатков по модулю $m$.

\definition{Приведённая система вычетов $Z^*_m$} множество остатков по модулю $m$, взаимно простых с ним.

\theoremn{Эйлер} $\forall x\in Z^*_m:x^{\phi(m)}=1$.

\definition{Порядок элемента $x$} минимальное такое $k$, что $x^k=1$. Например, порядок первообразного по модулю $m$ равен $\phi(m)$.\\

\theorem Максимальный порядок по модулю $m=pq$ равен $[p-1;q-1]$.\label{prodorders}\\

\lemman{Китайская теорема об остатках} Пусть есть модули $m_1,\ldots,m_n$, которые попарно взаимно просты. Тогда для любых $a_1\in\mathbb Z_{m_1},a_2\in\mathbb Z_{m_2},\ldots $ существует единственное решение $x\in Z_{m_1m_2\ldots m_n}$ такой системы: $\forall i:x\equiv a_i(\text{mod } x_i)$.\\

\prooft{prodorders} Пусть $g_1$ --- первообразный корень по модулю $p$, а $g_2$ --- первообразный корень по модулю $q$. Рассмотрим  $x:x\equiv g_1(p),x\equiv g_2(q)$. Тогда очевидно, что $ord(x)=[p-1;q-1]$.\QEDA\\

\hdr{Как искать большие простые числа}

\definition{Число Кармайкла} непростое $m$ такое, что для всех $a$ выполнено $a^{m-1}\equiv 1\mod m$.

\shdr{Тест Ферма}

Выберем $m$, для которого мы проверяем простоту. Будем брать разные числа $a$ и проверять, что $a^m\equiv a$. К сожалению, тест работает не идеально: существуют числа Кармайкла. Но во-первых, чисел Кармайкла мало, во-вторых, по их модулю существуют числа с большим порядком. С другой стороны, у него довольно большая точность.\\

\theorem Пусть $m$ составное и не число Кармайкла. Тогда хотя бы половина остатков по модулю $m$ не проходят тест Ферма.\footnote{На самом деле, можно доказать, что если $C$ --- множество хороших остатков, то $C\mid m-1$.}

\proof Пусть $b$ не проходит тест, а $C$ --- множество тех чисел, которые проходят. Тогда каждое число вида $bc_i$ не проходит тест, значит, <<плохих>> не меньше, чем <<хороших>>.\QEDA\\

\theorem Составное $m$ --- число Кармайкла тогда и только тогда, когда оно свободно от квадратов для каждого $p\mid m$ верно $p-1\mid m-1$.\label{karma}

\proof Пусть свойство выполняется. Докажем, что $\forall a\in Z^*_m$ выполняется $a^{m-1}\equiv 1\mod m$. Для каждого $p_i$ оно выполняется, поскольку $p_i-1\mid m-1$ и по малой теореме Ферма.

Теперь предположим, что $m$ --- число Кармайкла и что $p^2\mod m$. Тогда пусть $m=p^k\cdot d,k\geq 2,p\nmid d$. Пусть $a\equiv 1+p\mod p^k$ и  $a\equiv1\mod d$. Найдём  $a^{m-1}$:
\[
	a^{m-1}=(1+p)^{m-1}=1+(m-1)p+Mp^2=1-p\mod p^2
.\] Противоречие.

Наконец, докажем, что $p-1\mid m-1$. Это так, потому что можно взять первообразный корень по модулю каждого простого делителя $m$.
\QEDA\newpage

\shdr{Тест Соловея-Штрассена}

Заметим, что если $m$ --- простое, то \[
	a^{\frac{m-1}{2}}=\binom am,
\] т.е. символу Лежандра $a$ по модулю $m$. Оказывается, для составных чисел можно обобщить символ Лежандра (получится символ Якоби). Алгоритм делает следующее. Он берёт большое нечётное число $m$ и случайное число $a$ и проверяет факт выше (для символа Якоби). Скорость такая же, но алгоритм игнорирует числа Кармайкла.\\

\shdr{Тест Миллера-Рабина}

Пусть $p-1=2^s\cdot d$, где $d$ нечётное. Тогда для любого $a$ выполнено либо $a^{2^{s-1}d}=-1$, либо $a^{2^{s-2}d}=-1$, либо \ldots, либо $a^d=\pm 1$.

\theoremn{Миллер, Рабин} Если $m$ составное, то вероятность того, что для $a$ выполнено хотя бы одно из сравнений выше, не больше  $\frac{1}{4}$.

\end{document}
